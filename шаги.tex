%&17pt
\pdfhorigin=8mm
\hsize=\pdfpagewidth \advance\hsize by-2\pdfhorigin
\pdfvorigin=14.2mm
\vsize=270mm

\nopagenumbers
\parindent=0pt
\centerline{\font\bf=omsl10 at 18pt \bf ШАГИ ЦИГУН}

{\bf Общие требования}\par
Все шаги выполняют с максимальным расслаблением. Как будто внутри стоит сосуд с водой, нужно
не расплескать воду и не поднять муть со дна.
Все шаги выполняются играючи, несерьезно, как бы их выполнял ребенок. НЕ нужно относиться к
ним как занятию, или работе.

{\bf 1 Беззаботное скитание (простой)}\par
Нога делает шаг, становится на пятку, разноименная рука выносится вперед, Лао-гун вниз. Это
положение фиксируется далее с другой стороны. Спина прямая, минь-мэнь выпуклый.

{\bf 2 Шаг целостности}\par
Аналогично 1, но ладони вниз, как бы разгребают песок. Фиксировать внимание на связи Лао-гун,
юн-цюань --- земля

{\bf 3 Шаг из Даньтяня}\par
Нога делает шаг, руки слегка поднимают в стороны --- движение от Даньтяня --- открытие (усилие)
Другая нога подставляется к опорной, руки опускаются , движение к Даньтянь --- закрытие
(расслабление).

{\bf 4 Тройной}\par
Шагающая нога ставится на носок напротив стоящей, разноименная рука (лао-гун вниз) начинает
движение вперед (ЯН)
Шагающая нога становится на пятку чуть впереди стоящей, разноименная рука (лао-гун вниз)
продолжает движение вперед. Одноименная рука ладонь к себе опускается к туловищу
Шагающая нога делает шаг, разноименная рука выносится вперед. Лао-гун вниз

{\bf 5 Багуа медленный}\par
Ноги вместе, чуть согнуты, опускаем ци, колено левой ноги начинает толкать вперед правую
ногу. Правая нога выносится вперед, ступня скользит по земле, носок не задирать. Вес
перенести на правую ногу. Затем левая подтягивается к правой , при этом двигается
параллельно земле, скользя по ней. Колени вместе смотрят строго вперед. Абсолютное
расслабление, контроль минь-мэнь
То же с другой стороны.

{\bf 6 Багуа быстрый}\par
Левая нога впереди, ступня параллельно земле, носок не задирать, колени вместе смотрят
вперед. Вес сразу на левой ноге, контроль, как в~5.

\vfil\eject

{\bf 7 Шаг солдата медленный}\par
Левая нога поднимается. Ступня до уровня колена, параллельно земле, нога полусогнута. Рука ---
разноименная поднимается вместе с ногой. Это положение зафиксировать, опустить ци и
расслабиться.
То же с другой стороны.

{\bf 8 Шаг солдата быстрый}\par
Маршировать с отмашкой, колени высоко.

{\bf 9 Шаг назад}\par
Левая нога отшагивает назад, вес на нее, затем на нее делается присед, как на табурет. Затем
корпус поднять.
То же с другой стороны. Расслабиться.

{\bf 10 Легкий шаг}\par
Катимся пятка-носок-пятка. Руки чуть в стороны для равновесия и, чуть помахав, как крылья.
Три быстрых вдоха через нос и три быстрых выдоха через нос
То же с другой стороны.

{\bf 11 Играющий шаг (танец)}\par
Левая нога, носок вниз вперед, затем пятка вниз, не касаясь земли, затем становиться на пятку,
потом подшаг. Руки помахивают при подшаге левая рука удлиняется
То же с другой стороны.

{\bf 12 Журавль ИНЬ}\par
Левая нога поднимается, ступня описывает круг к себе и становится на пятку. Руки
разноименные повтор ног.
То же с другой стороны.

{\bf 13 Шаг ИНЬ ЯН}\par
Левая нога делает пустой шаг на переднюю часть стопы. Правая рука точкой хэ-гу
накладывается на ключицу слева. Левая рука точкой Хэ-гу на минь-мэнь
То же с другой стороны.

{\bf 14 Шаг с разворотом}\par
Левая нога делает пустой шаг на переднюю часть стопы. Корпус влево. Кисти параллельно земле
как бы гладят. Структуру не ломать, подшаг.
То же с другой стороны.

{\bf 15 Шаг впитывания ЦИ-1}\par
Левая нога делает пустой шаг на переднюю часть стопы, корпус влево до уровня плеча. Взгляд
вдоль левой кисти. Вдох. Вдохом впитываем ЦИ окружающей среды через пальцы лао-гун. Подшаг
Все то же с другой стороны.

{\bf 16 Шаг впитывания ЦИ-2}\par
Аналогично 15, но руки согнуты и поднимаются рядом с туловищем на вдохе. Ци впитывается через
лао-гун, юн-цюань и байхуэй (макушка).
\bye
