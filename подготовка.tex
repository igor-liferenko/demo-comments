%&17pt
\pdfhorigin=24mm \hsize=\pdfpagewidth \advance\hsize by-2\pdfhorigin
\pdfvorigin=20mm
\vsize=257mm
\nopagenumbers
\centerline{\bf Упражнение 1.}
\centerline{\bf Для успокоения и концентрации}
\medskip
Сядьте с прямой спиной, дышите диафрагмой. Поставьте руки перед грудью ладонями друг к другу.
Медленно сомкните ладони. Пальцы смотрят вверх. Закройте глаза.

Сосредоточьтесь на точке по центру лба между бровями. Через несколько секунд ваши чувства придут
к гармонии, вы успокоитесь и сможете сосредоточиться для принятия правильного решения в любом
вопросе.
\bigskip
\centerline{\bf Упражнение 2.}
\centerline{\bf Для получения силы и знания от Космоса}
\medskip
Встаньте с прямой спиной, дышите диафрагмой. Поставьте руки перед грудью ладонями вверх. Закройте
глаза.

Сосредоточьтесь на области горла. Оставайтесь в этом положении не более минуты. Затем приложите
ладони к области сердца. Откройте глаза. Вы воссоединили свою связь с Высшим источником, и теперь у вас есть силы и знание, чтобы продолжить свой путь в правильном направлении.

\bigskip
\centerline{\bf Упражнение 3.}
\centerline{\bf Для преодоления негативных эмоций}
\medskip
Встаньте прямо, дышите диафрагмой. Согните руки в локтях, локти прижаты к бокам.

Сильно сожмите кисти в кулаки, затем резко разожмите их, выкидывая напряженные растопыренные
пальцы вперёд.

Повторите несколько раз, в конце оставшись в положении, когда пальцы напряжены и растопырены. Закройте глаза. Сосредоточьтесь на области позвоночника у основания шеи. Оставайтесь в этом положении не более минуты. Затем резко встряхните кисти рук и откройте глаза.

Напряжение вызванное стрессом или негативными эмоциями сброшено, вы почувствуете лёгкость.

\bigskip
\centerline{\bf Упражнение 4}
\centerline{\bf Для защиты от посторонних влияний}
\medskip
Встаньте или сядьте с прямой спиной. Дышите диафрагмой. Поставьте руки перед грудью, ладони
направлены навстречу друг другу. Потрите ладони
друг о друга, затем разведите их в стороны --- просто позвольте им разойтись.

Затем поверните руки ладонями к груди, ребром вверх, кончики пальцев направлены друг к другу.
Закройте глаза. Сосредоточьтесь на области солнечного сплетения. Оставайтесь в этом положении не
более минуты.

Откройте глаза. Теперь, если вам надо сохранить спокойствие, самообладание и чувство защищенности
в сложной ситуации, при общении с другими людьми --- вы во всеоружии.

\bigskip
\centerline{\bf Упражнение 5.}
\centerline{\bf Для успешной самореализации}
\medskip
Встаньте прямо, дышите диафрагмой. Закройте глаза. Поставьте руки перед грудью ладонями к себе, ребром вверх, кончики пальцев обращены к кончикам пальцев противоположной руки.

Сосредоточьтесь на области солнечного сплетения. Затем открывайте глаза и одновременно делайте
кистями рук раскрывающийся жест будто ваши ладони --- это створки ворот, которые расходятся наружу. Сосредоточьтесь на области глаз.

Повторите ещё два раза. Это упражнение особенно
полезно при трудностях самораскрытия --- в общении, работе или творчестве.

Итак, вы не только настроили себя должным образом, но и размяли пальцы, запустили процессы
движения энергии в них --- что вам очень пригодится
при освоении мудр. Поэтому можно без промедления приступать к основному курсу наших занятий.
\bye
