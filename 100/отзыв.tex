Каждый день, в середине дня, мы ждали очередной
Chaque samedi, en fin de journée, on parle méditation et on pratique quelques exercices avec Christophe André. Aujourd'hui, une petite introduction pour bien définir ce qu'est la méditation. En quoi ça consiste ? À quoi ça sert ? Depuis quand ça existe ?

Un été pour méditer
Comment méditer peut nous aider à mieux comprendre le fonctionnement de notre esprit, de notre attention, de nos émotions. Réfléchir sur l'instant présent et sur la spiritualité. Parler des pensées, mais aussi du corps. Comprendre pourquoi et comment méditer peut nous aider à cultiver de meilleurs liens à nous même et aux autres. Voilà ce que Christophe André vous propose chaque semaine.

Publicité
Pour l’instant, je me contente d’écouter le bruit que fait le monde lorsque je n’y suis pas (Christian Bobin, Autoportrait au radiateur)

Comprendre la méditation
C'est gratuit, c'est plutôt facile, c'est bon pour le corps et l’esprit, cela aide à ne plus avoir peur de la mort et à mieux savourer la vie. Si vous en avez peur ou vous vous crispez par d'autres moyens, l'occasion se présente à vous de tout comprendre, de bien vous détendre et, pourquoi pas, de prendre le temps de méditer. Il existe mille et une manières de définir la méditation mais la plus simple est de la comprendre comme un entrainement de l'esprit pour aider à cultiver des capacités psychologiques telles que l'attention, l'équilibre émotionnel, le recul, le calme intérieur, la tolérance, la lucidité sur nous-mêmes et le monde de la psychologie.

C'est un entrainement de l'esprit qui consiste à reconnaître que votre seule volonté ne vous suffit pas pour évoluer. Si vous voulez progresser, il faut y travailler régulièrement et accepter l'idée d'une culture mentale. C'est un entrainement de l'esprit, pour en repousser certaines limites et, finalement, cultiver ce qui nous aide à devenir meilleurs.

Ce n'est pas quelque chose de nouveau, même si elle fait, actuellement, l'objet d'une mode, c'est une vieille histoire : cela fait plus de 2500 ans que l'on médite et mais seulement une trentaine d'années que les recherches scientifiques ont confirmé toutes les vertus de la méditation et ses bénéfices sur notre santé.

Pour qu’une chose soit intéressante, il suffit de la regarder longtemps (Gustave Flaubert, Correspondance)

La ou les méditation(s)
Il en existe plusieurs formes. Toutes les cultures ont développé des traditions méditatives. Cependant, lorsqu'on parle de la méditation, c'est en général de pleine conscience dont il s'agit. Elle est la codification contemporaine d'un ensemble de techniques issues de la tradition bouddhiste..
Известна фраза 'путешествие в десять тысяч километров начинается с первого шага'. Этот марафон стал для кого то первым для кого то очередным этапом продвижения по пути Дао.
Идёт тихая революция в наших западных обществах, с распространением восточных практик и их
подходом:
остановиться, замедлиться, меньше потреблять, наслаждаться существованием, спокойно встречать невзгоды...
Выделить время для практик --- это восстановить эмоциональный баланс, внутреннее спокойствие,
ясность, терпимость и благонамеренность.
В современном мире который ускоряется, забрасывает нас своими экранами и требованиями,
мы часто оказываемся оторванными от самих себя и наших истинных потребностей...
Развитие тела, развитие сознания
Изучать себя, развивать себя, исследовать себя
Ощущать потоки ци
Начался марафон---и как вымело все мысли из головы. Ни о чем я больше не вспоминал и не заботился.
Стал делать - позвонки стали вставать на место.
Слышны периодически щелчки, после которых хотелось выпрямиться и в спине возникало приятное тепло
Интересные задания
Интересные экскурсы в историю
Участие в марафоне помогало удерживать фокус внимания, ибо нельзя отставать!
Двигаться в потоке --- это здорово!
Для занятий этими практиками не нужно никакого оборудования. Всё что нужно--это немного времени и
немного любопытства.
Естественно, и дальше в жизни мы понесём эти практики с собой.
Учились переходить на тонкие ощущения
Невероятное разнообразие практик
Встроить в повседневность
Марафон заставил острее различать грань между тем, что срочно и тем, что важно. Поскольку для того, чтобы найти время для марафона, нужно было чётко расставить приоритеты. Расставленных приоритов я буду придерживаться и впредь.
За время марафона прошла целая жизнь. Много ярких впечатлений:
как я ехал в автобусе и пытался поймать состояние столба, как...
Спасибо идейному вдохновителю марафона Андрею Владимировичу Шираю и всем кто трудился за кадром.
Закончит так же как 1 часть temps a mediter
