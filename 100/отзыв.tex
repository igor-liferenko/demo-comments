%&14pt
Каждый день, на протяжении более чем трёх месяцев, мы ждали очередную практику в
телеграм-канале «100 дней даосских практик». В течение этих ста дней проходил
онлайн-марафон.

Чем же мы занимались в ходе этого марафона?
Мы пытались лучше понять как устроен наш дух, как концентрировать наше внимание,
как управлять нашими эмоциями---посредством регулярной работы с телом.
Марафон также способствовал тому, чтобы задуматься о духовности и
научиться присутствовать в настоящем моменте---ведь мы уделяли внимание не только телу,
но и мыслям. Мы раскрывали новые связи с самими собой и с другими.


Вот что нам помог осуществить этот марафон.

Каждый день, в середине дня, мы ждали очередной
Chaque samedi, en fin de journee, on parle meditation et on pratique quelques exercices avec
 Christophe Andre. Aujourd'hui, une petite introduction pour bien definir ce qu'est la meditation.
 En quoi ca consiste ? A quoi ca sert ? Depuis quand ca existe ?

Un ete pour mediter
Comment mediter peut nous aider a mieux comprendre le fonctionnement de notre esprit, de notre
 attention, de nos emotions. Reflechir sur l'instant present et sur la spiritualite. Parler des
 pensees, mais aussi du corps. Comprendre pourquoi et comment mediter peut nous aider a cultiver
 de meilleurs liens a nous meme et aux autres. Voila ce que Christophe Andre vous propose chaque semaine.

Publicite
Pour l'instant, je me contente d'ecouter le bruit que fait le monde lorsque je n'y suis pas (Christian Bobin, Autoportrait au radiateur)

Comprendre la meditation
C'est gratuit, c'est plutot facile, c'est bon pour le corps et l'esprit, cela aide a ne plus avoir peur de la mort et a mieux savourer la vie. Si vous en avez peur ou vous vous crispez par d'autres moyens, l'occasion se presente a vous de tout comprendre, de bien vous detendre et, pourquoi pas, de prendre le temps de mediter. Il existe mille et une manieres de definir la meditation mais la plus simple est de la comprendre comme un entrainement de l'esprit pour aider a cultiver des capacites psychologiques telles que l'attention, l'equilibre emotionnel, le recul, le calme interieur, la tolerance, la lucidite sur nous-memes et le monde de la psychologie.

C'est un entrainement de l'esprit qui consiste a reconnaitre que votre seule volonte ne vous suffit pas pour evoluer. Si vous voulez progresser, il faut y travailler regulierement et accepter l'idee d'une culture mentale. C'est un entrainement de l'esprit, pour en repousser certaines limites et, finalement, cultiver ce qui nous aide a devenir meilleurs.

Ce n'est pas quelque chose de nouveau, meme si elle fait, actuellement, l'objet d'une mode, c'est une vieille histoire : cela fait plus de 2500 ans que l'on medite et mais seulement une trentaine d'annees que les recherches scientifiques ont confirme toutes les vertus de la meditation et ses benefices sur notre sante.

Pour qu'une chose soit interessante, il suffit de la regarder longtemps (Gustave Flaubert, Correspondance)

La ou les meditation(s)
Il en existe plusieurs formes. Toutes les cultures ont developpe des traditions meditatives. Cependant, lorsqu'on parle de la meditation, c'est en general de pleine conscience dont il s'agit. Elle est la codification contemporaine d'un ensemble de techniques issues de la tradition bouddhiste..
Известна фраза 'путешествие в десять тысяч километров начинается с первого шага'. Этот марафон стал для кого то первым для кого то очередным этапом продвижения по пути Дао.
Идёт тихая революция в наших западных обществах, с распространением восточных практик и их
подходом:
остановиться, замедлиться, меньше потреблять, наслаждаться существованием, спокойно встречать невзгоды...
Выделить время для практик --- это восстановить эмоциональный баланс, внутреннее спокойствие,
ясность, терпимость и благонамеренность.
В современном мире который ускоряется, забрасывает нас своими экранами и требованиями,
мы часто оказываемся оторванными от самих себя и наших истинных потребностей...
Развитие тела, развитие сознания
Изучать себя, развивать себя, исследовать себя
Ощущать потоки ци
Начался марафон---и как вымело все мысли из головы. Ни о чем я больше не вспоминал и не заботился.
Стал делать - позвонки стали вставать на место.
Слышны периодически щелчки, после которых хотелось выпрямиться и в спине возникало приятное тепло
Интересные задания
Интересные экскурсы в историю
Участие в марафоне помогало удерживать фокус внимания, ибо нельзя отставать!
Двигаться в потоке --- это здорово!
Для занятий этими практиками не нужно никакого оборудования. Всё что нужно--это немного времени и
немного любопытства.
Естественно, и дальше в жизни мы понесём эти практики с собой.
Учились переходить на тонкие ощущения
Невероятное разнообразие практик
Встроить в повседневность
Марафон заставил острее различать грань между тем, что срочно и тем, что важно. Поскольку для того, чтобы найти время для марафона, нужно было чётко расставить приоритеты. Расставленных приоритов я буду придерживаться и впредь.
За время марафона прошла целая жизнь. Много ярких впечатлений:
как я ехал в автобусе и пытался поймать состояние столба, как...
Спасибо идейному вдохновителю марафона Андрею Владимировичу Шираю и всем кто трудился за кадром.
Закончит так же как 1 часть temps a mediter
