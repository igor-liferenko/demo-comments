%&10pt
\nopagenumbers
\pdfhorigin=8mm
\hsize=\pdfpagewidth \advance\hsize by-2\pdfhorigin
\pdfvorigin=12mm
\vsize=278mm

\baselineskip=14.4pt
\font\twelverm=omssqi8 at12pt \twelverm
\font\weekF=omssbx12
\font\aboutF=omssi10

\font\yy=umranda
\font\titleF=omssdc10 at20pt
\centerline{\titleF Онлайн-марафон «100 дней даосских практик»}
\vskip2\baselineskip
\leftline{\weekF Неделя 1}
\noindent Даосский рок-н-ролл\par\noindent
{\aboutF Качаемся и крутимся, запускаем потоки ци.}
\vskip 7.2pt plus2.4pt minus2.4pt
\leftline{\weekF Неделя 2}
\noindent Дао офисных креветок\par\noindent
{\aboutF Разминаемся сидя, стоя, в перерывах и не отходя от монитора.}
\vskip 7.2pt plus2.4pt minus2.4pt
\leftline{\weekF Неделя 3}
\noindent Две заставы\par\noindent
{\aboutF Прорабатываем 2 из 3 главных мест в теле.}
\vskip 7.2pt plus2.4pt minus2.4pt
\leftline{\weekF Неделя 4}
\noindent Главное --- хвост!\par\noindent
{\aboutF Сильные и гибкие ноги --- основа здоровья и долголетия.}
\vskip 7.2pt plus2.4pt minus2.4pt
\leftline{\weekF Неделя 5}
\noindent Дао сидя, Дао лёжа\par\noindent
{\aboutF Практики работы с тазом и позвоночником на полу/скамейке.}
\vskip 7.2pt plus2.4pt minus2.4pt
\leftline{\weekF Неделя 6}
\noindent Практики Радости\par\noindent
{\aboutF Классическая даосская алхимия, позитивные эмоции каждый день.}
\vskip 7.2pt plus2.4pt minus2.4pt
\leftline{\weekF Неделя 7}
\noindent Дышим, дышим...\par\noindent
{\aboutF Базовые дыхательные упражнения на каждый день.}
\vskip 7.2pt plus2.4pt minus2.4pt
\leftline{\weekF Неделя 8}
\noindent Круги и восьмёрки\par\noindent
{\aboutF Развиваем пластику и координацию, приучаемся двигаться по округлым траекториям.}
\vskip 7.2pt plus2.4pt minus2.4pt
\leftline{\weekF Неделя 9}
\noindent Предметы силы\par\noindent
{\aboutF Учимся работать с традиционными тренажёрами ТКБИ.}
\vskip 7.2pt plus2.4pt minus2.4pt
\leftline{\weekF Неделя 10}
\noindent Столбовая практика\par\noindent
{\aboutF Основы и база столбовой работы.}
\vskip 7.2pt plus2.4pt minus2.4pt
\leftline{\weekF Неделя 11}
\noindent Столбы пяти стихий
\vskip 7.2pt plus2.4pt minus2.4pt
\leftline{\weekF Неделя 12}
\noindent Размягчение тела\par\noindent
{\aboutF Осваиваем комплекс мастера Хуан Цзы Чена.}
\vskip 7.2pt plus2.4pt minus2.4pt
\leftline{\weekF Неделя 13}
\noindent 12 демонов\par\noindent
{\aboutF Изучаем простые способы открытия меридианов Инь и Ян.}
\vskip 7.2pt plus2.4pt minus2.4pt
\leftline{\weekF Неделя 14}
\noindent Спиральные усилия\par\noindent
{\aboutF Тренируем цзин, развиваем ци, взращиваем шэнь.}
\vskip 14.4pt plus4.8pt minus4.8pt
\line{\hrulefill\quad\lower4pt\hbox{\yy\char"10}\quad\hrulefill}
\vskip 14.4pt plus4.8pt minus4.8pt
\centerline{Итоговая Интеграция}
\centerline{\aboutF Заключительное упражнение марафона.}

\normalbaselines\rm
\vfill\eject

\pageno=1
\font\ornF=drmsym14
\font\adviceF=omfib8 at10pt

\newdimen\fullhsize
\newdimen\gap \gap=10pt % space between columns
\fullhsize=\hsize
\divide\hsize by2
\advance\hsize by-.5\gap
\def\fullline{\hbox to \fullhsize}
\def\makeheadline{%
  \vbox to0pt{%
    \vskip-20pt
    \fullline{\ornF\setbox0=\hbox{\char"D4}%
      \hss\char"D3\hbox{\leaders\copy0\hskip36\wd0}%
      \llap{\special{color push rgb 1 1 1}%
        \vrule height8pt depth1pt width4pt\special{color pop}\kern-1pt}%
      \kern10pt\raise1pt\hbox{\tenit\folio}\kern14pt
      \hbox{\leaders\copy0\hskip35\wd0}\char"D2\hss}%
    \vss}\nointerlineskip}
\let\lr=L \newbox\leftcolumn
\output={%
  \if L\lr
    \global\setbox\leftcolumn=\columnbox
    \global\let\lr=R%
  \else
    \doubleformat
    \global\let\lr=L%
  \fi
  \ifnum\outputpenalty>-20000 \else\dosupereject\fi
}
\def\doubleformat{%
  \shipout\vbox{%
    \makeheadline
    \fullline{\box\leftcolumn\hfil\kern\gap\hfil\columnbox}
  }%
  \advancepageno
}
\def\columnbox{\leftline{\pagebody}}

\parindent=0pt
\long\def\title#1 #2\par{\noindent{\bf#1} $\underline{\strut\hbox{#2}}$ \medskip}
\def\titleX#1 #2 [#3]{\noindent{\bf#1} \boxit{#2}\enspace$^{#3}$\medskip}
%\def\boxit{\rlap{\kern-3pt\vbox to0pt{\vss
%  \hrule\kern-.4pt\hbox{\vrule height11pt\kern10pt\vrule}\kern-.4pt\hrule\kern-2pt}}}
\def\boxit#1{\vbox{\hrule\kern-.4pt\kern\ht\strutbox\kern\dp\strutbox\kern-.4pt\hrule
  \kern-\ht\strutbox\kern-\dp\strutbox\hbox{\vrule\kern3pt\strut#1\kern3pt\vrule}}}

\noindent{\setbox0=\hbox{\bf0}\kern\wd0} $\underline{\strut
  \hbox{Тихое сидение}}$ \medskip

1. Стопы держать параллельно друг другу.\par
2. Расслаблять руки.\par
3. Обращать внимание на влияние солнца на позу и баланс Инь Ян.\par
4. Расслаблять челюсти.

\bigskip

\title 1 Первое даосское вращение

Наклон тела чуть назад.
Не допустимо выпрямлять опорную ногу в колене. Сгибать до комфортного положения.

\bigskip

\title 2 Второе даосское вращение

Стопы неподвижны. И идёт нагрузка на них. Пятки шире, чем носки.
Ноги пружинят.

\bigskip

\title 3 Голова качается дракон улыбается

Делать стоя.

\bigskip

\title 4 Четыре переката головы

Продолжаем идею даоссского рок-н-ролла. Качаем телом, качаем, в частности, головой, качаем,
вращаем. Используем принципы мягкости, осознанного, естественного натяжения и внимания к себе, заботы о себе.
[00:29] Помните, что любая даосская практика может быть как бесконечно полезной и мощно поддерживающей,
оздоравливающей вас, так и вредной, разрушающей, если вы делаете её неправильно, невнимательно, неосознанно.
Сегодня мы делаем вращение головой по кругу. Казалось бы, это очень простое упражнение.
[00:56] И сейчас я вам покажу за 30 секунд. Мы все дружно начнём крутить головой. Не так все просто. Проблема в том,
что когда мы двигаем головой по кругу, у нас либо не происходит полноценного освобождения мышц шеи и плечевого
пояса, здесь у нас задействуется уже плечевой пояс, воротниковые мышцы, либо у нас происходит ретравматизация, то
есть повторное повреждение суставов, позвоночника, значит, соединений позвонков между собой.
[01:39] Возможны микроразрывы связок и так далее. Я не хочу вас пугать, говоря все эти неприятные подробности. Я
просто обращаю внимание и призываю, будьте аккуратны, будьте внимательны. Чтобы было аккуратно, мягко и глубоко,
мы разбиваем круг, по которому у нас катается голова, на 4 сектора. Значит, передний сектор. Голова катается. Спереди,
справа, налево. Боковой сектор справа. Голова катается вперед-назад справа. Задний сектор.
[02:24] Голова катается... Справа налево, сзади. И боковой сектор слева, голова катается вперед-назад, слева. Вот сейчас
мы будем это делать. Значит, когда будете выполнять практику самостоятельно, можно закрывать глаза. Даже
приветствуется идея закрытия глаз.
[02:48] Есть более сложные варианты, кому интересно, кто уже знает простой вариант, напишите об этом в чате нашего
марафона. Я предложу усложнённые, продвинутые варианты этого упражнения. Сейчас берём базовый вариант. Итак,
опускаем голову вниз, отпустили, сильно не давим, не тянем, просто расслабили мышцы. Почувствовали, что голова
повисла и начинаем мягко качать головой вправо-влево.
[03:19] Хорошо, если подбородок будет касаться грудной клетки, но если он не касается, не делайте это специально, не
стремитесь силой заставить голову коснуться грудной клетки. Стремимся к грудной клетке, но это произойдёт
постепенно. Сектор вправо-влево, спереди. Движение медленное, плавное. И при этом без рывков и без насилия. То есть
специально не сдерживаем голову. Она мягко катается, как тяжёлый шар на цепи. Справа, вперёд, назад.
[04:28] Запускается небольшое открытие рта. Поскольку мы расслабляем челюсти, то чуть-чуть приоткрывается рот.
Сзади, справа, налево. Четвертый сектор. Слева, спереди, назад, сзади, наперёд. И теперь ещё раз по секторам. Первый
сектор.
[05:27] Второй сектор. Третий сектор. Четвертый сектор. И теперь отпустили голову и мягко попробовали её прокатить по
кругу непрерывно.
[05:56] Возможно, с первого раза это не получится. Где-то голова застрянет и вам захочется катнуть её в другую сторону.
То же самое в другую сторону. Мы закончили сегодняшнюю практику. Она короче, чем предыдущая, на что обращаю
внимание. Вчера и сегодня я показывал упражнение сидя. Во многом это связано с особенностями видеосъёмки. Важно,
чтобы вы видели вблизи мою голову и как она двигается, и как работают мышцы шеи. Соответственно, эту практику
можно делать стоя. Более того, когда вы будете делать эту практику стоя, обратите внимание, как по-другому работает
позвоночник. Тему того, что голову мы крутим спойлер-спойлер. От низа спины практически от таза. Мы будем
разбирать на итоговом занятии этой недели, который у нас состоится на седьмой день марафона, соответственно, в
пятницу. То есть, эту практику можно делать сидя, но когда её делаешь стоя, эффект чуть-чуть сильнее, чуть-чуть
интенсивнее. Пробуем разные варианты.
Можно делать сидя и стоя.

\bigskip

\title 5 Тигр играет с добычей

\bigskip

\title 6 Третье даосское вращение

\bigskip

\titleX 7 Слушаем ци [1]

\bigskip

\title 8 Тигр и журавль

Та рука, что поддерживает локоть не устаёт. Ее можно просто опереть на грудную клетку. Рука,
которой крутим, должна активно работать, с усилием. Делайте движение так, будто вы на себя и
от
себя двигаете тяжёлый предмет. Прорабатывается вся дельта, нагрузка может быть весьма
большой, по
самочувствию. Если устаёт плечо руки, которая держит локоть и/или рука, которая изображает
клюв и
лапу, больше устаёт не в плече, а в кисти, то чаще менять направление вращения и руку.

\bigskip

\title 9 Проверка карманов

\bigskip

\title 10 Руки-змеи

Для проработки плеч можно в любой стойке, и даже сидя. Если хочется пропустить ци вниз, тогда
дракон лучше.

\bigskip

\title 11 Косое раскрытие плеч

Напряжение, напряжение, максимальное напряжение, отдых.
Дыхание: три коротких выдоха - один длинный.

\bigskip

\title 12 Танцующие змеи

\bigskip

\title 13 Танцующий журавль

\bigskip

\titleX 14 Проработка плеч до таза [2]

\bigskip

\title 15 Проветривание рёбер

\bigskip

\title 16 Лыжник

\bigskip

\title 17 Обезьяна лезет на дерево

\bigskip

\title 18 Приседания

Когда пятки вместе таз больше открывать.
В узкой столбовой стойке (когда стопы на ширине плеч) стопы ставить параллельно.
Если в приседании на одной ноге плохо держу равновесие, значит таз сильно уходит вбок.
Помогает более широкая стойка, тогда таз остаётся между ногами и равновесие лучше.
Допускается чтобы проекция колена выходила за передний край стопы.

\bigskip

\title 19 Качание тазом

При поднятии весь таз немного напрягается (мышцы тазового дна тоже напрягаем), не только
точки куа.
Точки куа --- просто точки контроля.

\bigskip

\title 20 Вращение коленями

Проекция колена может выходить за передний край стопы.
Стопы могут открываться от пола.
Полный круг коленями делать.

\bigskip

\titleX 21 Спонтанные падения [3]

\bigskip

\title 22 Нога-копьё

\bigskip

\title 23 Пролезть в дыру в заборе

\bigskip

\title 24 Перешагнуть забор

\bigskip

\title 25 Восьмёрка ногой

\bigskip

\title 26 Император восходит на трон

Если трудно удержать равновесие на этапе восьмёрки:\par
1. Той рукой, которая пойдет вверх - можно опираться на стену или спинку стула.\par
2. Делать упражнение с минимальной амплитудой, чтобы меньше терять равновесие.\par
3. Делать сразу наклон без предварительной восьмёрки.

\bigskip

\title 27 Чечётка

Полезна для оздоровления коленей и развивает навык координации в ногах.
Желательно чтобы одна нога была впереди, чтобы был перенос веса и чтобы коленный сустав
включился.

\bigskip

\titleX 28 Наблюдаем за походкой [4]

\bigskip

\title 29 Бабочка

\bigskip

\title 30 Шагающий экскаватор

\bigskip

\title 31 Полушпагат

\bigskip

\title 32 Восьмёрка стопой

\bigskip

\title 33 Крокодил греет спину

\bigskip

\title 34 Массаж стоп

\bigskip

\titleX 35 Комплекс сидя/лёжа перед сном [5]

\bigskip

\title 36 Дыхание неба и земли

\bigskip

\title 37 Тряска

\bigskip

\title 38 Большая весёлая обезьяна

\bigskip

\title 39 Игра на гитаре

\bigskip

\title 40 Журавль в медитации

\bigskip

\title 41 Вихрь

\bigskip

\titleX 42 Цзю ян шень гун [6]

\bigskip

\title 43 Кайхэ сжатие/расширение

\bigskip

\title 44 Кайхэ всплытие/погружение

\bigskip

\title 45 Кайхэ откат/накат

\bigskip

\title 46 Пробуждение дракона

Делать с усилием.
Варианты на животе и на спине.
Делать с закрытыми глазами.
Максимальные границы спирали --- диафрагма и лобковая кость.
Варианты продавливать, прикасаться, на расстоянии от тела.
Тело участвует.
Можно делать сидя, но важно положение ног: ноги должны быть достаточно широко расставлены,
как в стойке, так чтобы были такие же ощущения как стоя.
Лучше всего дышать естественно. Иногда в такт движению рук, в иногда и не привязываясь.
Принципиально то, что внимание идёт вместе с руками. А за вниманием --- ци.

Массаж живота можно делать как по часовой стрелке так и против.
{\it В беседе с учителем упоминалось, что для кишечника полезнее когда
сначала против часовой стрелки а потом
по часовой.}

\bigskip

\title 47 Четыре вдоха один выдох

Можно делать откинувшись на кресле или скамейке с высокой спинкой.

\bigskip

\title 48 Дыхание пятками

Если возникает напряжение в стопах, когда на пятках приподнимаем носки, то чтобы не терять
равновесие, начать с того что при перекате на пятки не сильно отрывать носки от земли -
достаточно переноса веса. В дальнейшем тело само найдет оптимальную амплитуду.

\bigskip

\titleX 49 Управление дыханием [7]

\bigskip

\title 50 Круги руками

На уровне живота.
На уровне грудной клетки.
На уровне головы. \par
Свободная рука на животе, бедре, пояснице. \par
В трёх плоскостях: фронтальная, горизонтальная, сагитальная (одной ногой можно вышагнуть
вперёд).
\par Сидя, стоя, на ходу. \par
Можно с проворотом кисти, можно без.
Попробовать стоя по шею в воде.

\bigskip

\title 51 Шёлковая нить

За 1 секунду рука проходит соответственно 10 минутам минутной стрелки часов.

\bigskip

\title 52 Мужская восьмёрка коленями

По началу допустимо чтобы переносился вес на стопах - стопы встают на внешнее ребро. В
дальнейшем стремиться чтобы стопы стояли полностью по всей поверхности.

\bigskip

\title 53 Женская восьмёрка коленями

По началу допустимо чтобы переносился вес на стопах - стопы встают на внутреннее ребро. В
дальнейшем стремиться чтобы стопы стояли полностью по всей поверхности.

\bigskip

\title 54 Танцующие чаши

От себя; сначала доводить до плеча/обратно, прочувствовать как работает плечевой сустав,
локоть,
ноги, позвоночник, как распределяется вес, потом полную спираль. Аналогично к себе.

Непарные спирали от себя, непарные спирали к себе (когда одна рука наверху --- другая в низу).
С поворотом корпуса при парных к себе.

С поворотом корпуса при непарных (разворот в сторону руки которая идёт к себе когда она
проходит
возле пояса; разноимённая нога выставляется вперёд когда рука в крайней верхней точке).

\bigskip

\title 55 Два дракона играют с жемчужиной

Не желательно чтобы плечи раскачивались из стороны в сторону, больше вперёд назад.

\bigskip

\titleX 56 Танец тайчи [8]

\bigskip

\title 57 Шар тайцзи --- вращение на столе

Очень хорошо делать вращение Лаогун.

\bigskip

\title 58 Шар тайцзи --- вращение у стенки

Мяч должен выскакивать --- в этом смысл. Учимся дозировать усилие, «приклеиваться»
к мячу и «следовать» за ним.

\bigskip

\title 59 Тайцзи Бань

Выполнить «танец тайчи» (№56).

\bigskip

\title 60 Наматываем ленту

Сильно не натягивать, лента должна быть чуть провисшей.

\bigskip

\title 61 Тайцзи Бо

\bigskip

\title 62 Шары здоровья

\bigskip

\titleX 63 Могун [9]

\bigskip

\title 64 Узкий столб

\bigskip

\title 65 Динамический столб

\bigskip

\title 66 Четыре опоры в столбе

На что опирается тело, какие опоры мы используем, что даёт нашему телу ощущение
стабильности, устойчивости, укоренённости, сбалансированности? Эта практика даст вам
некоторое
понимание этих опор. Важный момент это то, что сегодня мы будем говорить не о физических
опорах, а о некотором внутреннем ощущении, внутренних опорах, которые можно и
нужно прочувствовать стоя столбом.

1) Начинаем с обычной столбовой позиции, только не поднимая руки вверх.
Ноги стоят на ширине плеч. Таз
чуть осажен. Тело раскрепощённое. Макушка всплывает вверх.
Руки на уровне таза.
В этом положении надавливаем руками вниз таким образом, чтобы
большие пальцы были на боковом шве, ладони смотрели вниз, а все пальцы вперёд.
В этом положении можно закрыть глаза, сосредоточиться на ощущении оси, которая
проходит через центры тазовых суставов и является
нашей нижней опорой. По сути, если провести эту ось сквозь всё тело, то она пройдёт
через область
малого таза в районе между промежностью и линией Шэньцюэ-Минмэнь.
Почувствуйте ощущение в этой области. Поймайте это состояние ---
образ мощной горизонтальной оси, которая удерживает ваше тело,
на которую можно буквально сесть.

2) Когда возникнет чувство укоренённости, отпустите руки, так чтобы они немного всплыли вперёд
на уровень живота.
Естественным образом, без
напряжения. Задача почувствовать, как тело опирается на поясницу. Для этого мы немножко
надавливаем руками спереди на воображаемую подушку, которая дальше толкает нас в живот, а
живот мягко опирается на поясницу. Возникает опора спереди-назад в области живота.
Постоять в этом положении некоторое время.

3) Ещё даём возможность рукам всплыть повыше.
Руки приходят на уровень сердца. Это как раз та самая область где руки часто находятся
в стандартном столбе. Важно
здесь то, что мы используем руки для того, чтобы опять почувствовать опирание. В данном
случае опора --- это наши лопатки. За счёт того, что руки находятся спереди,
округляем лопатки
и ощущаем как тело опирается на лопатки сзади на спине.
Итак, у нас есть опора в тазу, у нас есть опора в пояснице, сейчас у нас
третья опора --- это лопатки. За счёт лопаток спина чуть-чуть округляется, уходят межрёберные
различные невралгии, уходят боли в области сердца, сдавленность какая-то, дискомфорт в
области сердца. И плечевой пояс в целом округляется, освобождается от напряжения.

4) Идём дальше вверх. Руки опираются на плечи сверху.
Если расположить кисти рук над плечевыми суставами, они должны лечь на надлопаточные
области сверху.
Как будто вы удерживаете на каждом плече по большому одеялу, скатанному в рулон.
При
этом сохраняем осаженность всего тела и все предыдущие опоры. Если руки в этом положении
сильно устают, можно их немножко опустить вниз.

Из этого положения обратно возвращаемся в третью опору.
Переходим ко второй опоре --- возвращаем руки на уровень живота.
Затем переходим к первой опоре --- возвращаем руки на уровень таза и
опускаем руки --- стоим в естественном столбовом положении без
напряжения рук. Наблюдаем за всем телом, как оно выстроилось, 
как и на что оно опирается, где оно устойчиво, где оно не устойчиво,
просто присутствуем в этом состоянии.
В конце можно опять по держать руки возле живота.
Можно поделать «Пробуждение дракона».

\bigskip

\title 67 Широкий столб (оздоровительная свая)

Эта практика позволяет максимально отдыхать и расслабляться. Несмотря на
общий настрой на раскрепощение, на расслабление, узкий столб не всегда даёт
эффект быстрого отдыха, расслабления, потому что нужно много фокусов внимания
распределить по телу, много частей тела осознать и прочее. То есть это всё-таки так
или иначе работа.
Ну, в конце концов, {\it гун\/} --- слово, которое переводится как «работа». И поэтому есть
специальная практика, когда вы хотите стоять столбом, но не хотите сильно работать. И этому
есть вполне веские основания. Ну, например, у вас плохое самочувствие. Или вы, например,
хотите попрактиковать прямо перед сном, чтобы буквально рухнуть в постель и заснуть.
И вам точно не хочется перед сном никакой серьёзной работы. Или у вас некоторые недомогания:
простуда, температура (до 38$^\circ$). Но тем не менее хочется
попрактиковать, но не хочется совсем напрягаться. Вот для всех этих ситуаций идеально
подходит практика широкого столба.

Отличается он двумя моментами от того, что мы делали до этого:

1) Мы садимся чуть шире.
Ноги у нас не на ширине плеч, а полторы-две ширины плеч.
Обнимаю ногами большой шар, как будто сижу на лошади или на большой трубе, на дереве.
Передние края стоп направлены в стороны под 45$^\circ$.

2) Руки располагаю перед собой на уровне живота.

Эта поза называется позой всадника.
В этом положении максимально раскрепощаемся, и по мере выполнения практики, ваше тело
начинает чуть-чуть оседать вниз. Обращайте внимание на чувство комфорта. Если ваши ноги
начинают уставать, можете немножко привстать. И наоборот, если вы чувствуете, что сидите
высоко, можно сесть пониже. Найдите ту высоту, на которой тело сможет
расслабиться.

Стекаем вниз, раскрепощаемся, сбрасываем напряжение, заземляемся. Всю лишнюю, всю негативную
энергию отдаём в землю. Избавляемся от вредной, грязной, застарелой ци.

На выходе можно не вставать, если вы делаете это перед сном, или если вы чувствуете себя
нездоровыми, то можно прямо из этого положения мягко лечь на кровать, уснуть, отдохнуть. Если
вы хотите продолжать какую-то активность после такого столба, мягко соберитесь, вернитесь в
узкий столб. Можно покрутиться, покачаться, встряхнуть руками и ногами и продолжить те дела,
которыми вы занимались до столбовой практики. Надеюсь, что вам удалось поймать вот это
особенное состояние мягкости и растекания тела в широком столбе.

\bigskip

\title 68 Низкий столб (поза всадника)

Начинаем с
узкого столба (стопы на ширине плеч). Если покажется, что в узком столбе не комфортно, можно
сесть чуть пошире.
Стопы параллельно.
Стоим в столбе, руки держим перед собой, на выдохе опускаемся вниз,
чуть-чуть подавая руки вперёд, как будто через
точки Лаогун (точки сброса) выходит из тела напряжение. Когда вы садитесь вниз,
ваше тело естественным образом напрягается.
Попробуйте вывести это напряжение через ладони, через точки Лаогун. Садимся до положения,
когда бёдра горизонтально полу. В принципе можно сесть и ниже, но долго тогда сидеть не
надо. По сути «низкий столб» --- это приседание в столбе. Вопрос в том, в какой момент вы
остановитесь и насколько быстро вы будете делать это приседание.

Ещё раз. Сели, пошли вниз. Мягко встали вверх.

То же самое можно попробовать с чуть большей задержкой в нижней позиции,
чуть большим раскрытием таза, расширением стойки. Опускаться до позиции бёдра параллельно
полу, не уходить совсем вниз.
Стараемся контролировать вертикальную ось, то есть не ложимся вперёд, держим спину как
можно более вертикально.

Если вы сделаете эту практику 5 раз, это хорошая ежедневная разминка.
Она даёт эффект оживления всего тела. Очень полезно делать особенно утром или в середине
дня, когда нужно сбросить какую-то сонливость, усталость. Дать телу дополнительный заряд
бодрости. Если вы делаете эту практику 10 раз, это уже такая хорошая среднего уровня
интенсивности тренировка. Соответственно, там уже идёт и кардионагрузка, и
большая внутренняя работа по распределению напряжения и прочее. Выполнение этого
упражнения свыше 10 раз подряд рекомендуется только здоровым людям. Если есть какие-то
хронические заболевания, лучше ограничиться выполнением этого упражнения не более 10 раз в
день. Следите за равновесием, следите за геометрией тела и за внутренним балансом и
спокойствием.

Пятки не должны отрываться.

\bigskip

\title 69 Столбовые приседания

Столбовые приседания --- это быстрый вариант выполнения упражнения «низкий столб»
с акцентом на координацию работы всего тела и с акцентом на сохранение внутреннего
какого-то спокойствия и баланса. Если физически
тяжело приседать максимально вниз, то можно делать это упражнение, опираясь руками на
спинку стула или опираясь руками на край стола.

Начинаем с узкого столба (стопы на ширине плеч).
Стопы параллельно.
Поймали столбовое состояние.
Можно немножко попружинить, расслабиться. Максимально расслаблено, мягко, спокойно сесть.
Внизу просто свисаем вниз. Чтобы встать, поджимаем таз. То есть, подъем вверх мы
осуществляем через поджимание таза.
Как будто нас что-то подталкивает под ягодицы. И вот это поджимание таза запускает волну, на
которой мы встаём.

Если хочется добавить ещё эффект дыхательный и проработку плечевого пояса, то можно делать
так: вдох\\сели\\выдох, вдох\\встали\\выдох, вдох\\сели\\выдох, вдох\\встали\\выдох.

Обязательно подбираем таз, подворачиваем его под себя и только потом встаём.

Столбовая практика
приседаний предполагает, что вы садитесь и стекаете вниз, буквально свисаете с костей.

Эта позиция очень популярна когда человек хочет просто отдохнуть, расслабиться, но сесть
некуда, и человек садится просто на корточки. То есть это очень естественная позиция, в ней
тело может, должно отдохнуть, расслабиться, но для этого надо настроиться на это состояние
расслабления. И когда вы встаёте из этой естественной позиции, следите за тем, чтобы тело не
перекашивалось. Ровненько сели вниз, ровненько встали вверх без наклонов и перекосов.

Пятки не должны отрываться.

\bigskip

\titleX 70 Обнимаем деревья [10]

\bigskip

\title 71 Уцзи

\bigskip

\title 72 Деревянный столб

\bigskip

\title 73 Металлический столб (саньтиши)

Ставим ноги широко. Полторы-две ширины плеч.
И дальше поворачиваемся влево. Поворачиваем себя руками. И вслед за руками немножко
поворачиваются ноги. Можно начать с деревянного столба. Сели в деревянный столб, только чуть
пошире. И повернулись влево.
Давим руками. При этом у нас присутствуют так называемые девять округлостей: 1. согнутые
ноги, 2. округлый таз, 3. округлая спина, 4. округлые плечи, 5. округлые локти, 6. округлые
запястья, 7. округлые пальцы. Все тело скруглённое. Стоим равномерно на двух ногах. Давим
руками, толкаемся спиной. Возникает такой распор в теле.
Обращаем внимание на то, чтобы у нас были сильно натянуты мышцы во всем теле. Ощущение
натяжения мышц, ощущение таких звенящих мышц. И при этом все суставы максимально округлены.
Единственное исключение у нас шея. Шея мягко вытягивается вверх, так же как в деревянном
столбе макушка тянется вверх.
Могут быть болезненные ощущения в области стоп-голеностопа. На эти места идёт большая
нагрузка. Постепенно стоп-голеностоп расслабиться.
Постояли. Дальше. Через верх вдох-выдох. Стопы повернулись вправо, но не полностью, на
45--60$^\circ$, а руки повернулись вправо на 90$^\circ$.

Если вас сильно беспокоят какие-то ощущения в теле, можете чаще менять стойку, но желательно
выстоять на каждой ноге хотя бы под 2 минуты.

\bigskip

\title 74 Земляной столб (бань-мабу)

\bigskip

\title 75 Усмирение дракона

Стихия воды --- это стихия мягкая и текучая.
Соответственно, позиция водяного столба --- это позиция, в которой мы не фиксируем тело на
одном месте, постоянно чуть-чуть двигаемся, ищем удобные и комфортные положения для всех
частей тела. Базовая метафора: представьте себе огромного водяного дракона,
который может вылетать из воды, летать по воздуху и так далее. И вот ваша задача --- подхватить
телом эту огромную летающую змею и, соответственно, её покорить, подчинить себе, подружиться
с ней и летать вместе с ней.

Исходное положение --- стандартная столбовая стойка. Поворачиваемся на 180$^\circ$. Одна нога
стоит полностью на стопе, вторая нога стоит на носке, пятка в воздухе. Передняя нога, на ней
практически 80% веса, задняя нога 20% веса, коленка опирается на икроножную мышцу передней
ноги. Получается полуприсед.
Задняя нога под углом относительно вертикали, примерно~30$^\circ$ и упирается пяткой.
Под этим же углом находится
наш корпус, голова и обе руки.
Сидим в наклоне вбок, голова тоже под углом.
При этом всё тело скручивается винтом по оси. Получается винт от стоп и до
макушки.
Одна рука толкает вверх-вперёд, другая --- назад-вниз.

Вначале потренировать переход в эту позицию отдельно, чтобы
делать это уверенно,
комфортно и без потери равновесия. В одну и в другую сторону.

Максимально расслабляемся,
раскрепощаемся, убираем напряжение с позвоночника, все тело мягко провисает. Ни в коем
случае не напрягаемся, сидим свободно, спокойно. Чуть-чуть можно покачиваться, искать
комфортное положение.

\bigskip

\title 76 Огненный столб

\bigskip

\titleX 77 Переходы пяти стихий [11]

\bigskip

\title 78 Два потока

\bigskip

\title 79 Раскрепощаем руки

Данное упражнение --- разновидность первого даосского вращения.
Здесь делается акцент на естественном раскачивании,
специально руками не машем. Ну и стопы можно приклеить к полу, чтобы устойчивее было.

\bigskip

\title 80 Собираем ци

\bigskip

\title 81 Раскрепощаем плечи

\bye\bigskip

\adviceF
Время после 21.00 в даосской традиции считается очень полезным для практики.
В это время мозг у человека обычно переходит в режим отдыха, активность сознания
снижается и меньше мешает --- практика становится более глубокой.

\bye
