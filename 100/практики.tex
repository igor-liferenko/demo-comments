%&10pt
\nopagenumbers
\pdfhorigin=8mm
\hsize=\pdfpagewidth \advance\hsize by-2\pdfhorigin
\pdfvorigin=12mm
\vsize=278mm

\baselineskip=14.4pt
\font\twelverm=omssqi8 at12pt \twelverm
\font\weekF=omssbx12
\font\aboutF=omssi10

\font\yy=umranda
\font\titleF=omssdc10 at20pt
\centerline{\titleF Онлайн-марафон «100 дней даосских практик»}
\vskip2\baselineskip
\leftline{\weekF Неделя 1}
\noindent Даосский рок-н-ролл\par\noindent
{\aboutF Качаемся и крутимся, запускаем потоки ци.}
\vskip 7.2pt plus2.4pt minus2.4pt
\leftline{\weekF Неделя 2}
\noindent Дао офисных креветок\par\noindent
{\aboutF Разминаемся сидя, стоя, в перерывах и не отходя от монитора.}
\vskip 7.2pt plus2.4pt minus2.4pt
\leftline{\weekF Неделя 3}
\noindent Две заставы\par\noindent
{\aboutF Прорабатываем 2 из 3 главных мест в теле.}
\vskip 7.2pt plus2.4pt minus2.4pt
\leftline{\weekF Неделя 4}
\noindent Главное --- хвост!\par\noindent
{\aboutF Сильные и гибкие ноги --- основа здоровья и долголетия.}
\vskip 7.2pt plus2.4pt minus2.4pt
\leftline{\weekF Неделя 5}
\noindent Дао сидя, Дао лёжа\par\noindent
{\aboutF Практики работы с тазом и позвоночником на полу/скамейке.}
\vskip 7.2pt plus2.4pt minus2.4pt
\leftline{\weekF Неделя 6}
\noindent Практики Радости\par\noindent
{\aboutF Классическая даосская алхимия, позитивные эмоции каждый день.}
\vskip 7.2pt plus2.4pt minus2.4pt
\leftline{\weekF Неделя 7}
\noindent Дышим, дышим...\par\noindent
{\aboutF Базовые дыхательные упражнения на каждый день.}
\vskip 7.2pt plus2.4pt minus2.4pt
\leftline{\weekF Неделя 8}
\noindent Круги и восьмёрки\par\noindent
{\aboutF Развиваем пластику и координацию, приучаемся двигаться по округлым траекториям.}
\vskip 7.2pt plus2.4pt minus2.4pt
\leftline{\weekF Неделя 9}
\noindent Предметы силы\par\noindent
{\aboutF Учимся работать с традиционными тренажёрами ТКБИ.}
\vskip 7.2pt plus2.4pt minus2.4pt
\leftline{\weekF Неделя 10}
\noindent Столбовая практика\par\noindent
{\aboutF Основы и база столбовой работы.}
\vskip 7.2pt plus2.4pt minus2.4pt
\leftline{\weekF Неделя 11}
\noindent Столбы пяти стихий
\vskip 7.2pt plus2.4pt minus2.4pt
\leftline{\weekF Неделя 12}
\noindent Размягчение тела\par\noindent
{\aboutF Осваиваем комплекс мастера Хуан Цзы Чена.}
\vskip 7.2pt plus2.4pt minus2.4pt
\leftline{\weekF Неделя 13}
\noindent 12 демонов\par\noindent
{\aboutF Изучаем простые способы открытия меридианов Инь и Ян.}
\vskip 7.2pt plus2.4pt minus2.4pt
\leftline{\weekF Неделя 14}
\noindent Спиральные усилия\par\noindent
{\aboutF Тренируем цзин, развиваем ци, взращиваем шэнь.}
\vskip 14.4pt plus4.8pt minus4.8pt
\line{\hrulefill\quad\lower4pt\hbox{\yy\char"10}\quad\hrulefill}
\vskip 14.4pt plus4.8pt minus4.8pt
\centerline{Итоговая Интеграция}
\centerline{\aboutF Заключительное упражнение марафона.}

\normalbaselines\rm
\vfill\eject

\pageno=1
\font\ornF=drmsym14
\font\adviceF=omfib8 at10pt

\newdimen\fullhsize
\newdimen\gap \gap=10pt % space between columns
\fullhsize=\hsize
\divide\hsize by2
\advance\hsize by-.5\gap
\def\fullline{\hbox to \fullhsize}
\def\makeheadline{%
  \vbox to0pt{%
    \vskip-20pt
    \fullline{\ornF\setbox0=\hbox{\char"D4}%
      \hss\char"D3\hbox{\leaders\copy0\hskip36\wd0}%
      \llap{\special{color push rgb 1 1 1}%
        \vrule height8pt depth1pt width4pt\special{color pop}\kern-1pt}%
      \kern10pt\raise1pt\hbox{\tenit\folio}\kern14pt
      \hbox{\leaders\copy0\hskip35\wd0}\char"D2\hss}%
    \vss}\nointerlineskip}
\let\lr=L \newbox\leftcolumn
\output={%
  \if L\lr
    \global\setbox\leftcolumn=\columnbox
    \global\let\lr=R%
  \else
    \doubleformat
    \global\let\lr=L%
  \fi
  \ifnum\outputpenalty>-20000 \else\dosupereject\fi
}
\def\doubleformat{%
  \shipout\vbox{%
    \makeheadline
    \fullline{\box\leftcolumn\hfil\kern\gap\hfil\columnbox}
  }%
  \advancepageno
}
\def\columnbox{\leftline{\pagebody}}

\parindent=0pt
\catcode`\@=11
\def\understrut#1{$\setbox0=\hbox{#1} \dp0=\dp\strutbox \m@th \underline{\box0}$}
\catcode`\@=12
\long\def\title#1 #2\par{\noindent{\bf#1} \understrut{#2} \medskip}
\long\def\titleX#1\par{\noindent\boxit{#1}\medskip}
%TODO: check that \boxit puts bottom line at the same level as \understrut
%\def\boxit{\rlap{\kern-3pt\vbox to0pt{\vss
%  \hrule\kern-.4pt\hbox{\vrule height11pt\kern10pt\vrule}\kern-.4pt\hrule\kern-2pt}}}
\def\boxit#1{\vbox{\hrule\kern-.4pt\kern\ht\strutbox\kern\dp\strutbox\kern-.4pt\hrule
  \kern-\ht\strutbox\kern-\dp\strutbox\hbox{\vrule\kern3pt\strut\TENRM#1\kern3pt\vrule}}}

\titleX Тихое сидение

1. Стопы держать параллельно друг другу.\par
2. Расслаблять руки.\par
3. Обращать внимание на влияние солнца на позу и баланс Инь Ян.\par
4. Расслаблять челюсти.

\bigskip

\title 1 Первое даосское вращение

Наклон тела чуть назад.
Не допустимо выпрямлять опорную ногу в колене. Сгибать до комфортного положения.

\bigskip

\title 2 Второе даосское вращение

Стопы неподвижны. И идёт нагрузка на них. Пятки шире, чем носки.
Ноги пружинят.

\bigskip

\title 3 Голова качается дракон улыбается

Делать стоя.

\bigskip

\title 4 Четыре переката головы

Продолжаем идею даоссского рок-н-ролла. Качаем телом, качаем, в частности, головой, качаем,
вращаем. Используем принципы мягкости, осознанного, естественного натяжения и внимания к себе, заботы о себе.
Помните, что любая даосская практика может быть как бесконечно полезной и мощно поддерживающей,
оздоравливающей вас, так и вредной, разрушающей, если вы делаете её неправильно, невнимательно, неосознанно.
Сегодня мы делаем вращение головой по кругу. Казалось бы, это очень простое упражнение.
И сейчас я вам покажу за 30 секунд. Мы все дружно начнём крутить головой. Не так все просто. Проблема в том,
что когда мы двигаем головой по кругу, у нас либо не происходит полноценного освобождения мышц шеи и плечевого
пояса, здесь у нас задействуется уже плечевой пояс, воротниковые мышцы, либо у нас происходит ретравматизация, то
есть повторное повреждение суставов, позвоночника, значит, соединений позвонков между собой.
Возможны микроразрывы связок и так далее. Я не хочу вас пугать, говоря все эти неприятные подробности. Я
просто обращаю внимание и призываю, будьте аккуратны, будьте внимательны. Чтобы было аккуратно, мягко и глубоко,
мы разбиваем круг, по которому у нас катается голова, на 4 сектора. Значит, передний сектор. Голова катается. Спереди,
справа, налево. Боковой сектор справа. Голова катается вперед-назад справа. Задний сектор.
Голова катается... Справа налево, сзади. И боковой сектор слева, голова катается вперед-назад, слева. Вот сейчас
мы будем это делать. Значит, когда будете выполнять практику самостоятельно, можно закрывать глаза. Даже
приветствуется идея закрытия глаз.
Есть усложнённые, продвинутые варианты этого упражнения: вместо кругов крутить макушкой восьмёрки в
разных плоскостях.
Сейчас берём базовый вариант. Итак,
опускаем голову вниз, отпустили, сильно не давим, не тянем, просто расслабили мышцы. Почувствовали, что голова
повисла и начинаем мягко качать головой вправо-влево.
Хорошо, если подбородок будет касаться грудной клетки, но если он не касается, не делайте это специально, не
стремитесь силой заставить голову коснуться грудной клетки. Стремимся к грудной клетке, но это произойдёт
постепенно. Сектор вправо-влево, спереди. Движение медленное, плавное. И при этом без рывков и без насилия. То есть
специально не сдерживаем голову. Она мягко катается, как тяжёлый шар на цепи. Справа, вперёд, назад.
Запускается небольшое открытие рта. Поскольку мы расслабляем челюсти, то чуть-чуть приоткрывается рот.
Сзади, справа, налево. Четвертый сектор. Слева, спереди, назад, сзади, наперёд. И теперь ещё раз по секторам. Первый
сектор.
Второй сектор. Третий сектор. Четвертый сектор. И теперь отпустили голову и мягко попробовали её прокатить по
кругу непрерывно.
Возможно, с первого раза это не получится. Где-то голова застрянет и вам захочется катнуть её в другую сторону.
То же самое в другую сторону. Мы закончили сегодняшнюю практику. Она короче, чем предыдущая, на что обращаю
внимание. Вчера и сегодня я показывал упражнение сидя. Во многом это связано с особенностями видеосъёмки. Важно,
чтобы вы видели вблизи мою голову и как она двигается, и как работают мышцы шеи. Соответственно, эту практику
можно делать стоя. Более того, когда вы будете делать эту практику стоя, обратите внимание, как по-другому работает
позвоночник. Тему того, что голову мы крутим спойлер-спойлер. От низа спины практически от таза. Мы будем
разбирать на итоговом занятии этой недели, который у нас состоится на седьмой день марафона, соответственно, в
пятницу. То есть, эту практику можно делать сидя, но когда её делаешь стоя, эффект чуть-чуть сильнее, чуть-чуть
интенсивнее. Пробуем разные варианты.

\bigskip

\title 5 Тигр играет с добычей

\bigskip

\title 6 Третье даосское вращение

\bigskip

\title 7 Слушаем ци

\bigskip

\title 8 Тигр и журавль

Та рука, что поддерживает локоть не устаёт. Ее можно просто опереть на грудную клетку. Рука,
которой крутим, должна активно работать, с усилием. Делайте движение так, будто вы на себя и
от
себя двигаете тяжёлый предмет. Прорабатывается вся дельта, нагрузка может быть весьма
большой, по
самочувствию. Если устаёт плечо руки, которая держит локоть и/или рука, которая изображает
клюв и
лапу, больше устаёт не в плече, а в кисти, то чаще менять направление вращения и руку.

\bigskip

\title 9 Проверка карманов

\bigskip

\title 10 Руки-змеи

Для проработки плеч можно в любой стойке, и даже сидя. Если хочется пропустить ци вниз, тогда
дракон лучше.

\bigskip

\title 11 Косое раскрытие плеч

Напряжение, напряжение, максимальное напряжение, отдых.
Дыхание: три коротких выдоха - один длинный.

\bigskip

\title 12 Танцующие змеи

\bigskip

\title 13 Танцующий журавль

\bigskip

\title 14 Проработка плеч до таза

\bigskip

\title 15 Проветривание рёбер

\bigskip

\title 16 Лыжник

\bigskip

\title 17 Обезьяна лезет на дерево

\bigskip

\title 18 Приседания

Когда пятки вместе таз больше открывать.
В узкой столбовой стойке (когда стопы на ширине плеч) стопы ставить параллельно.
Если в приседании на одной ноге плохо держу равновесие, значит таз сильно уходит вбок.
Помогает более широкая стойка, тогда таз остаётся между ногами и равновесие лучше.
Допускается чтобы проекция колена выходила за передний край стопы.

\bigskip

\title 19 Качание тазом

При поднятии весь таз немного напрягается (мышцы тазового дна тоже напрягаем), не только
точки куа.
Точки куа --- просто точки контроля.

\bigskip

\title 20 Вращение коленями

Проекция колена может выходить за передний край стопы.
Стопы могут открываться от пола.
Полный круг коленями делать.

\bigskip

\title 21 Спонтанные падения

\bigskip

\title 22 Нога-копьё

\bigskip

\title 23 Пролезть в дыру в заборе

\bigskip

\title 24 Перешагнуть забор

\bigskip

\title 25 Восьмёрка ногой

\bigskip

\title 26 Император восходит на трон

Если трудно удержать равновесие на этапе восьмёрки:\par
1. Той рукой, которая пойдет вверх - можно опираться на стену или спинку стула.\par
2. Делать упражнение с минимальной амплитудой, чтобы меньше терять равновесие.\par
3. Делать сразу наклон без предварительной восьмёрки.

\bigskip

\title 27 Чечётка

Полезна для оздоровления коленей и развивает навык координации в ногах.
Желательно чтобы одна нога была впереди, чтобы был перенос веса и чтобы коленный сустав
включился.

\bigskip

\title 28 Наблюдаем за походкой

\bigskip

\title 29 Бабочка

\bigskip

\title 30 Шагающий экскаватор

\bigskip

\title 31 Полушпагат

\bigskip

\title 32 Восьмёрка стопой

\bigskip

\title 33 Крокодил греет спину

\bigskip

\title 34 Массаж стоп

\bigskip

\title 35 Комплекс сидя/лёжа перед сном

\bigskip

\title 36 Дыхание неба и земли

\bigskip

\title 37 Тряска

\bigskip

\title 38 Большая весёлая обезьяна

\bigskip

\title 39 Игра на гитаре

\bigskip

\title 40 Журавль в медитации

\bigskip

\title 41 Вихрь

\bigskip

\title 42 Цзю ян шень гун

\bigskip

\title 43 Кайхэ сжатие/расширение

\bigskip

\title 44 Кайхэ всплытие/погружение

\bigskip

\title 45 Кайхэ откат/накат

\bigskip

\title 46 Пробуждение дракона

Делать с усилием.
Варианты на животе и на спине.
Делать с закрытыми глазами.
Максимальные границы спирали --- диафрагма и лобковая кость.
Варианты продавливать, прикасаться, на расстоянии от тела.
Тело участвует.
Можно делать сидя, но важно положение ног: ноги должны быть достаточно широко расставлены,
как в стойке, так чтобы были такие же ощущения как стоя.
Лучше всего дышать естественно. Иногда в такт движению рук, в иногда и не привязываясь.
Принципиально то, что внимание идёт вместе с руками. А за вниманием --- ци.

Массаж живота можно делать как по часовой стрелке так и против.
{\it В беседе с Учителем упоминалось, что для кишечника полезнее когда
сначала против часовой стрелки а потом
по часовой.}

\bigskip

\title 47 Четыре вдоха один выдох

Можно делать откинувшись на кресле или скамейке с высокой спинкой.

\bigskip

\title 48 Дыхание пятками

Если возникает напряжение в стопах, когда на пятках приподнимаем носки, то чтобы не терять
равновесие, начать с того что при перекате на пятки не сильно отрывать носки от земли -
достаточно переноса веса. В дальнейшем тело само найдет оптимальную амплитуду.

\bigskip

\title 49 Управление дыханием

\bigskip

\title 50 Круги руками

На уровне живота.
На уровне грудной клетки.
На уровне головы. \par
Свободная рука на животе, бедре, пояснице. \par
В трёх плоскостях: фронтальная, горизонтальная, сагитальная (одной ногой можно вышагнуть
вперёд).
\par Сидя, стоя, на ходу. \par
Можно с проворотом кисти, можно без.
Попробовать стоя по шею в воде.

\bigskip

\title 51 Шёлковая нить

За 1 секунду рука проходит соответственно 10 минутам минутной стрелки часов.

\bigskip

\title 52 Мужская восьмёрка коленями

По началу допустимо чтобы переносился вес на стопах - стопы встают на внешнее ребро. В
дальнейшем стремиться чтобы стопы стояли полностью по всей поверхности.

\bigskip

\title 53 Женская восьмёрка коленями

По началу допустимо чтобы переносился вес на стопах - стопы встают на внутреннее ребро. В
дальнейшем стремиться чтобы стопы стояли полностью по всей поверхности.

\bigskip

\title 54 Танцующие чаши

От себя; сначала доводить до плеча/обратно, прочувствовать как работает плечевой сустав,
локоть,
ноги, позвоночник, как распределяется вес, потом полную спираль. Аналогично к себе.

Непарные спирали от себя, непарные спирали к себе (когда одна рука наверху --- другая в низу).
С поворотом корпуса при парных к себе.

С поворотом корпуса при непарных (разворот в сторону руки которая идёт к себе когда она
проходит
возле пояса; разноимённая нога выставляется вперёд когда рука в крайней верхней точке).

\bigskip

\title 55 Два дракона играют с жемчужиной

Не желательно чтобы плечи раскачивались из стороны в сторону, больше вперёд назад.

\bigskip

\title 56 Танец тайчи

\bigskip

\title 57 Шар тайцзи --- вращение на столе

Очень хорошо делать вращение Лаогун.

\bigskip

\title 58 Шар тайцзи --- вращение у стенки

Мяч должен выскакивать --- в этом смысл. Учимся дозировать усилие, «приклеиваться»
к мячу и «следовать» за ним.

\bigskip

\title 59 Тайцзи Бань

Выполнить «танец тайчи» (№56).

\bigskip

\title 60 Наматываем ленту

Сильно не натягивать, лента должна быть чуть провисшей.

\bigskip

\title 61 Тайцзи Бо

\bigskip

\title 62 Шары здоровья

\bigskip

\title 63 Могун

\bigskip

\title 64 Узкий столб

\bigskip

\title 65 Динамический столб

\bigskip

\title 66 Четыре опоры в столбе

На что опирается тело, какие опоры мы используем, что даёт нашему телу ощущение
стабильности, устойчивости, укоренённости, сбалансированности? Эта практика даст вам
некоторое
понимание этих опор. Важный момент это то, что сегодня мы будем говорить не о физических
опорах, а о некотором внутреннем ощущении, внутренних опорах, которые можно и
нужно прочувствовать стоя столбом.

1) Начинаем с обычной столбовой позиции, только не поднимая руки вверх.
Ноги стоят на ширине плеч. Таз
чуть осажен. Тело раскрепощённое. Макушка всплывает вверх.
Руки на уровне таза.
В этом положении надавливаем руками вниз таким образом, чтобы
большие пальцы были на боковом шве, ладони смотрели вниз, а все пальцы вперёд.
В этом положении можно закрыть глаза, сосредоточиться на ощущении оси, которая
проходит через центры тазовых суставов и является
нашей нижней опорой. По сути, если провести эту ось сквозь всё тело, то она пройдёт
через область
малого таза в районе между промежностью и линией Шэньцюэ-Минмэнь.
Почувствуйте ощущение в этой области. Поймайте это состояние ---
образ мощной горизонтальной оси, которая удерживает ваше тело,
на которую можно буквально сесть.

2) Когда возникнет чувство укоренённости, отпустите руки, так чтобы они немного всплыли вперёд
на уровень живота.
Естественным образом, без
напряжения. Задача почувствовать, как тело опирается на поясницу. Для этого мы немножко
надавливаем руками спереди на воображаемую подушку, которая дальше толкает нас в живот, а
живот мягко опирается на поясницу. Возникает опора спереди-назад в области живота.
Постоять в этом положении некоторое время.

3) Ещё даём возможность рукам всплыть повыше.
Руки приходят на уровень сердца. Это как раз та самая область где руки часто находятся
в стандартном столбе. Важно
здесь то, что мы используем руки для того, чтобы опять почувствовать опирание. В данном
случае опора --- это наши лопатки. За счёт того, что руки находятся спереди,
округляем лопатки
и ощущаем как тело опирается на лопатки сзади на спине.
Итак, у нас есть опора в тазу, у нас есть опора в пояснице, сейчас у нас
третья опора --- это лопатки. За счёт лопаток спина чуть-чуть округляется, уходят межрёберные
различные невралгии, уходят боли в области сердца, сдавленность какая-то, дискомфорт в
области сердца. И плечевой пояс в целом округляется, освобождается от напряжения.

4) Идём дальше вверх. Руки опираются на плечи сверху.
Если расположить кисти рук над плечевыми суставами, они должны лечь на надлопаточные
области сверху.
Как будто вы удерживаете на каждом плече по большому одеялу, скатанному в рулон.
При
этом сохраняем осаженность всего тела и все предыдущие опоры. Если руки в этом положении
сильно устают, можно их немножко опустить вниз.

Из этого положения обратно возвращаемся в третью опору.
Переходим ко второй опоре --- возвращаем руки на уровень живота.
Затем переходим к первой опоре --- возвращаем руки на уровень таза и
опускаем руки --- стоим в естественном столбовом положении без
напряжения рук. Наблюдаем за всем телом, как оно выстроилось, 
как и на что оно опирается, где оно устойчиво, где оно не устойчиво,
просто присутствуем в этом состоянии.
В конце можно опять по держать руки возле живота.
Можно поделать «Пробуждение дракона».

\bigskip

\title 67 Широкий столб (оздоровительная свая)

Эта практика позволяет максимально отдыхать и расслабляться. Несмотря на
общий настрой на раскрепощение, на расслабление, узкий столб не всегда даёт
эффект быстрого отдыха, расслабления, потому что нужно много фокусов внимания
распределить по телу, много частей тела осознать и прочее. То есть это всё-таки так
или иначе работа.
Ну, в конце концов, {\it гун\/} --- слово, которое переводится как «работа». И поэтому есть
специальная практика, когда вы хотите стоять столбом, но не хотите сильно работать. И этому
есть вполне веские основания. Ну, например, у вас плохое самочувствие. Или вы, например,
хотите попрактиковать прямо перед сном, чтобы буквально рухнуть в постель и заснуть.
И вам точно не хочется перед сном никакой серьёзной работы. Или у вас некоторые недомогания:
простуда, температура (до 38$^\circ$). Но тем не менее хочется
попрактиковать, но не хочется совсем напрягаться. Вот для всех этих ситуаций идеально
подходит практика широкого столба.

Отличается он двумя моментами от того, что мы делали до этого:

1) Мы садимся чуть шире.
Ноги у нас не на ширине плеч, а полторы-две ширины плеч.
Обнимаю ногами большой шар, как будто сижу на лошади или на большой трубе, на дереве.
Передние края стоп направлены в стороны под 45$^\circ$.

2) Руки располагаю перед собой на уровне живота.

Эта поза называется позой всадника.
В этом положении максимально раскрепощаемся, и по мере выполнения практики, ваше тело
начинает чуть-чуть оседать вниз. Обращайте внимание на чувство комфорта. Если ваши ноги
начинают уставать, можете немножко привстать. И наоборот, если вы чувствуете, что сидите
высоко, можно сесть пониже. Найдите ту высоту, на которой тело сможет
расслабиться.

Стекаем вниз, раскрепощаемся, сбрасываем напряжение, заземляемся. Всю лишнюю, всю негативную
энергию отдаём в землю. Избавляемся от вредной, грязной, застарелой ци.

На выходе можно не вставать, если вы делаете это перед сном, или если вы чувствуете себя
нездоровыми, то можно прямо из этого положения мягко лечь на кровать, уснуть, отдохнуть. Если
вы хотите продолжать какую-то активность после такого столба, мягко соберитесь, вернитесь в
узкий столб. Можно покрутиться, покачаться, встряхнуть руками и ногами и продолжить те дела,
которыми вы занимались до столбовой практики. Надеюсь, что вам удалось поймать вот это
особенное состояние мягкости и растекания тела в широком столбе.

\bigskip

\title 68 Низкий столб (поза всадника)

Начинаем с
узкого столба (стопы на ширине плеч). Если покажется, что в узком столбе не комфортно, можно
сесть чуть пошире.
Стопы параллельно.
Стоим в столбе, руки держим перед собой, на выдохе опускаемся вниз,
чуть-чуть подавая руки вперёд, как будто через
точки Лаогун (точки сброса) выходит из тела напряжение. Когда вы садитесь вниз,
ваше тело естественным образом напрягается.
Попробуйте вывести это напряжение через ладони, через точки Лаогун. Садимся до положения,
когда бёдра горизонтально полу. В принципе можно сесть и ниже, но долго тогда сидеть не
надо. По сути «низкий столб» --- это приседание в столбе. Вопрос в том, в какой момент вы
остановитесь и насколько быстро вы будете делать это приседание.

Ещё раз. Сели, пошли вниз. Мягко встали вверх.

То же самое можно попробовать с чуть большей задержкой в нижней позиции,
чуть большим раскрытием таза, расширением стойки. Опускаться до позиции бёдра параллельно
полу, не уходить совсем вниз.
Стараемся контролировать вертикальную ось, то есть не ложимся вперёд, держим спину как
можно более вертикально.

Если вы сделаете эту практику 5 раз, это хорошая ежедневная разминка.
Она даёт эффект оживления всего тела. Очень полезно делать особенно утром или в середине
дня, когда нужно сбросить какую-то сонливость, усталость. Дать телу дополнительный заряд
бодрости. Если вы делаете эту практику 10 раз, это уже такая хорошая среднего уровня
интенсивности тренировка. Соответственно, там уже идёт и кардионагрузка, и
большая внутренняя работа по распределению напряжения и прочее. Выполнение этого
упражнения свыше 10 раз подряд рекомендуется только здоровым людям. Если есть какие-то
хронические заболевания, лучше ограничиться выполнением этого упражнения не более 10 раз в
день. Следите за равновесием, следите за геометрией тела и за внутренним балансом и
спокойствием.

Пятки не должны отрываться.

\bigskip

\title 69 Столбовые приседания

Столбовые приседания --- это быстрый вариант выполнения упражнения «низкий столб»
с акцентом на координацию работы всего тела и с акцентом на сохранение внутреннего
какого-то спокойствия и баланса. Если физически
тяжело приседать максимально вниз, то можно делать это упражнение, опираясь руками на
спинку стула или опираясь руками на край стола.

Начинаем с узкого столба (стопы на ширине плеч).
Стопы параллельно.
Поймали столбовое состояние.
Можно немножко попружинить, расслабиться. Максимально расслаблено, мягко, спокойно сесть.
Внизу просто свисаем вниз. Чтобы встать, поджимаем таз. То есть, подъем вверх мы
осуществляем через поджимание таза.
Как будто нас что-то подталкивает под ягодицы. И вот это поджимание таза запускает волну, на
которой мы встаём.

Если хочется добавить ещё эффект дыхательный и проработку плечевого пояса, то можно делать
так: вдох\\сели\\выдох, вдох\\встали\\выдох, вдох\\сели\\выдох, вдох\\встали\\выдох.

Обязательно подбираем таз, подворачиваем его под себя и только потом встаём.

Столбовая практика
приседаний предполагает, что вы садитесь и стекаете вниз, буквально свисаете с костей.

Эта позиция очень популярна когда человек хочет просто отдохнуть, расслабиться, но сесть
некуда, и человек садится просто на корточки. То есть это очень естественная позиция, в ней
тело может, должно отдохнуть, расслабиться, но для этого надо настроиться на это состояние
расслабления. И когда вы встаёте из этой естественной позиции, следите за тем, чтобы тело не
перекашивалось. Ровненько сели вниз, ровненько встали вверх без наклонов и перекосов.

Пятки не должны отрываться.

\bigskip

\title 70 Обнимаем деревья

Итак, дорогие друзья, сегодня у нас завершающий день недели, посвященной столбовой практике. И давайте мы сегодня
сосредоточимся на том, каким образом столб можно реализовать и практиковать в движении. в перемещении, ну и в
частности, на природе. Значит, если вы находитесь где-то, где есть деревья, я подозреваю, что деревья есть вокруг
любого населенного пункта, где вы находитесь, соответственно, вам нужно просто найти такое дерево или группу
деревьев, рядом с которыми вам захочется делать столбовую практику. Это идея номер один.

Сегодняшнее задание, оно
будет таким набором идей, почему я не стал записывать видео, потому что с одной стороны не хочу вас ограничивать
рамками какой-то конкретной визуальной формы, а с другой стороны Для того, чтобы визуально показать все варианты,
о которых я буду говорить, пришлось бы потратить много времени. Так что, думаю, вы уловите задание без видео.
контента.

Итак, вы подходите к дереву или группе деревьев и просто встаете рядом с ними и стоите столбом,
прислушиваетесь. Зимой разница может быть небольшая, стоите вы рядом с соснами, с березами, с дубами и так далее.
Летом разница будет достаточно существенная, потому что разные деревья это разный фон, энергетический, запахи
разные, содержание кислорода и прочее, прочее. Соответственно, если вы делаете практику где-то, где тепло, где нет
снега, и деревья активные, не спящие, то вот примерьтесь, рядом с какими деревьями больше всего энергии от столба,
больше всего вот этого чувства укоренения и прочее. Отдельный момент это столбовая практика значит на горе,
столбовая практика на высоком холме, столбовая практика на крышах зданий на высоких этажах. То есть вот
поизучайте насколько это влияет.

Идея номер два. Обнимите дерево. То есть встаньте вплотную к дереву, которое вы
выбрали и обнимите его. Очень интересно, если вы попробуете обнять дерево, которое больше длины ваших рук в
обхвате. То есть такое здоровенное деревце, какой-нибудь дубочек или старинный тополь или что-то еще подобное. Вот,
соответственно, насколько изменится ваша стойка, когда вы будете стоять вплотную к дереву, обнимая его не мысленно,
а физически руками. Теоретически можно такое делать с людьми, но тут начинаются всякие последствия, приятные и
неприятные социальные взаимодействия. Так что здесь поосторожнее. У нас были случаи, которые я вам сейчас не
расскажу, они слишком откровенные. В общем, социальные взаимодействия ведут к последствиям. С деревьями, как
говорится, проще.

Практика номер три или идея номер три. Попробуйте ходить в столбе. Попробуйте просто
перемещаться в столбе. Руки, соответственно, уже держать перед собой не обязательно. Руки могут лежать на животе,
на бедрах, просто опущенные вниз. Но при этом максимально удерживайте столбовое состояние. Попробуйте ехать в
состоянии столба в общественном транспорте хотя бы какое-то время можно опять-таки стоя в метро, в автобусе, в
троллейбусе очень хорошо тренировать столб и более того очень выгодно в том плане что сами попробуйте там
множество плюшек и бонусов Но можно и сидя ехать в столбе, можно и исследовать варианты вот этой низкой сидячей
позиции, когда тем не менее вы максимально опираетесь на ноги. Кто ездил в новых вагонах московского метро, там
есть специальные промежуточные сидушки, полустоячие и полусидячие, где можно ехать как раз в том самом присиде,
который у нас в столбе используется. То есть, ваша задача поискать, как столб реализуется в обычной жизни, как он
работает именно в повседневности на прогулке, в процессе отдыха, в процессе поездки и прочее, прочее.

Соответственно, обязательно напишите об этом. Если будет желание, можете прислать видео. Я с удовольствием
посмотрю ваше видео, как вы стоите, ходите, сидите в столбе. И, соответственно, мы это обсудим. Да, кстати, еще один
прикольный момент. Если у вас есть фитбол, то есть большой гимнастический мяч размером 90, 100, 110 сантиметров
вот попробуйте сидеть в столбе на таком большом фитболе там тоже, уверяю, будут очень интересные ощущения очень
полезные с точки зрения столбовой практики ну и на этом Всё, жду ваших отзывов, жду ваши комментарии, вопросы и
возможно ваши примеры того, как столбовая практика реализуется у вас. Нас ждёт следующая неделя нашего
марафона. Всем успехов в практике!

\bigskip

\titleX НЕДЕЛЯ 11

Итак, друзья, всем привет. Мы начинаем
одиннадцатую неделю нашего марафона.
И эта неделя будет посвящена так
называемой практике «Столб пяти стихий»
или «Пять столбов». Вы уже
познакомились с основами столбовой работы Чжан
Чжуан или Чжуан Гун, И обратили внимание
на то, что все стойки очень похожи
друг на друга, которые мы вам предлагали, да, разница в
том, что ноги ставятся чуть пошире, чуть поуже,
садитесь вы чуть пониже, чуть повыше, но по большому
счету все одинаковое. Из этого,
прошу прощения, из этого
складывается впечатление, представление о том,
что столб это вот Он такой.
Значит, соответственно, когда мы говорим
про пять столбов или столб пяти стихий,
мы понимаем, что вообще могут быть разные
ценности, цели и аспекты
столбовой работы.

Соответственно, вот давайте
начнем с уже знакомого вам
столба, который на самом деле не просто столб, а деревянный
столб. То есть столбовая практика,
которая олицетворяет собой стихию дерева.
Что это значит? Это значит,
что когда вы практикуете этот столб,
вы используете силу,
которую дает стихия дерева.
Что такое дерево? Дерево---это сила,
которая, с одной стороны, глубоко укоренена
в земле, и, с другой стороны,
это сила, которая тянется вверх. Это сила гибкая,
но неуклонная,
мощная и, я бы сказал,
целеустремлённая. Соответственно,
когда мы практикуем деревянный столб, мы вот эти
качества берём,
черпаем, находим,
актуализируем. То есть здесь может быть терминология
и психологическая, и духовная,
и какая-то физиологическая буквально.
В любом случае стихия дерева---это оно. Ну и, соответственно,
если мы хотим ещё помимо
такой условно бездушной стихии
добавить момент какой-то души,
то, соответственно, стихия дерева---это зверь
обезьяна или леопард.
Так же, как и дерево,
Обезьяна прочно стоит на ногах и при
этом способна вытягиваться вверх. Это существо очень
упругое, пружинящее и, можно сказать,
целеустремленное. Аналогично можно сказать про леопарда.
Понятно, что целеустремленное---это наше человеческое,
такой антропоморфизм.
антропоцентричность, но тем не менее,
рекомендуется китайцами сосредотачиваться
на вот этих образах, на вот этих животных.
Условно можно сказать, что когда мы практикуем стихию
дерева, мы используем или подключаемся
к тотему обезьян и леопард.
Вот. Соответственно, вот у нас деревянный столб. Я не
буду про него подробно объяснять.
Мы это все разбирали на прошлой неделе.
Плюс у нас еще очень хорошее качественное
видео, которое мы записали заранее. Оно прям подробно проговаривает
еще раз все аспекты деревянного столба.

Дальше мы берем все то же самое,
применяем, например, к стихии металла. Это у нас будет,
значит, стойка сантиши, стойка трех
совершенств. И, соответственно, это будет аспект
металла. Что такое металл? Это твердость,
жесткость, да, то есть уже в
меньшей степени упругость, хотя и упругость тоже, но жестость
и категоричность. То есть, если дерево---это целеустремленность, то металл---это категоричность.
Это такое
однозначное «да», «нет». Это зеркало. То есть,
это что-то, что разделяет. Это клетка.
Это определенная форма, которая,
ну, границы. Металл всегда задает границы.
Это оружие. Соответственно, это граница жизни и смерти.
То есть вот это металл,
одновременно это еще красота, поскольку с
чем у нас связано зеркало, это с идеей красоты. С чем у нас связаны,
например, ножницы, тоже идея
какой-то красоты, аккуратности, дисциплины и
так далее. С чем у нас связано
то, что, например,
блестящее---это красивое. То есть металл,
любые блестки, любые у девочек стразики,
бриллианты---это все по красоту, то есть это вот тоже стихия
металла. Обратите внимание, стихия металла, она ни в коем случае не
мужская, точно так же, как и стихия дерева,
но и не женская.
полигендерная стихия, то есть она подходит
к любому полу. То же самое мы можем сказать,
набор определенных качеств, про стихию дерева,
моза. И металл
это у нас черный дракон, то есть если мы берем тотемного
зверя, то тотемный зверь черный дракон. То есть
это в китайской мифологии это можно сказать
такой ультимативный зверь. Зверь, который способен уничтожить
любого другого зверя. Вот такой несбалансированный, да?
Зверь, который прилетает, чтобы навести порядок,
покарать, наградить, похвалить,
кого-то уничтожить. Вот, соответственно, именно с этим
ультимативным зверем, с таким тотемом,
мы олицетворяем себя в этой стойке.

Значит, стойка земли или бань-мабу,
значит, тотем---тигр. Тигр---это классическое животное земли. И в отличие
от металла, который жесткий, земля мягкая.
Но мягкая не значит вялая и рыхлая.
Мягкая---значит устойчивая, мягкая\kern0pt---значит уверенная
в себе, спокойная. никуда не торопящаяся,
не суетливая. Соответственно,
в стихии Земли человек
как бы припадает к Земле и берет
максимум силы из Земли. Почему тигр?
Потому что мы представляем себе, что мы
покоряем тигра или хотим его обуздать
или подружиться с ним. То есть подружиться с тигром,
обнять тигра, подчинить тигра.
Могут быть разные образы.

Стихия воды.
Тотемный зверь---это водный дракон.
Водный дракон---это, по сути, по-нашему,
это летающая змея. Только очень большая летающая змея,
размером несколько раз больше человека.
И, соответственно, вот идея в том, чтобы слиться с этим
летающим драконом, стремиться вместе с ним куда-то
в путешествие, подобно тому, как поток воды
вовлекает вас, когда вы попадаете в него. И вот опять,
вода эта мягкая, да, безусловно, но вода настолько
мощная стихия, что вообще-то в современной практике
с помощью воды режут металл и камень.
Это вот к вопросу о том, что такое стихия воды.
Соответственно, мы можем сказать,
что вода---это самая подвижная столбовая
стойка. Когда вы стоите в водном столбе, вы не
совсем стоите, вы как будто падаете, стекаете. Представьте себе
воду, которая течет по наклонной плоскости.
Да, она опирается на наклонную плоскость, но при
этом она как бы все время в движении.
Вот так и человек, стоящий в водном
столбе, находится в постоянном движении, постоянно раскрепощается,
расслабляется. И это столько самое полезное для
здоровья позвоночника. Она очень мягко прорабатывает позвоночник
и дает ощущение такой жизни
и подвижности в позвоночнике.

Ну и,
наконец, у нас огненный столб. Огненный столб
это тотем птицы, причем большая птица.
Ну, что такое большая птица? Например, петух. Петух, вообще-то, может быть,
ну, конечно, меньше человека, но...
сравнить. Хороший бойцовый петух и для человека
может представлять опасность. То есть это тоже сила, но другая
сила. Какая? Взрывная.
Петух---это взрыв.
Опасная своей непредсказуемостью.
Хаотичная. И опять-таки, если вы хотите
другие образы, то это может
быть образ журавля, например, слетающий журавль.
Это может быть образ феникса,
мифической птицы.
И здесь идея в том, что вы обнимаете эту птицу
и вместе с ней взлетаете в небо. То есть берете силу
этой птицы, чтобы взлететь, чтобы воспарить. Но что это
за сила? Это сила творчества. Это сила спонтанности.
Это сила импровизации. Это сила непредсказуемости
и определенного хаоса. Огонь---это самое хаотичное. Стихия и огонь---это символ
трансформации, изменения перехода.

Вот,
соответственно, когда мы практикуем пять
столбов, мы практикуем одно
из пяти состояний. Попробуйте поймать
эти пять состояний на протяжении предстоящей недели.

А начнем мы с практики «Тихое сидение». Сделайте эту практику как
настройку перед и после столбовой работы в каждой стихии. И сравните опыт...

\bigskip

\title 71 Уцзи

\bigskip

\title 72 Деревянный столб

\bigskip

\title 73 Металлический столб (саньтиши)

Ставим ноги широко. Полторы-две ширины плеч.
И дальше поворачиваемся влево. Поворачиваем себя руками. И вслед за руками немножко
поворачиваются ноги. Можно начать с деревянного столба. Сели в деревянный столб, только чуть
пошире. И повернулись влево.
Давим руками. При этом у нас присутствуют так называемые девять округлостей: 1. согнутые
ноги, 2. округлый таз, 3. округлая спина, 4. округлые плечи, 5. округлые локти, 6. округлые
запястья, 7. округлые пальцы. Все тело скруглённое. Стоим равномерно на двух ногах. Давим
руками, толкаемся спиной. Возникает такой распор в теле.
Обращаем внимание на то, чтобы у нас были сильно натянуты мышцы во всем теле. Ощущение
натяжения мышц, ощущение таких звенящих мышц. И при этом все суставы максимально округлены.
Единственное исключение у нас шея. Шея мягко вытягивается вверх, так же как в деревянном
столбе макушка тянется вверх.
Могут быть болезненные ощущения в области стоп-голеностопа. На эти места идёт большая
нагрузка. Постепенно стоп-голеностоп расслабиться.
Постояли. Дальше. Через верх вдох-выдох. Стопы повернулись вправо, но не полностью, на
45--60$^\circ$, а руки повернулись вправо на 90$^\circ$.

Если вас сильно беспокоят какие-то ощущения в теле, можете чаще менять стойку, но желательно
выстоять на каждой ноге хотя бы под 2 минуты.

\bigskip

\title 74 Земляной столб (бань-мабу)

У меня расстояние от тазобедренных суставов до лодыжки 86 см.

\bigskip

\title 75 Усмирение дракона

Стихия воды --- это стихия мягкая и текучая.
Соответственно, позиция водяного столба --- это позиция, в которой мы не фиксируем тело на
одном месте, постоянно чуть-чуть двигаемся, ищем удобные и комфортные положения для всех
частей тела. Базовая метафора: представьте себе огромного водяного дракона,
который может вылетать из воды, летать по воздуху и так далее. И вот ваша задача --- подхватить
телом эту огромную летающую змею и, соответственно, её покорить, подчинить себе, подружиться
с ней и летать вместе с ней.

Исходное положение --- стандартная столбовая стойка. Поворачиваемся на 180$^\circ$. Одна нога
стоит полностью на стопе, вторая нога стоит на носке, пятка в воздухе. Передняя нога, на ней
практически 80% веса, задняя нога 20% веса, коленка опирается на икроножную мышцу передней
ноги. Получается полуприсед.
Задняя нога под углом относительно вертикали, примерно~30$^\circ$ и упирается пяткой.
Под этим же углом находится
наш корпус, голова и обе руки.
Сидим в наклоне вбок, голова тоже под углом.
При этом всё тело скручивается винтом по оси. Получается винт от стоп и до
макушки.
Одна рука толкает вверх-вперёд, другая --- назад-вниз.

Вначале потренировать переход в эту позицию отдельно, чтобы
делать это уверенно,
комфортно и без потери равновесия. В одну и в другую сторону.

Максимально расслабляемся,
раскрепощаемся, убираем напряжение с позвоночника, все тело мягко провисает. Ни в коем
случае не напрягаемся, сидим свободно, спокойно. Чуть-чуть можно покачиваться, искать
комфортное положение.

\bigskip

\title 76 Огненный столб

\bigskip

\title 77 Переходы пяти стихий

\bigskip

\title 78 Два потока

\bigskip

\title 79 Раскрепощаем руки

Данное упражнение --- разновидность первого даосского вращения.
Здесь делается акцент на естественном раскачивании,
специально руками не машем. Ну и стопы можно приклеить к полу, чтобы устойчивее было.

\bigskip

\title 80 Собираем ци

\bigskip

\title 81 Раскрепощаем плечи

Третье упражнение размягчения тела направлено на одну из важнейших зон на теле, зона так называемой
верхней заставы или плечевой пояс. как известно из Даосской системы и из современной телесно-ориентированной
европейской терапии. Плечи, шея, орла---это место очень сильного хронического застойного напряжения.
Особенно если вы работаете за компьютером, много сидите за мной в автомобиле, Значит, читая эти книги, ничего
не полезно прижимать вам. В этом смысле можно сказать, что примерно 90% современного населения. Семьдесят,
восемьдесят страдают хроническими напряжениями в ручке. Поэтому сегодня наша задача именно размягчить плечи.
Размягчить частные плечи. Каким образом мы это будем делать? Через покачивание руками.
Значит, вот я встану боком. Вот смотрите, движение. Вперед, руками вперёд. Простое и естественное инерционное
движение. Но наша задача делать это движение двумя руками в противофазе. Одну руку поставим вперёд, руками
вперёд. Одну руку вперёд, руками вперёд. По возможности краем сознания удерживаем образ двух отроков, восходящий
и нисходящий. Руки доводили движение до кончиков пальцев.
Можно делать это так, можно делать это так, можно делать это так. То есть, но в любом случае, Си доходит до
пальцев. Си доходит до кончиков. На следующей неделе мы с вами будем работать с двенадцатью демонами,
двенадцать меридианов. Вот комплекс Уан-Зе-Чена мягко прорабатывает двенадцать меридианов очень таким простым,
естественным способом. Простой способ. У нас активизируются двенадцатые меридианы, двенадцатые меридианы. Я
знаю, что цель творения---освободить. Видите, я поднимаю руку вверх, плечо напрягается. Например, вот так. Прямая
рука---это жёсткая, мёртвая, тяжёлая рука. Руки освобождаем, и движение освобождённых рук постепенно, постепенно,
этим волновым процессом освобождаем плечи.

\bigskip

\title 82 Растягиваем и раскрепощаем спину

\bigskip

\title 83 Раскрепощаем колени

Пятое, заключительное упражнение комплекса Хуан за Чен ва называется. Пятое упражнение. Давим ногами в землю,
даже можно точнее сказать, давим бедрами, размягчаем, раскрепощаем колени. Коленные суставы одни из самых
проблемных суставов в теле современного человека. С одной стороны, Мы недополучаем нагрузки для колебанных
усталых, потому что очень много сидим за компьютером, за рулем автомобиля, в общественном транспорте. То есть мы
меньше ходим и больше сидим. Вообще человек это существо, как мы знаем, прямоходящее, но современный городской
житель это скорее существо сутулосидящее или как-нибудь Приближающие. Для того, чтобы суставы работали
оптимально, необходимо, особенно в среднем старшем возрасте, коленные суставы правильно нагружать и давать им,
насколько это возможно, восстановление. Потому что частично, если сустав еще не полностью разрушен, то можно его
Если сустав разрушить, то можно сформировать мышечный корсет. Таким образом, чтобы мышцы сустава максимально
поддерживали. В пятом упражнении будет максимальная польза для тех, у кого сильно болят колени или вообще
проблемы с ногами. Встаем в позицию молоденькой дощечки. Пятка в пятке. Наушники на щечки. Округляем спину. Для
того, чтобы округлить спину, мы подворачиваем таз. И чуть-чуть стуливаем плечи. Выглядит это вот таким вот образом.
Собираемся. Задняя рога немножко развернута относительно передней, то есть угол постановки стоит вот такой вот. То
есть не так, а вот так. И, соответственно, сидя в этом положении, мы падаем на заднюю ногу, как бы проваливаемся
вниз. Ноги начинают нас выталкивать, но поскольку мы упали назад, Она толкает сильнее. Нас выталкивает вверх и
чуть-чуть вперед. Мы сидим на передней ноге. Здесь можно проверить. Закрестнул себя пяткой на ягодице. Теперь опять
проваливаемся. И нас выталкивает вверх и чуть-чуть назад. Вот так вот. Начинаем. Как мячик на воде прыгать вперед-
назад. Вниз, назад-вверх, вниз, вперед-вверх. Вниз, назад-вверх, вниз, вперед-вверх. Это первое фазовое упражнение.
Стопы прижаты к полу. На этой фазе И, соответственно, мы качаемся вперед-назад. У нас... Теперь вторая фраза. Сидим
назад на ноге. Положение ног не меняется. Дарим в пол пяткой передней ноги. Теперь расслабили, упустили ноги. и
подтягиваем ногу к себе в то же самое место, где давили пяткой. Теперь давим носком. Опять расслабили, выдохнули
ногу, давим пяткой. Давим носком. Давим пяткой. Давим носком. Носком. пяткой, носком, пяткой. И меняем ногу. Опять
начинаем с первой фазы. Подворачиваем таз, укрепляем спину. Мячик прыгает по волне. и и Теперь вторая фаза.
Выдавливаем пятку. Выдавливаем носок. Выдавливаем пятку. Здесь руки уже можно расслабить. Выдавливаем носок.
Руки свободны. Пятка. Носок. Пятка. Носок. Заканчиваем. На практику собираемся в позицию Удзи и в позиции Удзи
наблюдаем, как у нас в теле идёт два потока. Мы помним, какое у нас было состояние до. Насколько изменилось
соотношение по полу и между стопами. Лову оставляйте с руками на дальней тяге. Обратите внимание на положение
суставов. То есть, если, например, до начала практики вы стояли вот так вот. То есть, ноги колесом, колени между собой
не сходятся. И, соответственно, чтобы свести колени, надо прикладывать усилия напряжения. Теперь мы спокойно, без
напряжения, стоим. Ноги вместе. Естественно, ноги тела расстраиваются. И на этом все. Практикуем каждый день по 15
минут.

\bigskip

\title 84 Раскрепощаем тело

Сейчас мы быстро в таком экспресс-режиме сделаем все 5 важных дней. И
основной задачей сегодняшней практики вы, естественно, потом можете повторить чуть более тщательно и подробно.
Попробуем за 5--7 минут пройти все упражнения. И ключевой момент сегодняшнего задания\kern0pt---внимание на два потока
постоянно на протяжении всех упражнений. Итак, настрой полезен. Поймали два потока. Нисходящий, восходящий.
Уравновесим их. И пошли. Вращаем внимание на состояние друг друга. Тело постоянно идёт вверх-вниз. Вверх-вниз.
Вверх-вниз. Вверх и так далее. Может показаться, что я включил замедленное произведение на самом гибрид масштабе.
Я выполняю упражнения мягко, плавно, без рывков. И с вниманием на два потока, которые идут через тело. Второе
упражнение. Вверх. Вниз. Вверх. Вниз. Помните о том, что здесь соотношение бутоков может меняться. То есть вниз мы
можем быть больше, вверх меньше. Когда мы оседаем вниз, как вверх, вниз, и потом давай вверх, вниз максимально.
Вверх, вниз максимально. И так далее. Третье упражнение. Вверх, вниз. Вверх, Четвертое упражнение. Вверх, вниз. Они
переключаются, кто один, а кто другой. Более активные, более напрямотающие. Но постоянно балансируются между
собой. А сфотографироваться надо, а то я не могу. Глубоко и тщательно прорабатываются коленные и забедренные
суставы. На каждой руке две фазы. На первой фазе активно качаемся не только вверх-вниз, но и вперед-назад. На
второй фазе оставляем только движения вверх-вниз. Сплыли. И постоянно так. Запомнили ощущения. Сравнили, что
изменилось.

\bigskip

\titleX НЕДЕЛЯ 13

Комплекс «12 демонов» является одним из ключевых, наверное, комплексов цыгун. По крайней мере, В
традиционных школах ему уделяется довольно много внимания. Сейчас этот комплекс по определенным причинам стал
менее популярным. По сравнению, например, с «Состоянием столбов» или по сравнению с комплексами Цзин-Цзин,
русские парчи, 12 демонов стали таким не очень массовым Комплексом в основном эта методика работы с телом
используется в школах боевых искусств. Почему? Потому что она формирует правильные связи между руками, а также
связи рук с телом и выстраивает то, что называется внутренняя энергетика, Двенадцать
демонов---это вообще
двенадцать волшебных меридианов, которым у нас черты Ти. Соответственно, когда мы управляем этим потоком Ти,
когда Ти не то чтобы подчиняется нам, но не бунтует, движется, или ценностями в соответствии с законами гармонии,
тогда всё хорошо. Но когда мы болеем, когда мы невнимательны к себе, ТСИ начинает двигаться хаотично, где-то
возникает застой, где-то возникает, наоборот, избыточная активность ТСИ, и, соответственно, вот это Волшебные
меридианы превращаются в демоническую, разрушающую систему. То есть та же самая энергия, которая вас
поддерживала, подпитывала, давала вам силы, здоровье. Когда вы длительное время не внимательны к себе, это
начинает вас разрушать. Поэтому демоны. в том смысле, что это что-то, что имеет свою волю, свою энергию, и при этом
это что-то, чем нужно управлять, контролировать и так далее. Демоны в китайской традиции, поскольку это не
христианское понятие, это определённые части мироздания, которые очень мощные, насыщенные, наполненные, но при
этом они вам так просто не подчиняются, не поддаются. Их нужно обуздать, взять под контроль. В гаузских монастырях
есть специальная должность. Я разговариваю с людьми, которые буквально работают на таких должностях. Например,
охотник на демонов, погонщик демонов. Это люди, которые выполняют определенные практики и учиняют в себе эту
своевольную разнузданную практику. Но это все теория. Что нам нужно знать? Даже не теория, такая философия, что ли,
подхода. Что нам нужно знать? Что у нас Есть определенная система связей в теле, которая, можно так сказать,
разделена на две большие части. У нас есть индские меридианы и у нас есть янские меридианы. Индские меридианы у
нас проходят по внутренней поверхности рук. Это индские меридианы. Дальше. передняя поверхность тела и ноги,
внутренние, передняя и внутренняя поверхность ног. От помежности и до стоп, до кончиков пальцев. И у нас есть,
соответственно, янцы и перидиаты. Это внешняя поверхность рук. Задняя поверхность тела, то есть начиная макушки, по
некоторым источникам, начиная от лба назад, значит, соответственно, позвоночник, все, что вдоль позвоночника, и
задняя и боковая поверхность ног, соответственно, до подушек, до стоп. То есть это вот у нас такое разделение. Наша
задача, начиная с рук, руки---это всегда основа большинства даосских практик, Ну, на современном языке, говоря
терминами научными, мы можем сказать, что руки нам наиболее причиняются, подконтрольны, и они наиболее
чувствительны, потому что в руках находится максимум нервных окончаний, и проекция наших рук в мозгу составляет, я
вот сейчас боюсь соврать, чуть ли не половину от всего образа нашего тела. Нашими руками мы управляем
максимально эффективно, осознанно, разумно и прочее. В чём мы учимся этому с раннего детства. Если мы хотим
освоить какую-то сложную практику, если мы хотим научиться каким-то совсем неявным методам саморегуляции,
самоконтроля, то начинать мы это будем с рук. Проще всего начать срок, а потом уже постепенно перейти на всё. 

\bigskip

\title 85 Раскрытие инь-меридианов в стороны

Комплекс «12
демонов» очень большой, и мы возьмем из него малую часть, то, что вы можете начать делать буквально
каждый день. И если вам это понравится, запомните это ощущение, Я предлагаю попробовать делать этот
комплекс буквально как какую-то ежедневную, незаметную, не специальную разминку а-ля потягушки, зевания.
скручивания, то есть то, что вы делаете даже не в специальных тренировках, а просто в течение дня. Об этом мы
поговорим на седьмой день нашей практики. Сегодня первое простое упражнение. Значит, активизируем индские
меридианы. Иньские меридианы идут по ладони через локоть изнутри. и через подмышку, и дальше на
переднюю поверхность. Вот такое движение. Подняли руки вверх. Вашу крылья, да. И на выдохе... даем растянуться вот
этой всей линии. Привет мышечным поездам Майерса. который все это очень хорошо описал с точки зрения анатомии.
Китайцы не очень любили анатомию. В традиционном Китае анатомия практически как таковая не развивалась, потому
что китайцы не вскрывали трупы. Поэтому в 20-м уже веке это все было описано, что такое меридианы, как они проходят
по телу с точки зрения анатомических Вот у нас пинские меридианы. Еще раз. Вот эта линия. Мы натягиваем все от
кончиков пальцев. И вот сюда, и дальше пошло вниз. Поначалу у вас будет натягиваться только запястье, возможно,
локоть. В какой-то момент вы почувствуете, что начинает активно натягиваться область подмышек, сухожилия, фасцы.
Постепенно от подмышек это все перейдет на грудные мышцы. Постепенно-постепенно оно пойдет вниз по телу. То есть
этот процесс очень неспешный. Еще раз, демоны своевольные. Они не любят, когда ими жестко, директивно командуют.
Поэтому спокойно выдох. Максимум за один цикл рекомендуется делать 36 движений. Можно делать 24, можно делать
12. В принципе, можно делать без счёта, но здесь тоже такой момент, как бы не слишком увлекайтесь, давайте себе
отдых, поскольку это воздействие на сухожилие, и это воздействие на сухожилие в тех местах, в которых вы раньше
совсем не задумывались. Поэтому обратная связь поначалу будет плохой, недостаточной.
И главный принцип---не
навреди, то есть лучше меньше, но регулярно, чем много и потом восстанавливаться после травмы. Можно делать
несколько подходов. Сделаем, например, шесть движений. Обязательно с выдохом. И выровняйся. Наземляйся.
Практикуем. Исследуем, наблюдаем, анализируем ощущения, делимся своим опытом в чате.

\bigskip

\title 86 Раскрытие ян-меридианов в стороны

Разворачиваем руки ладонями вверх и немножко к себе. Чем-то похоже
на движение, если вы помните, в самом начале марафона мы делали руки змей. Вот это что-то подобное. Выдох. Выдох.
Напоминаю, индийские меридианы идут снаружи по локтю, по плечу, уходит на лопатки и вниз, вверх, до макушки, даже,
возможно, до теменной кости голба и вниз по позвоночнику. Дотягивается вся задняя половина тела. Очень важно здесь
быть внимательными, не торопиться. В какой-то момент у вас начнут работать фасции на голове. Это может быть
Полезный процесс, когда активизируются фасции на голове, апоневроз. Но это процесс полезный. Не торопитесь, не
спешите, не давайте волю паники какой-то. «А, я что-то сделал не так, у меня там сейчас что-то прорвется, лопнет сосуд
какой-нибудь». И всё. Техника, если её делать медленно, аккуратно и дозированно, техника безопасная. Техника
подходит для любых заболеваний. Единственное, если у вас сломаны кости, наверное, не стоит делать эти упражнения
на переломных костях. или если там порваны мышцы связки в остром состоянии, эти упражнения делать не надо. Но
как только вы более-менее восстановились, как только вы можете ходить без гипса и без всяких корсетов и прочего, вы
можете потихонечку-потихонечку делать эти упражнения. Они помогают вашему связочному аппарату, И закончили. Как
вы видите, техника очень простая. И, по сути, здесь нет никакой глубокой теории. Есть некоторая философия. Но если вы
уже настроены на тело, если вы уже последовательно вместе с нами прошли этапы нашего марафона, то вам будет
несложно активизировать эти меридианы, почувствовать соответствующие потоки, которые начинают идти по телу, и,
соответственно, поделиться этим опытом с другими людьми.

\bigskip

\title 87 Раскрытие инь-меридианов вперёд

Разберем упражнение, которое
может вызвать некоторые затруднения, хотя с виду оно Такое же простое и даже, я бы сказал, примитивное,
безыскусное, как и предыдущее, но есть свои нюансы. Итак, открываем индские меридианы в позиции руки спереди.
Значит, исходное положение вот такое. Дальше вытягиваем руки от пальцев, напоминаю, индские меридианы идут вот
так. Вытягиваем руки вот туда. Наш образ. Перед нами шар. И мы этот шар огибаем руками так, чтобы шар оказался под
нашими руками. Мысленно стараемся кончики пальцев завести в промежность. Тянем кончики пальцев в промежность.
Мы делаем это очень мягко. и последовать. Вот, смотрите. Назад можно руки не возвращать, ну или чуть-чуть совсем
вернуть. Видите, с каждым тактом мы чуть дальше проталкиваем руки. Обязательно в конце выравниваем тело,
устраняем перекос. Еще раз. Можно встать на носочки. И еще один подход. Подушки. Вот такие совсем простенькие
упражнения. С какой-то точки зрения несерьезные. С какой-то точки зрения выглядят как баловство. Кажется, что ничего
не происходит. Но мы уже с вами много раз про это говорили. В даовской традиции, когда кажется, что ничего не
происходит, происходит самое главное. И я еще раз повторю пословицу, которую моему учителю сказал один из
величайших мастеров боевых искусств и цигуна Ян Фан Шен. «Лучше тренироваться мало, чем много, Лучше
тренироваться правильно, чем долго, а лучше всего тренироваться волшебно и чудесно. Вот на этой неделе мы с вами
осваиваем такие волшебные и чудесные практики, которые при практически совершенно минимальном внешнем каком-
то эффекте дают очень глубокую мягкую и эффективную проработку всей сухожильной системы организма. Можете не
верить, но я почувствовал эффект от комплекса 12 демонов, но где-то через 3-4 года после того, как я начал делать, я
обнаружил, что мне хочется его делать регулярно. Попробуйте.

\bigskip

\title 88 Раскрытие ян-меридианов вперёд

Повесили руки перед собой и укладываем руки вперед.
Янский меридиан, внешняя поверхность руки, плечевой сустав, лопатка и далее позвоночник. Внутренние
ощущения, такой образ на которую можно настраиваться, как будто вам на руки кладут большой груз. Вот кто жил в
своем доме, в деревне, когда вам дрова на руки выкладывают, и вы несете эти дрова куда-то, да? Значит,
соответственно, вот натягивается вся структура нашего тела в идеале, еще раз, Вверху до лба, внизу до подушки. Можно
делать непрерывно 12, 24, 36 раз. Можно делать серии по 6 раз с отдыхом, с наблюдением. Особенно если у вас идут
очень сильные ощущения, возможно, болевые. Делаем по чуть-чуть, постепенно, постепенно, постепенно. Можно чуть
шире руки развести, но не слишком широко, Сильно вниз руки не опускаем. Смотрите, я поднимаю руку вверх. Вот у меня
вторая рука на области Вверх живота метеоритный отросток, то есть диафрагма. Если считать по пуговкам, это у нас раз,
два, три, четвертая пуговица. И, соответственно, вот я руку поднимаю. Видите, рука не слишком низко опускается. Если
вы начнете руку опускать ниже, у вас начнут работать другие связки, другие мышцы, и си может пойти. Ну, не то чтобы
это будет вредно, но просто вот этот эффект открытия перидианов вы можете не почувствовать. Поэтому движение
размашистое, оно вроде бы хорошо, но лучше делать более сдержанно, прочувствовать все тело.

\bigskip

\title 89 Маошаньский комплекс

Это простая техника мягкой
инактивизации всех остальных меридианов. Сразу скажу, мы не будем концентрироваться на конкретных меридианах.
Мы проводим такую общую Практика оживления всего тела. Комплекс хороший как утром, для того, чтобы проснуться и
взбодриться в течение дня, для того, чтобы сбросить с себя накопившуюся усталость, перевозбужденность и так далее.
И вечером, чтобы пошел отдохнуть после рабочего дня и переключиться на Какие-то домашние дела перед сном для
того, чтобы расслабиться и комфортно уйти в сон. То есть комплекс универсальный. Он балансирует, гармонизирует,
уравновешивает между собой все остальные потопки материи. Начинаем с небольшого массажа ладони. Умываем
ладони. Дальше берем в ладони что-то, можно сказать, чистое тси. Это чистое тси. Умываем лицо три раза. Раз.
Проводим полностью по голове.
Два. Три.
Скользим руками вниз по телу. И моем по четыре.
Так же. Раз. Два. Три.
Скользим руками вниз. Моем колени.
Раз. Два.
Три. Моем стопы.
Раз. Два. Три.
Сплываем до живота, плывем вживую.
Раз. Два. Три.
И повторяем этот цикл еще два раза. Всего у нас получится девять плываний в каждую ключевую зону тела.
Зону. Раз. Два. Три.
Очередь.
Раз. Два. Три.
Колени.
Раз. Два. Три. Стоп. Раз. Два. Три. Живот. Раз. Два, три, четыре, пять, шесть, семь, восемь, девять, один, два, три, четыре,
пять, шесть, два, три, четыре, пять, шесть, семь, восемь, девять, один, два, три, пять, шесть, семь, восемь, девять, один,
два, три, пять, шесть, семь, восемь, девять, один, два, три, пять, шесть, семь, восемь, девять, один, восемь, девять, один,
два, три, пять, шесть, семь, восемь, девять, один, два, три, пять, шесть, семь, восемь, девять, один, два, три, пять, шесть,
семь, восемь Стопы. И стопы. Повторяйте за собой.

\bigskip

\title 90 Прохлопывания

Еще одно упражнение, которое очень хорошо активизирует меридианы и позволяет включить нашу энергетическую
систему, одновременно устраняя клейкосы, зажимы, проводя такую Так что, начиная с базового вращения, называемого
первой науческой вращения, как вы заметили, оно много раз делается и, можно сказать, является одной из ключевых
схем нашей науческой практики. По ходу этого вращения начинаем поднимать руки чуть выше котей. Проходим область
рюкзака. Спереди и сзади. Еще чуть выше. Проходим грудную клетку. И сзади между лопаток. Проходим плечи.
Протопываем величество лимфы. Мягко кончиками пальцев простукиваем, разломываем челек. В первую очередь по
средней оси. Ну и чуть-чуть можно расходиться правым-левым по средней оси. Опять возвращаемся на живот.
Прохлопываем живот и поясницу. Область забедренных суставов. Можно дополнительно прохладить руки. Снизу, сверху.
И переходим наверх. Нахлопываем спереди. Козлоком. Изнутри. Колеем спереди. Козлоком. Изнутри. Более спереди.
Сбоку. Изнутри. Потоптались по столу. Прохлопывать стол можно на весу. Но это требует какой-то отдельной подготовки,
другой позы. А здесь можно просто потоптаться. И заканчиваем тем же самым первым вращением, с которого мы
начинали. Собрались, наблюдали, как ци течет по нашему телу, по гиридианам, осознавали, где в тени ци больше,
меньше, как меняется качество ци, где легче, где теплее, где холоднее. Осознали эту всю картину, наблюдали за ней, и
сегодня все.

\bigskip

\title 91 Интеграция по теме «12 демонов»

Сегодня у вас будет
одновременно очень простое и, по части, сложное задание. Вам будет необходимо выполнять практику открытия 12
меридианов в течение всего дня. С момента, как только вы прослушаете это задание, установите себе таймер на 2 часа
на любом телефоне, смартфоне, смарт-часах и так далее. И каждые 2 часа находите возможность хотя бы 5 минут или 10-
15 минут выполнять практику 12 демонов в любом варианте, который вам захочется. Можете повторять
последовательно все упражнения, которые были на этой неделе. Последовательное выполнение всех упражнений
займёт у вас где-то 15-20 минут. Можете делать одно или два упражнения, которые вам понравились больше всего.
Желательно, если вы делаете упражнения на янские меридианы, обязательно дополнять их упражнениями на инские
меридианы и наоборот, чтобы к концу практики у вас было более-менее сбалансированное состояние. Итак, работаем на
протяжении всего дня. Два часа перерыв. небольшая практика, два часа перерыв, небольшая практика. Для тех, кто
хочет максимального результата, продлите этот тренировочный цикл на сутки, то есть ставьте таймер на протяжении
всего сегодняшнего дня, и если сегодня было меньше условно 12-16 часов практики, то, соответственно, продлите
практику и завтра тоже до следующего задания следующей недели. Соответственно, напишите о том, как у вас это
получилось в чате по итогам дня уже на следующей неделе. Напишите о том, сколько раз вы делали практику, какие
упражнения делали больше всего или, в принципе, какие упражнения делали на этой интеграции, и какие эффекты, какие
изменения в самочувствии вы заметили на протяжении этого дня. Я с удовольствием отвечу на любые ваши вопросы, В
принципе, по тому, что происходит в нашем марафоне, я могу сделать вывод, что вы уже освоились, и, значит, если
возникают вопросы, то они носят какой-то сугубо технический характер. В целом люди, которые участвуют в марафоне,
вошли в практику и, как говорится, слушают поле ци и находятся в потоке ци.

\bigskip

\titleX НЕДЕЛЯ 14

Завершающую неделю марафона мы посвятим комплексной системе развития тела и сознания. В Китае она называется
«метод спиральных усилий»
(Lu\'oxu\'an Y\`unj\`\i n F\v a).
Суть этой практики в том, чтобы направить по определённым траекториям в теле сначала внимание/дух Шэнь. Потом за
вниманием пойдет поток Ци. И в итоге в теле сформируются определенные усилия (полезные привычки и биологические
структуры тела) -Цзинь.
-
Спиральные усилия это уже уровень серьезной практики, их наработка необходима в боевых искусствах, целительстве,
каллиграфии и тд. Тем не менее получить
представление об этих практиках можно уже сейчас (а когда-нибудь мы посвятим им отдельный марафон).
Во всех упражнениях этой недели важно соблюдать следующие принципы:
1. От медленного к быстрому.
-
Все движения сначала осваиваются медленно, иногда уже сверх медленно. Но по мере практики и появления чувства
потока Ци, движения можно и нужно ускорять. На этом марафоне мы рекомендуем ускорение х2-х3 от начального темпа,
хотя на видео может демонстрироваться и х5-х10.
2. В любой непонятной ситуации - выровняться и отдохнуть.
Упражнения данной недели условно безопасные. Это значит, что при их выполнении могут быть неприятные ощущения и
даже ухудшение самочувствия, если быть невнимательными к себе.
Как говорит мой Учитель: "лучшая защита от ошибок - это лень".
3. Если не шевелится и не трясется шевелить и встряхивать.
В спиральных усилиях важно почувствовать "упругое стряхивание" (кит. фацзин) и
вибрацию в руках. Но если поначалу руки не трясутся и не вибрируют их можно физически встряхивать в ходе каждого
движения и в конце движения. Хотя бы 3 раза.
4. В целом спиральные усилия это
Ян-практика, они возбуждают и разогревают. Их рекомендуется выполнять до обеда, лучше всего - с утра, после
небольшой (2-3 минуты) разминки. Вечером эти упражнения делаются редко, и не менее чем за 2 часа до сна.

\bigskip

\title 92 Спирали вверх/вниз

Начинаем практику спиральных усилий. И начнем мы ее с простых спиральных движений вверх-вниз. Вот
смотрите, мы можем находиться в любой стойке, вы можете стоять в столбе, вы можете стоять одну ногу вперед, другую
назад. В общем, я сейчас даже не буду про ноги говорить, потому что здесь вариантов комбинации бесчисленное
множество. Можете даже это делать, сидя на стуле, на табуретке, сидя на полу и прочее. Основное это то, что делают
наши руки и как руки связаны с корпусом. Вот у нас позиция рук. Мы можем сделать дыхание неба и земли. настроиться,
собраться, выровняться. И, кстати, как я уже написал в теоретической части к этой неделе, особенность спиральных
усилий, что вы каждый раз, когда чувствуете, что что-то идёт не так, вы просто выравниваетесь, собираетесь,
возвращаетесь к этой практике дыхания неба и земли. Вот у нас идёт движение. Видите, я не просто поднимаю руки
вверх, опускаю вниз, а закручиваю руками спираль. Это простая спираль. А теперь пробуем делать тройную спираль. По
отдельности каждой рукой, потом соединив руки. Раз. Два. Три. Вниз. Раз. Два. Три. Вниз. Другой рукой. Раз. Два. Три.
Вниз. Раз. Два. Три. Вниз. И вот так вот чередуем. Можем попробовать двумя руками, в том смысле, что две руки
работают согласованно, но не одинаково. И опять выравниваемся. Это у нас была восходящая спираль. Мы делали
движение. Вверх, вверх, вверх. Вниз. Спираль. Вверх. по центру вниз. Теперь то же самое делаем в нисходящую спираль.
Раз. Два. Три. Вверх. Раз. Два. Три. Вверх. То же самое. Играет а а музыка. Как-то так. Обязательно напишите о своих
ощущениях. Почувствовали спираль, не почувствовали. Увидели, не увидели. Поняли что-то, ничего не поняли. Пишите в
чат, там все обсудим.

\bigskip

\title 93 Руки-пружины

Второе упражнение из цикла «Спиральные усилия». На самом деле, это нельзя сказать, что отдельные упражнения.
Скорее, набор микроупражнений, которые можно делать разными способами, под разными углами, в разных
комбинациях и так далее. Общее название «Руки-пружины». Вот смотрите, вот у нас рука. Вот она согнутая, вот она
прямая. Теперь сделайте так, чтобы рука пружинила. Вот, например, вот так. Вот рука пружинит. А теперь сделайте так,
чтобы она пружинила без вашего усилия. Видите? Такое остаточное колебание. Тоже самое. Двумя. Задание, как сейчас
принято говорить, со звездочкой. То есть задание повышенной сложности. Попробуйте продлить вот эту вибрацию,
ощущение пружинности в руках, как можно дольше. Кажется, что это очень простая штука, когда смотришь со стороны.
Когда пробуешь сделать это первый раз, обнаруживаешь, что руки почему-то не хотят долго трястись и очень быстро
либо устают и дубеют, либо просто даже непонятно каким местом делать это движение. Обращаю внимание, что
дыхание при этом спокойное. Видите, я разговариваю. Тело тоже раскрепощено. Чуть-чуть под другим углом. То есть
работаем не сверху вниз, а снизу вверх. Здесь как бы искусственно запираем руку, создаем напряжение так, чтобы телу
было неудобно. И когда убираем это напряжение, у нас энергия освобождается, тело раскрепощается, и вот это вот
раскрепощенное тело как раз создает вибрацию. Пусть садится. Бумер, конечно. И под звездочкой попробовать сделать
как можно дольше. Вот такие с виду несложные практики, но если вы начнёте их делать, вы обнаружите множество
интересных нюансов, о которых можно поговорить в нашем чате. Можно записать видосики, прислать нам. Мы
посмотрим эти видосики и, соответственно, обсудим, дадим комментарии, обратную связь, советы, рекомендации, что
лучше-то править, изменить.

\bigskip

\title 94 Песочные часы

Тоже внешне очень простое
движение, и, возможно, даже какое-то время вы будете именно на этой простоте движения концентрироваться, и она
вам поможет раскрепоститься и почувствовать, кости, сухожилия, что там вообще происходит. Усложнение, углубление
движения это уже следующий этап. Итак, песочные часы. Исходное положение руки у нас на уровне точек Пуа, то есть
нижней донтянь. Значит, сводим руки перед грудью, выбрасываем руки перед собой на уровне плеч, чуть выше плеч.
сводим руки перед грудью, и опять я называю это место парковка, то есть это место, где у вас рукам привычно
находиться, но автомобилисты, я думаю, поймут смысл этой метафоры.
И, соответственно, вот
парковочное место номер раз, парковочное место номер два. Раз. Два. Раз. Два. Вроде все просто, да? Теперь. Переход
из одного парковочного места в другое осуществляем через сжатие. То есть сжаться Выбросить руки. Сжаться.
Выбросить руки. Выбросить руки не вперед, а в стороны. То есть вот в плоскости вперед-назад у нас руки не идут туда.
Мы делаем движение вот так. Сжаться. Сжаться. Примерно уровень Вверху, внизу. Одинаково. Как если бы мы стояли
почти касаясь корпусом какой-то стенки, лицом к стене. Вот у нас, соответственно, внизу руки около стены и вверху руки
около стены. Вот делаем движение. Сжаться. Раскрыться. Сжаться. Работает все тело, спина, позвоночник, плечи, бедра,
ноги. Как обычно, только голова ведет себя будро, то есть не увлекается в это безумие. Какими бы веселыми
притягательными оно ни было. Голова отдыхает, спокойно наблюдает за происходящим. Сжаться, раскрыться. сжаться,
раскрыться. И пробуем немножечко ускоряться. И еще раз медленно. Сжаться. Раскрыться. Сжаться. Раскрыться.
Сжаться. Раскрыться. Сжаться. Раскрыться. Очень увлекательная практика, особенно когда вы начинаете ускоряться.
На малых скоростях может показаться скучным движение, но как только вы ускоряетесь хотя бы в 3-5 раз, как
говорится, вечерок перестает быть тонным. Обязательно напишите, до каких скоростей вы дошли без фанатизма, без
боли, без повреждений каких-либо. спокойно, мягко, комфортно, насколько удалось ускориться, сохраняя вот эту
траекторию.

\bigskip

\title 95 Обратные песочные часы

Еще одно упражнение, продолжая работу с сжатием и расширением в рамках спиральных усилий. И сегодня у нас
обратные песочные часы. Скажем так, практика инвертированная относительно песочных часов. Они уже никакие не
песочные часы. Метафоры здесь могут быть любые. Просто движение, просто траектория сжатия, и парковочные места
два. Раз, два, три, четыре. Раз, два, три, четыре. Парковочное место раз, парковочное место два. Раз. Два. Три. Четыре.
Раз. Два. Три. Четыре. Расширение. Сжатие. Расширение. Отдых. Нейтральное положение. Расширение. Растягиваем
кружение. сжатие, расширение, нейтральное положение. Ускоряемся. Напоминаю, что я на видео показываю
максимально для себя максимально возможное ускорение, это довольно быстро, и не рекомендую делать такое
ускорение пока вам. Это скорее такой момент тренировки вашего восприятия, чтобы ваш глаз, ваш мозг учился
наблюдать за таким быстрым движением, осознавать его и допускать, что оно вообще возможно. Скорее всего, вы
будете делать гораздо медленнее, это неплохо. И я к такой скорости пришел совсем не сразу. Ну, где-то пара лет мне
понадобилось на то, чтобы дойти до таких скоростей с того момента, как я начал практиковать вот это спиральное
усилие. Не торопитесь, помните старую китайскую пословицу, поговорку и бу, и буди, дао, дао, буди. постепенно, день за
днём, продвигаемся к великой цели. Ну или дорога, состоящая из десяти... Дорога длиной в десять тысяч ли состоит из
маленьких шагов. Идём маленькими шагами, не торопимся, и будет нам успех, будет нам счастье, будет нам гунг-фу и
хорошее состояние.

\bigskip

\adviceF
Время после 21.00 в даосской традиции считается очень полезным для практики.
В это время мозг у человека обычно переходит в режим отдыха, активность сознания
снижается и меньше мешает --- практика становится более глубокой.

\bye
