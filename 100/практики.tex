%&10pt
\nopagenumbers
\long\def\title#1 #2\par{\noindent{\bf#1} $\underline{\hbox{#2}}$ \medskip}

\title 0 Тихое сидение

\item{1.} Стопы держать параллельно друг другу.
\item{2.} Расслаблять руки.
\item{3.} Обращать внимание на влияние солнца на позу и баланс Инь Ян.
\item{4.} Расслаблять челюсти.

\bigskip

\title 1 Первое даосское вращение

Наклон тела чуть назад.
Не допустимо выпрямлять опорную ногу в колене. Сгибать до комфортного положения.

\bigskip

\title 2 Второе даосское вращение

Стопы неподвижны. И идёт нагрузка на них. Пятки шире, чем носки.

\bigskip

\title 3 Голова качается дракон улыбается

Делать стоя.

\bigskip

\title 4 Четыре переката головы

Можно делать сидя и стоя.

\bigskip

\title 5 Тигр играет с добычей

\bigskip

\title 6 Третье даосское вращение

\bigskip

\title 7 Слушаем Ци

\bigskip

\title 8 Тигр и журавль

Та рука, что поддерживает локоть не устаёт. Ее можно просто опереть на грудную клетку. Рука,
которой крутим, должна активно работать, с усилием. Делайте движение так, будто вы на себя и от
себя двигаете тяжёлый предмет. Прорабатывается вся дельта, нагрузка может быть весьма большой, по
самочувствию. Если устаёт плечо руки, которая держит локоть и/или рука, которая изображает клюв и
лапу, больше устаёт не в плече, а в кисти, то чаще менять направление вращения и руку.

\bigskip

\title 9 Проверка карманов

\bigskip

\title 10 Руки-змеи

Для проработки плеч можно в любой стойке, и даже сидя. Если хочется пропустить ци вниз, тогда дракон лучше.

\bigskip

\title 11 Косое раскрытие плеч

Напряжение, напряжение, максимальное напряжение, отдых.
Дыхание: три коротких выдоха - один длинный.

\bigskip

\title 12 Танцующие змеи

\bigskip

\title 13 Танцующий журавль

\bigskip

\title 14 Проработка плеч до таза

\bigskip

\title 15 Проветривание рёбер

\bigskip

\title 16 Лыжник

\bigskip

\title 17 Обезьяна лезет на дерево

\bigskip

\title 18 Приседания

Когда пятки вместе таз больше открывать.
В узкой столбовой стойке (стопы на ширине плеч) лучше на параллельных стопах (не разводить носки в стороны).
Если в приседании на одной ноге плохо держу равновесие, значит таз сильно уходит вбок. Помогает более широкая стойка, тогда таз остаётся между ногами и равновесие лучше.
Допускается чтобы проекция колена выходила за передний край стопы.

\bigskip

\title 19 Качание тазом

При поднятии весь таз немного напрягается (мышцы тазового дна тоже напрягаем), не только точки куа.
Точки куа - просто точки контроля.

\bigskip

\title 20 Вращение коленями

Проекция колена может выходить за передний край стопы.
Стопы могут открываться от пола.
Полный круг коленями делать.

\bigskip

\title 21 Спонтанные падения

\bigskip

\title 22 Нога-копьё

\bigskip

\title 23 Пролезть в дыру в заборе

\bigskip

\title 24 Перешагнуть забор

\bigskip

\title 25 Восьмёрка ногой

\bigskip

\title 26 Император восходит на трон

Если трудно удержать равновесие на этапе восьмёрки:
\item{1.} Той рукой, которая пойдет вверх - можно опираться на стену или спинку стула.
\item{2.} Делать упражнение с минимальной амплитудой, чтобы меньше терять равновесие.
\item{3.} Делать сразу наклон без предварительной восьмёрки.

\bigskip

\title 27 Чечётка

Полезна для оздоровления коленей и развивает навык координации в ногах.
Желательно чтобы одна нога была впереди, чтобы был перенос веса и чтобы коленный сустав включился.

\bigskip

\title 28 Наблюдаем за походкой

\bigskip

\title 29 Бабочка

\bigskip

\title 30 Шагающий экскаватор

\bigskip

\title 31 Полушпагат

\bigskip

\title 32 Восьмёрка стопой

\bigskip

\title 33 Крокодил греет спину

\bigskip

\title 34 Массаж стоп

\bigskip

\title 35 Комплекс сидя/лёжа перед сном

\bigskip

\title 36 Дыхание неба и земли

\bigskip

\title 37 Тряска

\bigskip

\title 38 Большая весёлая обезьяна

\bigskip

\title 39 Игра на гитаре

\bigskip

\title 40 Журавль в медитации

\bigskip

\title 41 Вихрь

\bigskip

\title 42 Цзю ян шень гун

\bigskip

\title 43 Кайхэ сжатие/расширение

\bigskip

\title 44 Кайхэ всплытие/погружение

\bigskip

\title 45 Кайхэ откат/накат

\bigskip

\title 46 Пробуждение дракона

Делать с усилием.
Варианты на животе и на спине.
Делать с закрытыми глазами.
Максимальные границы спирали --- диафрагма и лобковая кость.
Варианты продавливать, прикасаться, на расстоянии от тела.
Тело участвует.
Можно делать сидя, но важно положение ног: ноги должны быть достаточно широко расставлены, как в стойке, так чтобы были такие же ощущения как стоя.
Лучше всего дышать естественно. Иногда в такт движению рук, в иногда и не привязываясь. Принципиально то, что внимание идёт вместе с руками. А за вниманием --- ци.
Массаж живота можно делать как по часовой стрелке так и против.

\bigskip

\title 47 Четыре вдоха один выдох

Можно делать откинувшись на кресле или скамейке с высокой спинкой.

\bigskip

\title 48 Дыхание пятками

Если возникает напряжение в стопах, когда на пятках приподнимаем носки, то чтобы не терять равновесие, начать с того что при перекате на пятки не сильно отрывать носки от земли - достаточно переноса веса. В дальнейшем тело само найдет оптимальную амплитуду.

\bigskip

\title 49 Управление дыханием

\bigskip

\title 50 Круги руками

На уровне живота.
На уровне грудной клетки.
На уровне головы. \par
Свободная рука на животе, бедре, пояснице. \par
В трёх плоскостях: фронтальная, горизонтальная, сагитальная (одной ногой можно вышагнуть вперёд).
\par Сидя, стоя, на ходу. \par
Можно с проворотом кисти, можно без.

\bigskip

\title 51 Шёлковая нить

За 1 секунду рука проходит соответственно 10 минутам минутной стрелки часов.

\bigskip

\title 52 Мужская восьмёрка коленями

По началу допустимо чтобы переносился вес на стопах - стопы встают на внешнее ребро. В дальнейшем стремиться чтобы стопы стояли полностью по всей поверхности.

\bigskip

\title 53 Женская восьмёрка коленями

По началу допустимо чтобы переносился вес на стопах - стопы встают на внутреннее ребро. В дальнейшем стремиться чтобы стопы стояли полностью по всей поверхности.

\bigskip

\title 54 Танцующие чаши

От себя; сначала доводить до плеча/обратно, прочувствовать как работает плечевой сустав, локоть,
ноги, позвоночник, как распределяется вес, потом полную спираль. Аналогично к себе.

Непарные спирали от себя, непарные спирали к себе (когда одна рука наверху --- другая в низу).
С поворотом корпуса при парных к себе.

С поворотом корпуса при непарных (разворот в сторону руки которая идёт к себе когда она проходит
возле пояса; разноимённая нога выставляется вперёд когда рука в крайней верхней точке).

\bigskip

\title 55 Два дракона играют с жемчужиной

Не желательно чтобы плечи раскачивались из стороны в сторону, больше вперёд назад.

\bigskip

\title 56 Танец тайчи

\bigskip

\title 57 Шар тайцзи --- вращение на столе

\bigskip

\title 58 Шар тайцзи --- вращение у стенки

\bigskip

\title 59 Тайцзи-бань

\bigskip

\title 60 Наматываем ленту

Сильно не натягивать, лента должна быть чуть провисшей.

\bigskip

\title 61 Тайцзи-бо

\bigskip

\title 62 Шары здоровья

\bigskip

\noindent Рекоммендации

Время после 21.00 в даосской традиции считается очень полезным для практики.
В это время мозг у человека обычно переходит в режим отдыха, активность сознания
снижается и меньше мешает --- практика становится более глубокой.

\bye
