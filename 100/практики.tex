%&10pt
\font\headfont=drmsym14
\font\recommfont=omff10

\pdfhorigin=8mm
\hsize=\pdfpagewidth \advance\hsize by-2\pdfhorigin
\pdfvorigin=12mm
\vsize=278mm

\newdimen\fullhsize
\newdimen\gap \gap=10pt % space between columns
\fullhsize=\hsize
\divide\hsize by2
\advance\hsize by-.5\gap
\def\fullline{\hbox to \fullhsize}
\def\makeheadline{%
  \vbox to0pt{%
    \vskip-20pt
    \fullline{\headfont\setbox0=\hbox{^^cf}\hss^^d0\hbox{\leaders\copy0\hskip75\wd0}^^d1\hss}%
    \vss}\nointerlineskip}
\def\makefootline{%
  \baselineskip24pt\lineskiplimit0pt
  \fullline{\the\footline}}
\let\lr=L \newbox\leftcolumn
\output={%
  \if L\lr
    \global\setbox\leftcolumn=\columnbox
    \global\let\lr=R%
  \else
    \doubleformat
    \global\let\lr=L%
  \fi
  \ifnum\outputpenalty>-20000 \else\dosupereject\fi
}
\def\doubleformat{%
  \shipout\vbox{%
    \makeheadline
    \fullline{\box\leftcolumn\hfil\kern\gap\hfil\columnbox}
    \makefootline
  }%
  \advancepageno
}
\def\columnbox{\leftline{\pagebody}}

\nopagenumbers
\parindent=0pt
\long\def\title#1 #2\par{\noindent{\bf#1} $\underline{\hbox{#2}}$ \medskip}

\title 0 Тихое сидение

1. Стопы держать параллельно друг другу.\par
2. Расслаблять руки.\par
3. Обращать внимание на влияние солнца на позу и баланс Инь Ян.\par
4. Расслаблять челюсти.

\bigskip

\title 1 Первое даосское вращение

Наклон тела чуть назад.
Не допустимо выпрямлять опорную ногу в колене. Сгибать до комфортного положения.

\bigskip

\title 2 Второе даосское вращение

Стопы неподвижны. И идёт нагрузка на них. Пятки шире, чем носки.
Ноги пружинят.

\bigskip

\title 3 Голова качается дракон улыбается

Делать стоя.

\bigskip

\title 4 Четыре переката головы

Можно делать сидя и стоя.

\bigskip

\title 5 Тигр играет с добычей

\bigskip

\title 6 Третье даосское вращение

\bigskip

\title 7 Слушаем Ци

\bigskip

\title 8 Тигр и журавль

Та рука, что поддерживает локоть не устаёт. Ее можно просто опереть на грудную клетку. Рука,
которой крутим, должна активно работать, с усилием. Делайте движение так, будто вы на себя и
от
себя двигаете тяжёлый предмет. Прорабатывается вся дельта, нагрузка может быть весьма
большой, по
самочувствию. Если устаёт плечо руки, которая держит локоть и/или рука, которая изображает
клюв и
лапу, больше устаёт не в плече, а в кисти, то чаще менять направление вращения и руку.

\bigskip

\title 9 Проверка карманов

\bigskip

\title 10 Руки-змеи

Для проработки плеч можно в любой стойке, и даже сидя. Если хочется пропустить ци вниз, тогда
дракон лучше.

\bigskip

\title 11 Косое раскрытие плеч

Напряжение, напряжение, максимальное напряжение, отдых.
Дыхание: три коротких выдоха - один длинный.

\bigskip

\title 12 Танцующие змеи

\bigskip

\title 13 Танцующий журавль

\bigskip

\title 14 Проработка плеч до таза

\bigskip

\title 15 Проветривание рёбер

\bigskip

\title 16 Лыжник

\bigskip

\title 17 Обезьяна лезет на дерево

\bigskip

\title 18 Приседания

Когда пятки вместе таз больше открывать.
В узкой столбовой стойке (когда стопы на ширине плеч) стопы ставить параллельно.
Если в приседании на одной ноге плохо держу равновесие, значит таз сильно уходит вбок.
Помогает более широкая стойка, тогда таз остаётся между ногами и равновесие лучше.
Допускается чтобы проекция колена выходила за передний край стопы.

\bigskip

\title 19 Качание тазом

При поднятии весь таз немного напрягается (мышцы тазового дна тоже напрягаем), не только
точки куа.
Точки куа --- просто точки контроля.

\bigskip

\title 20 Вращение коленями

Проекция колена может выходить за передний край стопы.
Стопы могут открываться от пола.
Полный круг коленями делать.

\bigskip

\title 21 Спонтанные падения

\bigskip

\title 22 Нога-копьё

\bigskip

\title 23 Пролезть в дыру в заборе

\bigskip

\title 24 Перешагнуть забор

\bigskip

\title 25 Восьмёрка ногой

\bigskip

\title 26 Император восходит на трон

Если трудно удержать равновесие на этапе восьмёрки:\par
1. Той рукой, которая пойдет вверх - можно опираться на стену или спинку стула.\par
2. Делать упражнение с минимальной амплитудой, чтобы меньше терять равновесие.\par
3. Делать сразу наклон без предварительной восьмёрки.

\bigskip

\title 27 Чечётка

Полезна для оздоровления коленей и развивает навык координации в ногах.
Желательно чтобы одна нога была впереди, чтобы был перенос веса и чтобы коленный сустав
включился.

\bigskip

\title 28 Наблюдаем за походкой

\bigskip

\title 29 Бабочка

\bigskip

\title 30 Шагающий экскаватор

\bigskip

\title 31 Полушпагат

\bigskip

\title 32 Восьмёрка стопой

\bigskip

\title 33 Крокодил греет спину

\bigskip

\title 34 Массаж стоп

\bigskip

\title 35 Комплекс сидя/лёжа перед сном

\bigskip

\title 36 Дыхание неба и земли

\bigskip

\title 37 Тряска

\bigskip

\title 38 Большая весёлая обезьяна

\bigskip

\title 39 Игра на гитаре

\bigskip

\title 40 Журавль в медитации

\bigskip

\title 41 Вихрь

\bigskip

\title 42 Цзю ян шень гун

\bigskip

\title 43 Кайхэ сжатие/расширение

\bigskip

\title 44 Кайхэ всплытие/погружение

\bigskip

\title 45 Кайхэ откат/накат

\bigskip

\title 46 Пробуждение дракона

Делать с усилием.
Варианты на животе и на спине.
Делать с закрытыми глазами.
Максимальные границы спирали --- диафрагма и лобковая кость.
Варианты продавливать, прикасаться, на расстоянии от тела.
Тело участвует.
Можно делать сидя, но важно положение ног: ноги должны быть достаточно широко расставлены,
как в стойке, так чтобы были такие же ощущения как стоя.
Лучше всего дышать естественно. Иногда в такт движению рук, в иногда и не привязываясь.
Принципиально то, что внимание идёт вместе с руками. А за вниманием --- ци.
Массаж живота можно делать как по часовой стрелке так и против.

\bigskip

\title 47 Четыре вдоха один выдох

Можно делать откинувшись на кресле или скамейке с высокой спинкой.

\bigskip

\title 48 Дыхание пятками

Если возникает напряжение в стопах, когда на пятках приподнимаем носки, то чтобы не терять
равновесие, начать с того что при перекате на пятки не сильно отрывать носки от земли -
достаточно переноса веса. В дальнейшем тело само найдет оптимальную амплитуду.

\bigskip

\title 49 Управление дыханием

\bigskip

\title 50 Круги руками

На уровне живота.
На уровне грудной клетки.
На уровне головы. \par
Свободная рука на животе, бедре, пояснице. \par
В трёх плоскостях: фронтальная, горизонтальная, сагитальная (одной ногой можно вышагнуть
вперёд).
\par Сидя, стоя, на ходу. \par
Можно с проворотом кисти, можно без.

\bigskip

\title 51 Шёлковая нить

За 1 секунду рука проходит соответственно 10 минутам минутной стрелки часов.

\bigskip

\title 52 Мужская восьмёрка коленями

По началу допустимо чтобы переносился вес на стопах - стопы встают на внешнее ребро. В
дальнейшем стремиться чтобы стопы стояли полностью по всей поверхности.

\bigskip

\title 53 Женская восьмёрка коленями

По началу допустимо чтобы переносился вес на стопах - стопы встают на внутреннее ребро. В
дальнейшем стремиться чтобы стопы стояли полностью по всей поверхности.

\bigskip

\title 54 Танцующие чаши

От себя; сначала доводить до плеча/обратно, прочувствовать как работает плечевой сустав,
локоть,
ноги, позвоночник, как распределяется вес, потом полную спираль. Аналогично к себе.

Непарные спирали от себя, непарные спирали к себе (когда одна рука наверху --- другая в низу).
С поворотом корпуса при парных к себе.

С поворотом корпуса при непарных (разворот в сторону руки которая идёт к себе когда она
проходит
возле пояса; разноимённая нога выставляется вперёд когда рука в крайней верхней точке).

\bigskip

\title 55 Два дракона играют с жемчужиной

Не желательно чтобы плечи раскачивались из стороны в сторону, больше вперёд назад.

\bigskip

\title 56 Танец тайчи

\bigskip

\title 57 Шар тайцзи --- вращение на столе

Очень хорошо делать вращение Лаогун.

\bigskip

\title 58 Шар тайцзи --- вращение у стенки

Мяч должен выскакивать --- в этом смысл. Учимся дозировать усилие, «приклеиваться»
к мячу и «следовать» за ним.

\bigskip

\title 59 Тайцзибань

Танцевать «Танец Тайчи» с тайцзибань.

\bigskip

\title 60 Наматываем ленту

Сильно не натягивать, лента должна быть чуть провисшей.

\bigskip

\title 61 Тайцзибо

\bigskip

\title 62 Шары здоровья

\bigskip

\title 63 Могун

\bigskip

\title 64 Узкий столб

\bigskip

\title 65 Динамический столб

\bigskip

\title 66 Четыре опоры в столбе

На что опирается тело, какие опоры мы используем, что даёт нашему телу ощущение
стабильности, устойчивости, укоренённости, сбалансированности? Эта практика даст вам
некоторое
понимание этих опор. Важный момент это то, что сегодня мы будем говорить не о физических
опорах, а о некотором внутреннем ощущении, внутренних опорах, которые можно и
нужно прочувствовать стоя столбом.

1) Начинаем с обычной столбовой позиции, только не поднимая руки вверх.
Ноги стоят на ширине плеч. Таз
чуть осажен. Тело раскрепощённое. Макушка всплывает вверх.
Руки на уровне таза.
В этом положении надавливаем руками вниз таким образом, чтобы
большие пальцы были на боковом шве, ладони смотрели вниз, а все пальцы вперёд.
В этом положении можно закрыть глаза, сосредоточиться на ощущении оси, которая
проходит через центры тазовых суставов и является
нашей нижней опорой. По сути, если провести эту ось сквозь всё тело, то она пройдёт
через область
малого таза в районе между промежностью и линией Шэньцюэ-Минмэнь.
Почувствуйте ощущение в этой области. Поймайте это состояние ---
образ мощной горизонтальной оси, которая удерживает ваше тело,
на которую можно буквально сесть.

2) Когда возникнет чувство укоренённости, отпустите руки, так чтобы они немного всплыли вперёд
на уровень живота.
Естественным образом, без
напряжения. Задача почувствовать, как тело опирается на поясницу. Для этого мы немножко
надавливаем руками спереди на воображаемую подушку.

3) Руки всплывают на уровень сердца. Это как раз та самая область в стандартном столбе. Важно
здесь то, что мы... Мы используем руки для того, чтобы опять почувствовать опирание. В данном
случае опора --- это наши лопатки. За счёт того, что руки находятся спереди, мы опираем лопатки
и ощущаем, как тело опирается на лопатки. В тазу у нас есть опора лестницы, сейчас у нас
третья опора --- это лопатки. За счёт лопаток спина чуть-чуть округляется, уходят межрёберные
различные патологии, уходят боли в области сердца, сдавленность какая-то, дискомфорт в
области сердца. И плечевой пояс в целом округляется, освобождается от напряжения.

4) Идём дальше вверх. Руки ставятся на плечи сверху. Сложите руки. Вы должны лечь на плечи
сверху.
Как будто вы удерживаете на плечах два больших одеяла, которые находятся над плечами. При
этом сохраняем саженность всего тела и все предыдущие опоры. Если руки в этом положении
сильно устают, можно их немножко опустить.

В этом положении обратно возвращаемся в третью опору. Снова видно моё тело с разных сторон. Я
неподвижно в одной опоре. Возвращаем руки на уровень. Стоим в естественном положении и
напряжении. Наблюдаем все тело, как оно выстроилось, где оно устойчиво, где оно неустойчиво.
В конце можно опять держать руки.

Можно поделать «Пробуждение дракона».

\bigskip

\title 67 Широкий столб

Эта практика позволяет максимально отдыхать и расслабляться. Несмотря на
общий настрой на раскрепощение, на расслабление, узкий столб не всегда даёт
эффект быстрого отдыха, расслабления, потому что нужно много фокусов внимания
распределить по телу, много частей тела осознать и прочее. То есть это всё-таки так
или иначе работа.
Ну, в конце концов, {\it гун\/} --- слово, которое переводится как «работа». И поэтому есть
специальная практика, когда вы хотите стоять столбом, но не хотите сильно работать. И этому
есть вполне веские основания. Ну, например, у вас плохое самочувствие. Или вы, например,
хотите попрактиковать прямо перед сном, чтобы буквально рухнуть в постель и заснуть.
И вам точно не хочется перед сном никакой серьёзной работы. Или у вас некоторые недомогания:
простуда, температура (до 38 градусов). Но тем не менее хочется
попрактиковать, но не хочется совсем напрягаться. Вот для всех этих ситуаций идеально
подходит практика широкого столба.

Отличается он двумя моментами от того, что мы делали до этого:

1) Мы садимся чуть шире.
Ноги у нас не на ширине плеч, а полторы-две ширины плеч.
Обнимаю ногами большой шар, как будто сижу на лошади или на большой трубе, на дереве.
Передние края стоп направлены в стороны под 45$^\circ$.

2) Руки располагаю перед собой на уровне живота.

Эта поза называется позой всадника.
В этом положении максимально раскрепощаемся, и по мере выполнения практики, ваше тело
начинает чуть-чуть оседать вниз. Обращайте внимание на чувство комфорта. Если ваши ноги
начинают уставать, можете немножко привстать. И наоборот, если вы чувствуете, что сидите
высоко, можно сесть пониже. Найдите ту высоту, на которой тело сможет
расслабиться.

Стекаем вниз, раскрепощаемся, сбрасываем напряжение, заземляемся. Всю лишнюю, всю негативную
энергию отдаём в землю. Избавляемся от вредной, грязной, застарелой ци.

На выходе можно не вставать, если вы делаете это перед сном, или если вы чувствуете себя
нездоровыми, то можно прямо из этого положения мягко лечь на кровать, уснуть, отдохнуть. Если
вы хотите продолжать какую-то активность после такого столба, мягко соберитесь, вернитесь в
узкий столб. Можно покрутиться, покачаться, встряхнуть руками и ногами и продолжить те дела,
которыми вы занимались до столбовой практики. Надеюсь, что вам удалось поймать вот это
особенное состояние мягкости и растекания тела в широком столбе.

\bigskip

\title 68 Низкий столб

Начинаем с
узкого столба (стопы на ширине плеч). Если покажется, что в узком столбе не комфортно, можно
сесть чуть пошире.
Стопы параллельно.
Стоим в столбе, руки держим перед собой, на выдохе опускаемся вниз,
чуть-чуть подавая руки вперёд, как будто через
точки Лаогун (точки сброса) выходит из тела напряжение. Когда вы садитесь вниз,
ваше тело естественным образом напрягается.
Попробуйте вывести это напряжение через ладони, через точки Лаогун. Садимся до положения,
когда бёдра горизонтально полу. В принципе можно сесть и ниже, но долго тогда сидеть не
надо. По сути «низкий столб» --- это приседание в столбе. Вопрос в том, в какой момент вы
остановитесь и насколько быстро вы будете делать это приседание.

Ещё раз. Сели, пошли вниз. Мягко встали вверх.

То же самое можно попробовать с чуть большей задержкой в нижней позиции,
чуть большим раскрытием таза, расширением стойки. Опускаться до позиции бёдра параллельно
полу, не уходить совсем вниз.
Стараемся контролировать вертикальную ось, то есть не ложимся вперёд, держим спину как
можно более вертикально.

Если вы сделаете эту практику 5 раз, это хорошая ежедневная разминка.
Она даёт эффект оживления всего тела. Очень полезно делать особенно утром или в середине
дня, когда нужно сбросить какую-то сонливость, усталость. Дать телу дополнительный заряд
бодрости. Если вы делаете эту практику 10 раз, это уже такая хорошая среднего уровня
интенсивности тренировка. Соответственно, там уже идёт и кардионагрузка, и
большая внутренняя работа по распределению напряжения и прочее. Выполнение этого
упражнения свыше 10 раз подряд рекомендуется только здоровым людям. Если есть какие-то
хронические заболевания, лучше ограничиться выполнением этого упражнения не более 10 раз в
день. Следите за равновесием, следите за геометрией тела и за внутренним балансом и
спокойствием.

\bigskip

\title 69 Столбовые приседания

Столбовые приседания --- это быстрый вариант выполнения упражнения «низкий столб»
с акцентом на координацию работы всего тела и с акцентом на сохранение внутреннего
какого-то спокойствия и баланса. Если физически
тяжело приседать максимально вниз, то можно делать это упражнение, опираясь руками на
спинку стула или опираясь руками на край стола.

Начинаем с узкого столба (стопы на ширине плеч).
Стопы параллельно.
Поймали столбовое состояние.
Можно немножко попружинить, расслабиться. Максимально расслаблено, мягко, спокойно сесть.
Внизу просто свисаем вниз. Чтобы встать, поджимаем таз. То есть, подъем вверх мы
осуществляем через поджимание таза.
Как будто нас что-то подталкивает под ягодицы. И вот это поджимание таза запускает волну, на
которой мы встаём.

Если хочется добавить ещё эффект дыхательный и проработку плечевого пояса, то можно делать
так: вдох\\сели\\выдох, вдох\\встали\\выдох, вдох\\сели\\выдох, вдох\\встали\\выдох.

Обязательно подбираем таз, подворачиваем его под себя и только потом встаём.

Столбовая практика
приседаний предполагает, что вы садитесь и стекаете вниз, буквально свисаете с костей.

Эта позиция очень популярна когда человек хочет просто отдохнуть, расслабиться, но сесть
некуда, и человек садится просто на корточки. То есть это очень естественная позиция, в ней
тело может, должно отдохнуть, расслабиться, но для этого надо настроиться на это состояние
расслабления. И когда вы встаёте из этой естественной позиции, следите за тем, чтобы тело не
перекашивалось. Ровненько сели вниз, ровненько встали вверх без наклонов и перекосов.

\bigskip

\title 70 Обнимаем деревья

\bigskip

\title 71 Уцзи

\bigskip

\title 72 Деревянный столб

\bigskip

\title 73 Металлический столб (саньтиши)

Ставим ноги широко. Полторы-две ширины плеч.
И дальше поворачиваемся влево. Поворачиваем себя руками. И вслед за руками немножко
поворачиваются ноги. Можно начать с деревянного столба. Сели в деревянный столб, только чуть
пошире. И повернулись влево.
Давим руками. При этом у нас присутствуют так называемые девять округлостей: 1. согнутые
ноги, 2. округлый таз, 3. округлая спина, 4. округлые плечи, 5. округлые локти, 6. округлые
запястья, 7. округлые пальцы. Все тело скруглённое. Стоим равномерно на двух ногах. Давим
руками, толкаемся спиной. Возникает такой распор в теле.
Обращаем внимание на то, чтобы у нас были сильно натянуты мышцы во всем теле. Ощущение
натяжения мышц, ощущение таких звенящих мышц. И при этом все суставы максимально округлены.
Единственное исключение у нас шея. Шея мягко вытягивается вверх, так же как в деревянном
столбе макушка тянется вверх.
Могут быть болезненные ощущения в области стоп-голеностопа. На эти места идёт большая
нагрузка. Постепенно стоп-голеностоп расслабиться.
Постояли. Дальше. Через верх вдох-выдох. Стопы повернулись вправо, но не полностью в 45-60
градусов, а руки повернулись вправо в 90 градусов.

Если вас сильно беспокоят какие-то ощущения в теле, можете чаще менять стойку, но желательно
выстоять на каждой ноге хотя бы под 2 минуты.

\bigskip

\title 74 Земляной столб (бань-мабу)

\bigskip

\title 75 Водный столб (сянлун чжуан)

Стихия воды --- это стихия мягкая и текучая.
Соответственно, позиция водяного столба --- это позиция, в которой мы не фиксируем тело на
одном месте, постоянно чуть-чуть двигаемся, ищем удобные и комфортные положения для всех
частей тела. Базовая метафора: представьте себе огромного водяного дракона,
который может вылетать из воды, летать по воздуху и так далее. И вот ваша задача --- подхватить
телом эту огромную летающую змею и, соответственно, её покорить, подчинить себе, подружиться
с ней и летать вместе с ней.

Исходное положение --- стандартная столбовая стойка. Поворачиваемся на 180 градусов. Одна нога
стоит полностью на стопе, вторая нога стоит на носке, пятка в воздухе. Передняя нога, на ней
практически 80% веса, задняя нога 20% веса, коленка опирается на икроножную мышцу передней
ноги. Получается полуприсед.
Задняя нога под углом относительно вертикали, примерно~30$^\circ$ и упирается пяткой.
Под этим же углом находится
наш корпус, голова и обе руки.
Сидим в наклоне вбок, голова тоже под углом.
При этом всё тело скручивается винтом по оси. Получается винт от стоп и до
макушки.
Одна рука толкает вверх-вперёд, другая --- назад-вниз.

Вначале потренировать переход в эту позицию отдельно, чтобы
делать это уверенно,
комфортно и без потери равновесия. В одну и в другую сторону.

Максимально расслабляемся,
раскрепощаемся, убираем напряжение с позвоночника, все тело мягко провисает. Ни в коем
случае не напрягаемся, сидим свободно, спокойно. Чуть-чуть можно покачиваться, искать
комфортное положение.

\bigskip

\title 76 Огненный столб

\bigskip

\recommfont
\noindent Рекоммендации

Время после 21.00 в даосской традиции считается очень полезным для практики.
В это время мозг у человека обычно переходит в режим отдыха, активность сознания
снижается и меньше мешает --- практика становится более глубокой.

\bye
