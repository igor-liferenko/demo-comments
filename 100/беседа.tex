%&12pt
\pdfpagewidth=297mm
\pdfpageheight=210mm
\pdfhorigin=1in
\pdfvorigin=0pt
\input QUIRE
\shhtotal=\pdfpagewidth
\htotal=.5\shhtotal
\vtotal=\pdfpageheight
\shoutline=0pt
\shstaplewidth=0pt
\shcrop=0pt
\shfootline={}
\shthickness=.27mm
\quire{20}

\horigin=9mm
\hsize=\htotal \advance\hsize by-2\horigin
\advance\hsize by-\QUIRE
\output={\ifodd\pageno\else\hoffset=\QUIRE\fi \plainoutput}

\vorigin=3.8mm
\vsize=\topskip \advance\vsize by37\baselineskip

\footline={\raise1pt\line{\hss\tenrm\folio\hss}}

\font\speakerF=omssbx10
\def\A{\item{\speakerF А.}}
\def\T{\item{\speakerF У.}}
\setbox0=\hbox{\speakerF А.\enskip}
\parindent=\wd0

\A
Почему именно 100 дней практики? Откуда это пошло?

\T
Сразу же начнём с последнего пункта. Если мы чем-то занимаемся, должно пройти какое-то время,
чтобы в теле, в
психике начались какие-то изменения. Есть разные... Православные традиции --- это, как обычно
40 дней. Более
короткий путь, там, в китайской традиции, это может быть и 50, и 70 дней.\hfil\break
\indent
То есть, должно пройти какое-то время, чтобы в теле начались... Под телом я подразумеваю
совокупность
психофизических факторов человека, а не только просто физическое тело. Должно пройти время
изменения. Почему 100
дней, я пытался выяснить, но этот вопрос достаточно тёмный.\hfil\break
\indent
По старым писаниям, это привязано к лунному циклу, как раз там 4 примерно лунных месяца,
отсюда такое время,
наиболее такой меньший вариант, что ли, плюс 100 дней тебе дают крепкий фундамент, ты
начинаешь... Это достаточно
длительное время для человека, делать одно и то же 100 дней, это требует некоторых серьёзных
усилий для людей,
которые не занимались, некоей стабильности. Человек начинает делать одно и то же сто
дней.
\hfil\break
\indent
Сам термин «гунфу» --- мастерство, полученное в результате... Если расшифровывать иероглифы,
мастерство, полученное в
результате затраты сил, усилий достаточно больших и серьёзных. Ну, в плане для каждого
человека эта серьезность и
большие усилия, они разные, естественно. Для одного, там, постоять в стойке 5 минут это уже
проблема, а для другого час
что-то поделать никакого труда не составляет. Но здесь серьёзность в том, что мы это делаем
каждый день в течение
трёх с половиной месяцев, да, если 100 дней брать. Тогда можем наблюдать какие-то
изменения в теле, но
приравнивать 100 дней... В общем, это нумерология китайская. Серьёзных работ у нас по этому поводу
не было. Писал там
Богачихин, переводил, на эту тему, нумерологию. В 90-е годы выпускал брошюры когда всё переводили.
Но когда мой
учитель по Ян увидел что я читаю он спросил через переводчика, что он читает, перевели, он
сказал, что в самом Китае
мало кто это понимает.\hfil\break
\indent
Ну, традиционно 100 дней. Есть ещё, там, Бодхидхарма 9 лет сидел, пялясь на стену, понимаешь, и
такие цифры. Суть в
том, что «гунфу» --- это мастерство, полученное в результате затрат сил и времени.
Поэтому первое, что мы
должны понимать: чтобы чем-либо заниматься, мы должны иметь правильный метод.\hfil\break
\indent
Затрат на освоение этого метода и практике этого метода --- силы, и немаловажный вопрос ---
время. Если у нас этот
фактор есть, мы можем получить гунфу. Поэтому все эти байки про освоение боевых искусств
крестьянами, которые
должны там в полях работать были каждый день и добывать хлеб на пропитание --- это
байки.\hfil\break
\indent
Хотя исключения, конечно, были в истории Китая.

\A
Как раз потому, что у крестьян времени на это не было, да?

\T
Да, да, да. У меня, помню, ученица, когда первый раз побывала в Индии, которая у нас считается
религиозной такой, она
говорит, да, религиозная, все там школьники бегают, но это на праздниках они бегут, а там у них
нет времени у простого
народа отдаваться религии и духовному росту. Поэтому это требует времени и плюс правильный
метод, то есть знать как,
знать что, когда, знать какие силы мы должны приложить. И время, это немаловажный
фактор.\hfil\break
\indent
И очень часто, допустим, если так отклоняться, китайская медицина или традиционная, та же
индусская медицина --- Аюрведа, где основной постулат это
лечение посредством диеты. Но
это требует времени.\hfil\break
\indent
Вот правильный метод и время. Естественно если там, на поле боя, как прописано в древних
текстах аж там, 4 века до
нашей эры, ранили кого-то стрелой --- там нет времени его лечить диетами и всё прочее. Надо
оперативным путём
вытащить стрелу, остановить кровотечение.\hfil\break
\indent
Сейчас,
поскольку времена изменились,
у нас тоже нехватка времени.
Отсюда практика, которая противоречит вообще восточным практикам. Бегун бежит, вроде
наматывает круги, тут же у него смартфон, наушники в ушах, и он слушает там какую-то музыку
или развлечение, может ещё английский язык учит.\hfil\break
\indent
Но это противоречит полностью всем восточным практикам, да и не только восточным --- христианским
духовным практикам, где полная отдача внимания и собранность на том, что ты
делаешь. Физически. То
есть единение тела, ума и внутренней твоей энергетики. Под этим мы подразумеваем
все
составляющие твоих полей
гравитационных, электромагнитных и все прочеё.\hfil\break
\indent
То, что в Китае описывается словом Ци, у нас неправильно
переводят как «энергия». Ци ---
это совокупность всевозможных психофизических составляющих полей: дыхания,
движения крови, какие-то обменные процессы, метаболизм.\hfil\break
\indent
Это требует времени.
Традиционно в Китае так сложилось, что минимум времени, которое необходимо затратить для
освоения какой-то практики --- 100 дней. Это тот минимум, который
позволяет увидеть
какие-то результаты в теле и какие-то изменения.\hfil\break
\indent
Причём они часто, из моего опыта, ты практикуешь, 100 дней делаешь и вроде
ничего не происходит, а окружающие, особенно которые тебя видели до этой практики и после,
они замечают какие-то изменения.
Поскольку это во времени так растянуто, изменения могут быть для практикующего и не
очень очевидны, но со
временем потом вдруг --- раньше он не мог это сделать, а потом вдруг почему-то делает.
Это
как стрелка на часах:
вроде смотришь --- она не двигается. Отвлёкся --- а она уже на полчаса куда-то пробежала, время
поменялось. Поэтому сто дней это немаловажный такой фактор --- это именно время.
Оно выступает вот той
четвертой составляющей
в описании пространства и играет большую
роль.\hfil\break
\indent
В йоге немножко по-другому меряют. В йоге не сто дней длятся практики овладения,
хотя там знают о таких временных промежутках. В традиционной
йоге, как духовной
дисциплине, когда осваивают какие-то асаны, то жгутся ароматические палочки, свечи.
Как правило, ты
принимаешь какую-то
позицию и в течение того времени, пока горит эта свечка, примерно два с половиной часа
ты должен в этой
позе находиться.
И когда ты эти два с половиной часа... Понятно, что там без раскрепощения, расслабления,
подготовки ты не выдержишь,
не сделаешь это. Но если ты это делаешь, то считается позиция освоена и ты больше её не
практикуешь. Ты её
практикуешь только тогда, когда необходимо какое-то воздействие на психо-физику, там,
лечение, ещё что-то там
усилить, углубить и так далее. Но она остаётся с тобой уже на всю жизнь. Поэтому если ты
проходишь правильную практику в течение
ста дней и закрепляешь результат, то ты больше её не делаешь.\hfil\break
\indent
Понятно, что если ты лет десять не в теме, то тебе понадобится какое-то время, чтобы выйти на
тот
же самый уровень, но
оно будет в разы короче, чем пока ты осваиваешь. А если ты ведёшь нормальный образ жизни,
продолжаешь
практиковать, то ты не теряешь этот навык.\hfil\break
\indent
Поэтому время --- очень важный фактор. Одни за сто дней могут чего-то достигнуть, другим
может потребоваться больше.
Это время, которое необходимо, чтобы необратимые изменения, в данном случае
положительные изменения, которые
система требует, они начали работать.\hfil\break
\indent
Наверное, есть какие-то гении, которые в других жизнях практиковали. Им достаточно метод
попробовать, и у них всё сразу получается. Это гении. Но в расчёт это брать не следует.
Моцартом можно
восхищаться, но
обычный музыкант, конечно, должен трудиться и работать, работать, пока не выйдет на
какой-то уровень.

\A
Понятно. Можно ли сказать, что вот с точки зрения такой практической, сто дней это во многом
требование для
самодисциплины? Понять что-то можно и за день, и за неделю, но самодисциплину,
привычку что-то делать, нужно вырабатывать.

\T
Да, понять можно и за пять минут, а освоить уже потребуется больше времени --- час-два.
Но чтобы наступили
изменения, то есть чтобы это стало твоим устойчивым навыком, на уровне
уже рефлексов,
потребуется длительное время. То есть когда это уже твоей сутью становится.
Кому-то 50--40 дней, кому-то 100 дней, а
кому-то и больше, не
обязательно 100 дней. Сто дней, это если используется правильная методика, правильная тренировка,
правильное понимание.
Без этого никак.

\A
Есть такая шутка среди преподавателей. Называется «Ещё один всё понял».

\T
Понять-то можно, а вот ввести это в тело, чтобы это стало работать, требуется
время. Иногда очень
большое время, и ста дней может не хватить. Особенно, если сам изначально метод, он такой,
не очень понятный. А
даосская методика, она как раз, если брать вот такие традиционные методики, они как раз очень
расплывчатые. Для нашего
мышления очень сложно зацепиться там за что-то. Мы привыкли к каким-то опорам, нам должно
логически быть понятно.
Ну, вот в такой парадигме мы выросли, а там этого нет.\hfil\break
\indent
Вроде как так, а вроде как и не так. Вроде как должно быть жарко, а вроде как и холодно.
Вот в этом вся сложность. Очень противоречивые вещи. Поэтому
сегодня так, а завтра
по-другому. И вот как тут...?\hfil\break
\indent
Когда учитель есть, он объясняет. И даже очень часто ничего не объясняет --- подразумевается, что
учитель это сильная личность, которую можно просто копировать, даже
на подсознательном уровне. Ты как-бы копируешь его
модель поведения. В древней Индии Брахманы обучали чтению гимна именно вот таким
образом. То есть все
знали в данной деревне, в данном городе вот такой есть мастер Брахман, который считается всеми
признанным духовным
лидером, учителем с большой буквы.\hfil\break
\indent % 17:58
Ему отдавали там детей соответствующие там сословия, на 5 лет, на 25 лет, и он просто жил как
слуга в доме, как член
семьи, в доме учителя, вот, а об этом вот китайская школа, да, это вот именно об этом, это семья,
вот, и человек там
вроде жил, там делал какие-то задания, ну, кстати, такая же была в эпоху возрождения,
мастерская художественные там
вот все эти наши великие художники такая же идея вот мастерские артиста да то есть класс там
того-то того-то да там
или или изготовление оружия например какого-то физиков 20 века выполняли работу копировали
мастеров, а потом там
в нужный момент нужное время когда учитель считал нужным и ученик, он давал какие-то
небольшие, но секретные,
условно говоря, наставления, и ученик осваивал эту всю тему.\hfil\break
\indent
Очень часто они такие неуловимые, но по цигуну, как правило, приводят там такой пример, вот
Neoglory приводят в одну
из программ такой пример, что изготовление статуэток, типа терракотовых, воинов и прочих вот
статуэток глиняных,
прежде чем их обжечь, то есть там вылеплено всё правильно, не сложно научиться относительно
скопировать, но чтобы
придать живой вид им, там мастер показывал ученику пример, там похлопать надо по щекам
легкими хлопками сделать
чуть-чуть такую маленькую улыбку и только потом в печь отправить, и фигурка уже выглядела
другой, она была живой.\hfil\break
\indent
Вот такого рода секреты, до них можно дойти и в наших условиях не имея такого рода учителей,
но это требует такой
определенной умственной работы, то есть собирание разных практик и попытка вычленить что-то
одно.\hfil\break
\indent
Но это опять же требует времени и достаточно длительного, ну вот как раз вот 100 дней тот
минимум. Всё-таки на это
лучше ориентироваться. На самом деле, вот этот временной промежуток, он известен, ты, как
психолог, это знаешь
прекрасно, освоить какую-либо профессию или стать даже чемпионом мира в какой-то виде
спорта, можно при... там,
звёзды, если сошлись правильно, там и за три года подготовки ты можешь стать.\hfil\break
\indent
Но стать действительно настоящим мастером всё-равно требуется 10 лет примерно. Так же как и в
освоении любой
специальности. Или в науке. Да, это хорошо знают те же самые медики, когда этот интерн
обучается. То есть все равно
институт, плюс это практика, аспирант, интерн, все равно 10 лет примерно. Тогда ты можешь стать
как врачом.\hfil\break
\indent
Также вот в освоении китайской медицины как нам рассказывал доктор Цзян ну если у тебя есть
какой-то учитель какраньше был это хорошо показано в фильме Куросавы «Красная борода»
какой-то такой учитель и приезжает к нему
ученик получивший европейское образование он с этим учителем общается фильм такой тяжелый,
но его стоит
посмотреть красава снят если есть такой учитель туда там все быстрее, а если такого учителя
нет как обучая обучаются
китайской медицины нормальные врачи кто хочет освоить как нам говорил дзян ты закончил
институт начинаешь
практику то есть тебе приходит больной ты не знаешь что с ним делать ты записываешься на
курсы, повышение
квалификации, при каждом это есть, и разбирается этот вопрос, осваиваешь его, лечишь
больного, приходит следующий
с непонятным, и так 10 лет, тогда есть у тебя шанс стать нормальным врачом китайской
медицины.\hfil\break
\indent
Традиционно, имеется в виду, но на Западе я понимаю примерно то же самое.

\A
То же самое все, да.

\T
Поэтому ориентация хотя бы, ну сложно конечно 100 дней там соблюдать всевозможные правила и
все прочее, но мы
ниже поговорим. Можно сначала там допустим хотя бы поделать там месяц каждый день. Ключевые
слова --- {\itкаждый день},
оно вот в этом имеется в виду. То есть я постоянно, я как бы постоянно в этом потоке. То
есть заниматься надо не
сразу час, а потом 2--3 дня отходить от этой практики, 5 минут, но каждый день в этом
ключе.\hfil\break
\indent
Вот. 5 минут, но каждый день, да. День пропустил, как правило, если день пропускаешь, ну вот
когда я делал эти цигуны,
там день пропускаешь, значит добавляешь 3 дня, 2 дня пропускаешь, добавляешь 10 дней, 3 дня
пропускаешь, значит
все по-новой начинаешь. Вот такие условия. И это тебя обгоняет в рамки то, что ты назвал
самодисциплиной.\hfil\break
\indent
Это очень важно. Потому что по-разному осваивать все эти методики, кого-то очень быстро и
легко получается, и в этом
засада. То есть у меня была эта проблема. То есть, ты легко осваиваешь, и потом тебе
становится неинтересно, хочется
дальше, а это практиковать там рутину одно и то же. И в результате ты топчешься на месте, ты
все новое схватываешь,
осваиваешь, а старое не прорабатывается.\hfil\break
\indent
Опять же, времени не хватает. А когда надо делать одно и то же длительное время, то тогда ты
приучаешь себя, и в этом,
ну, ключ к мастерству, то, что китайцы называли гунфу, и называют гунфу. Неважно.

\null
\centerline{\bf Вопрос про Увэй.}
\null

\A
Вот смотри, про увэй люди много читали, про увэй сейчас, как говорится, из каждого
«матюгальника» звучит. Недеяние,
пустотность и так далее.
Основной вопрос вот чтобы на практике где это проявляется как понять есть увы нету вы как
понять это в тренировках
потому что у нас люди не только цигуном занимаются кто-то тайчи занимается кто-то там футбол
хоккей ну то есть
разные виды так сказать телесных практик в том числе и спорт и и вне тренировок, где это тоже
проявляется, не знаю, в
еде, в общении с людьми. Что такое увэй на практике? Не как философия, а прям вот в жизни.

\T
Значит, в жизни самый простой пример, который мне близок. Ты садишься в лодку, тебе надо из
точки, А приплыть в
точку Б, условно говоря, какое-то время. Течение тебя несет, и ты не сопротивляешься этому
течению. Ты отдал своему
течению, но периодически ты веслом себя корректируешь направление. Ты знаешь, что течение
туда донесет, но ты
периодически веслом регулируешь направление, чтобы тебя там на камни куда-то не закрутило,
куда-то не прибило к
берегу.\hfil\break
\indent
Ты выравниваешь, контролируешь. И всё, что тебе надо делать это не сопротивляться этому
течению ну вот это есть в
моем понимании то есть и не плане цигун или прочее ты делаешь создаешь условия для
правильной практики там тело
соблюдаешь детали, но не ожидаешь какого-то явного результата то есть ожидаемого результата
ты его не планируешь
то есть не моделируешь себе что я должен получить то-то, то-то, то-то --- нет.\hfil\break
\indent
И подгоняешь в результате своё... сознание же у нас мощно работает, да, это только кажется, что
так оно. Вот. Если ты
планируешь результат, а потом ты его вдруг не получаешь, тот результат, который тебе сказали
или который ты ожидал, а
получаешь совсем другое.\hfil\break
\indent
Если ты очень жесток, то есть нет тебе этого непротивления, то ты в результате либо не
получаешь практики той, которая
должна, либо не ловишь то, что нужно. Поэтому ты создаешь условия для правильной практики,
наблюдаешь и
отслеживаешь детали, но не планируешь результат ожидаемый. Ты не знаешь, что ты получишь в
результате. Второй
пример самый простой. Ты создаешь условия, сажаешь дерево или там семя какое-то, Взрыхляешь
почву, поливаешь
его, но не тянешь этот росток, не ожидаешь, что там вырастет.\hfil\break
\indent
Ты можешь не знать, какой семья ты посадил, какой день у тебя там вырастет. Ты просто растешь,
создаешь условия,
оно растет само по себе. Поэтому самая основная задача в цигун и практике вот этих вещей, это
создать правильное
намерение, создать условия, а дальше ты только наблюдаешь и не вмешиваешься.\hfil\break
\indent
И изредка корректируешь вот эти условия, потому что ну мы в мире находимся, влияние среды
внешней это может там
ну как бы создавать дискомфорт особенно если мы пока не привыкли к этим вещам, ну вот так.

\A
Очень яркая метафора про лодку, а можно ли так сказать, что если река, на которой находится
твоя лодка, течет не в ту
сторону, тогда ищи другую реку. Не надо пытаться грести там, умирать от усталости...

\T
Это кстати мне в свое время, мне было 17 лет, мой знакомый, сильно повлиявший на меня очень
умный тут известный в
городе каратист был, он мне так объяснил есть направление, да, тебе надо попасть, ну почему-то
ему там Мурманск
приспичило и тебе надо идти и прийти из Ленинграда тогда еще в Мурманск, но ты сидишь и
ничего не делаешь и никуда
не идешь, я ничего не могу сделать, ты сидишь, ну ты пошел и вдруг ты пошел не в Мурманск, а
пошел в Москву.\hfil\break
\indent
Я могу догнать тебя тогда и палкой тебя загнать и повернуть в Мурманск. Я тогда могу и
поправить тебя. А если ты
ничего не делаешь, нет. Поэтому если ты сел в другую реку, она течет в другую сторону. Это все
равно уже начало
работы. Можно объяснить, что она не туда течёт.\hfil\break
\indent
Ты не попадешь в пункт В, а ты попадешь там в пункт С, а тебе туда не надо. Ты решил другую вещь,
поэтому надо
пересесть либо в другую реку, либо там корректировать и смотреть какие-то пути, поэтому это
да, безусловно. Я же
сказал, что Париж закрыт, но мне туда не надо. Да, да, да. Отсюда правильный метод, это вот об
этом, то есть
правильный метод, это вот как раз об этом, надо сесть именно в ту реку, которая тебя ведет, ну
это если это твоя река,
если это тебе надо.\hfil\break
\indent
Кто-то хочет стать мастером там, я не знаю, тайцзицюаня, а кто-то хочет супер-пупер здорово
играть на скрипке.
Методики могут быть похожи по внутреннему содержанию, но сам метод, он разный. Обучение там и
метод с телом, и так
далее, и так далее.\hfil\break
\indent
Хотя общее можно найти безусловно.

\null
\centerline{\bf Вопрос про баланс тела и ума.}
\null

\A
Окей, хорошо, следующий вопрос связан вопрос с тем, что у нас в основном обучаются люди,
которые работники
интеллектуального труда, то есть люди, которые привыкли, что в жизни главное это мозги, то
есть мозгами деньги
зарабатываются, мозги нужно тренировать и мозги всегда главные, то есть как бы тело
вторично, не в смысле что оно
плохое, просто оно должно за мозгами следовать.\hfil\break
\indent
И вопрос такой, от практик которые люди изучают, внезапно тело начинает брать на себя
лидерство, то есть тело
начинает совершать какие-то движения, а мозг оказывается не главный, мозг только наблюдает
за этим, от этого у
людей возникает там у кого-то паника, у кого-то напряжение и так далее.
Вот вопрос, нужно ли чтобы всегда был баланс тела и ума или где-то тело может опережать,
где-то ум может опережать,
как вот этот вопрос?

\T
Понятно, значит здесь естественно должно быть сотрудничество, но когда мы начинаем обращать
наши мозги, особенно
очень умных, интеллигентных людей, мозги обращают внимание на тело, тело естественно
оживает, оно начинает жить
своей жизнью, оно и так живет своей жизнью, как бы там всякие умные люди не
кричали о том, что они вот всё у них от мозгов, ну они во-первых спят, во-вторых они ходят в
туалет, в-третьих им
требуется еда, в-четвёртых какие-то проблемы со здоровьем, и когда жрать нечего, особо там не
до духовности, когда ты
там голодаешь, это известный такой факт, поэтому это всё такое, снобистское такое, это самое,
вот, когда мы начинаем
какие-то практики делать и позволяем телу стать свободным оно естественно проявляет свой
характер безусловно
здесь кстати работа с настоящим оружием да в ушло оно как раз в какой-то степени
символизирует сотрудничество с
одной стороны мы не позволяем допустим взяли в руки меч мы не позволяем мечу управлять нами,
но с другой стороны
мы позволяем телу жить своей жизнью.\hfil\break
\indent
Поэтому, что я хочу сказать, лучше всего, допустим, есть экипаж, да, запряжены лошади и карета,
это тело. Есть кучер,
это, скажем так, который управляет этим делом, условно говоря, энергией, а есть главный, кто
направляет и дает
задание, куда ехать. Это уже будет тот, кто управляет всем. Это, условно говоря, я условно
Такое пример обычно писали в
начале 20 века про оккультные книжки все время, когда было это модно.\hfil\break
\indent
Но баланс он необходим, поэтому если мы очень сильно ударяемся мозги, то тело начинает
разрушаться.
Соответственно, Кастанеда в своих книгах это сравнивал,  словами Дона Хуана, с луковицей.
Луковица начинает
рассыпаться. Она раскаивается, и в йоге этот фактор известный, да, когда кундалини
поднимаются, жалит мозг, самадхи
наступает, нирвана после самадхи, 40 дней... если ты не возвращаешься на землю, - ты уходишь из
этой жизни. Ну, ты
просветлился, и что тебе здесь делать? А тело разрушается. А что там, мы пока не знаем, да, на
этот вопрос мало кто
ответит, а мы должны вернуться.\hfil\break
\indent
Ну, мастерам боевых искусств  проще, тот же Уэсиба, когда  испытал просветление, тот же
Гагеншмидт, один из силачей,
он испытал просветление, благодаря гирям. Но они возвращались к обычной жизни, этот
Гагеншмидт продолжал гири
тягать, а Уэсиба там выступать и обучать боевому искусству. Спустился с горы и начал с людьми
общаться.\hfil\break
\indent
Да, да, да, вот это очень важный фактор, а если ты туда ушел, ну и чего? Ну ушел ты туда, что ты
для мира, кто, и кем ты
туда ушел, нет, безусловно, там какой-то момент наступает, ты должен покинуть тело по
восточной философии, вот, как-
то там, перейти в другую комнату, на новый этап, вот, поэтому очень важно, но здесь просто надо
этот факт учитывать.
Главное, чтобы не возникало напряжение, уходить от любых панических каких-то настроений и от
восторженных
настроений тоже.\hfil\break
\indent
И не ждать никаких...

\A
Крайние эмоции мешают, да?

\T
Да-да-да. Вот. Это как раз, вот увы, мы не должны ждать. Сегодня у меня тело затряслось,
заколебалось, если я завтра
начинаю практиковать и ждать снова, а вот тело должно... Оно же у меня вчера вибрировало, а
почему оно сегодня не
вибрирует или вчера у меня даньтянь горел, а я сделал все то же самое, а у меня сегодня
даньтянь, и мне холод вот это
как раз уже выход из этой увы, то есть я планирую результаты ожидания, а оно может не
соответствовать моему
психотипу и состоянию на данный момент, конечно если рядом там великий какой-нибудь учитель
Будда Бодхихарма,
может быть он может объяснить мои проблемы и совет, ну таких не каждому везет, таких вроде
как рядом нет, поэтому
очень такой немаловажный фактор, не вдаваться в крайности, ничего не ждать, но самое главное,
как восторг и радость
и крайние проявления эмоций, как ты правильно сказал.\hfil\break
\indent
Проявление эмоций надо избегать, так же как и паническое, это нормально, мы начали
заниматься телом, естественно
оно начинает оживать, вчера я не чувствовал как я и какой у меня позвоночник, сегодня я
почувствовал и испытываю по
этому поводу радость. Эмоции должны быть обязательно положительные, но не планируемые, не
ожидаемые. Потому
что сегодня мне будет тепло, и у кого-то там наоборот тепло, а у кого-то холод.\hfil\break
\indent
Важно, чтобы были какие-то ощущения. И даже если их нет, тоже ничего страшного. Рано или
поздно они какие-то
появятся. Но тело --- это тот инструмент, который позволяет нам быть в реальности, а не уходить
куда-то в иллюзии,
потому что ум любит скакать, планировать, ну, классику читать, там, Мертвые души, они
построили мост из Петербурга в
Москву, и государь-император нас там встречает, ну, вот это как раз вот об этом.

\A
Как китайские традиции относятся к сновидениям?
Насколько это связано с цигуном или это вообще не про цигун?

\T
Нет, но сновидения тоже могут меняться. Здесь я на вопрос не отвечу. Во-первых, в китайской
традиции есть роман, по
которому, по изучению которого целый институт работает: "Сон в красном тереме". Как раз об
этом, Да, и вообще-то в
даосской традиции есть знаменитое выражение... "Мне вчера приснилась бабочка, и я там
подумал, то ли бабочке
приснился Чжуанцзы, то ли Чжуанцзы приснилась бабочка".\hfil\break
\indent
Вот, вот, скажем, вот так. Но поскольку это такая тема, я не уверен, что в разных культурных
традициях сны одинаково
трактуются. То есть всё-таки культура, этот фон, он влияет на мир, на человека. И поэтому я не
уверен, что трактовка
снов там У Юнга и трактовка снов там какого-нибудь православного человека или даже не
православного, а советскогоили русского человека, она будет не одна и та же.\hfil\break
\indent
Поэтому тема такая тёмная. То, что сновидения могут быть и влияют, но поскольку я с ними особо
не работал, я бы к ним
относился с такой осторожностью, потому что, опять же, фаза глубокого сна и серьезного
такого уединения в самадхи,
даже в йоге это известно, описано --- это сон без сновидений, он присутствует, это как раз вот
тот сон.
Единственное, что с точки зрения сна надо ложиться спать там до 23 часов по нашему времени вот
с 23 до где-то часу
вот это время когда надо.

\A
Но это золотые часы сна мы в это в книжке в нашей это обсуждаем.

\T
Да да да если ты не владеешь практиками серьезными уже вот практиками сидения изменения
состояния потому что
эти часы они также вот этот переход и Ян, то есть 23 часа, 5--7, 11--13 часов и 17--19 часов. Если
ты владеешь в эти
практики, понимаешь, что надо делать, то тогда ты можешь и не спать.\hfil\break
\indent
Это серьёзные практики для изменения. Но вот 3--4 часа --- это как раз тот час, когда мы можем
поменять очень
серьёзную продвинувшуюся в результатах.
Но это если ты владеешь, ты можешь как бы заменить сон поскольку если серьезно практикуешь,
то есть момент
наступает, солнце сокращается и сновидения такие, они могут быть как яркими, а могут быть и
вообще без сновидений, а
могут быть наоборот очень яркими то, что у меня, но здесь с работой со снами я тут не очень
компетентен поскольку
такой темы серьезных исследований китайских именно я не встречал. Хотя вот опять же целые
институты по этому
роману. "Сон в красном тереме" называется.

\null
\centerline{\bf Питание и цигун.}
\null

\A
Значит со сном понятно, что не совсем понятно, а что с едой? Вот много рекомендаций люди
читают в литературе по
еде, по поводу того, как связана еда с цигуном. Опять, мы в нашей книге про это говорим, но вот
твоё мнение коротко.

\T
Думать про еду, не думать... Опять же, если ты практикуешь серьезный активный цигун, который
серьезно начинает
запускать и ускорять метаболизм твой, то есть движение крови и ци, ну как вот мы
практиковали, я там, Рома и прочие,
то избегаешь всего, опять же, всех крайностей, избегаешь острого, очень острого, очень
солёного, очень, там, очень-
очень-очень.\hfil\break
\indent
То есть стараешься проходить посередине, то, что мы говорили, вот так же и в эмоциях, ты
избегаешь крайностей. Ну, а
если выработано, уже ты серьёзно практикуешь, как-то продвинулся в и сотрудничать с телом, то
здесь, если за
запущенное чувство тела, то ты ешь то, что хочется. Да, но то, что хочется, но опять же разумно,
если тебе там хочется
чипсов или пива выпить, ну, наверное, можно сделать вывод, что тебе, наверное, может быть на
данный момент это
надо, если ты считаешь, что со силой работаешь.\hfil\break
\indent
Но если тебе каждый день начинает хотеться пиво, то, наверное, надо уже подумать, что тут
что-то не то.

\A
Или сладкого каждый день хочется.

\T
Да-да. Опять же, ну, прежде всего, избегать рафинированных, очень обработанных продуктов. К
сожалению, современное
производство продуктов, оно немножко не туда идёт, неестественное. И поэтому говорить
сложно, поэтому надо
стараться выбирать продукты более-менее естественные, без каких-то сильных обработок, с
добавками ЕР
всевозможными, вот это всё надо избегать, с добавками там всевозможных усилителей вкуса,
всего прочего, да, это
сейчас сложно выбрать, но всё равно читать из чего, что состоит, возможных, Хотя понятно, что
и обманывают и все
прочее, но все-таки как-то контролировать и доверять своему вкусу, более-менее натуральная
еда должна быть.

\A
Опять обманывать проще человека, который не разбирается и даже не пытается разобраться.

\T
Ну я проще, я там беру, там берешь там тот же хлеб, берешь там, если там много всевозможных
добавок, ну его нафиг,
берешь там просто есть рожь, соль, вода, то нормально. Хотя сейчас такого сложно найти,
понятно почему, сейчас
промышленность ускоряется, надо больше продавать, естественно добавляют всевозможные, те
же самые, если ты
покупаешь там какой-то йогурт, даже детский, проще сделать его самому, купить там цельного
молока, добавить каких-то
палочек все прочее чем все продукты ну как-то как вот так любое если углублен уже в практике
цигун то есть мы
говорим остальные любые как бы слезки то есть не знать что алкоголь нельзя можно алкоголь
там все проще но за 100
дней ты сильно погружен в практику она достаточно сложной чисто это психологически, ты 100
дней делаешь одно и то
же, как правило, дважды в день, и добавка каких-то возбудителей, типа алкоголя, много чеснока,
допустим, или лука, того
же самого, это чрезмерная нагрузка на нервную систему, на возбудимость, и ты просто можешь
вылететь из этой
практики.\hfil\break
\indent
Внимание на нормально чтобы эта еда была как можно с моей точки зрения подвержена менее
всевозможные там
обработки, но естественно не есть паспорт там чипсы и все прочее прочее, но если очень
хочется то можно потому что
это иногда организм может потребовать то есть типа вот такого-то есть не надо себя гонять в
очень жесткие рамки, но
все-таки следить то есть не наедаться на ночь (варенье печенье и всего прочего) и заменять
еду одними я конечно
утрировал.

\A
Но точно так же вот как например есть вот эти все так называемые жесткие требования там
левая рука должна быть
поверх правой, а потом смотришь на китайского мастера он наоборот делает спрашиваешь его
почему у вас наоборот он
говорит мне так захотелось.

\T
Всё достаточно просто. Требования такие есть. Ну, допустим, в современном цигуне у мужчин,
как правило, правая рука
сверху. Если мы берём алхимию, более такие древние, то у мужчин левая рука сверху. Моё мнение
--- всё очень просто.
Ты, не думая о ней, находясь в состоянии увы и сложил руки, у тебя ляжет та рука, которая нужна
на данный момент. Ну,
вот и всё. Поэтому всё достаточно просто, также как есть мы начинаем живот растирать вроде
как сначала против
часовой стрелки, а вроде другие говорят.\hfil\break
\indent
По часовой стрелке, если ты опять в неком состоянии увы, то куда оно повернулось туда и
повернулось.
Потому что есть методики по кишечнику сначала по часовой стрелке и потом против, а потом
опять по по часовой
стрелке, поскольку по часовой стрелке у нас лучше заканчивать это движение кишечника, проще
там освоить, поэтому
такие факторы нет и опять же разные, допустим у индусов правая рука считается нормальной,
левой нечистой рукой, а у
китайцев может быть это наоборот, там левая Ян, правая Инь и так далее. И опять же вот в эти
все методики как говорил
мой учитель не надо лезть; сами китайцы в этом не очень хорошо разбираются и найти учителя по
движению вот этой
энергии как по фэншую и в Китае (это 92 год) было достаточно сложно то в этом серьезно
разбирается кстати Ван Сян
чьего это дело ну в те года, А в современном Китае сейчас, наверно, найти совсем это очень
сложно, поскольку в
современном Китае все семимильными шагами идёт к товарно-держичным отношениям и в этих
делах всё продаётся,
всё покупается.\hfil\break
\indent
Я не говорю, что этого нет, есть, и найти можно, но очень сложно. Как говорил Мастер Ян Фанши,
найти боевое тайцзи,
настоящее тайцзи в современном Китае, это десятый год, очень трудно. Можно, но очень сложно.

\null
\centerline{\bf На какую математику опирается практика цигун?}
\null

\A
Вот мы постоянно, когда объясняем упражнения, говорим там повторяем 9 раз, повторяем там 5
раз, или значит там 5 стихий, 8 сил.
Вопрос, вот эта математика к ней нужно привязываться или условно сделали девять раз, но
можно и 9 и 8 и 10 и ничего страшного насколько важна вот эта математика выполнения упражнений
ну во первых сразу могу сказать ничего.

\T
Страшного если ты не делаешь серьезной практики потому что сам подсчет количество
упражнений --- это контроль сознания. Ты этот счет и количество повторов, ты начинаешь себя не
контролировать, а осознавать, что ты делаешь. И сам подсчет концентрирует, ну, концентрирует
не очень хорошее слово, но собирает сознание. Ты осознан.\hfil\break
\indent
И поэтому ты делаешь, а вроде, а сколько я сделал? Да не знаю. Это такой удел мастера вообще-то,
который делает сколько ему хочется, то есть это подвинутый. Когда начинающий тоже так же
себя ведет, это скорее он не понимает, что делать, он не осознан. Поэтому подсчет упражнений
не важен. 5 сил, 8 триграмм и все прочее --- это такие философские термины, так сложилось для
удобства описания систем.\hfil\break
\indent
Ну, нашей там европейский x y z да почему так принято ну вот принято так и все были бы другие
буквы, но ничего страшного, но принятые x y z у них вот такая классификация она имеет там
философских по тексту объяснено того сказать там, а одно рождает 2 2 рождает и так далее все
объяснено
углубляться не будем в эти философские ну вот вот так вот на самом деле там конечно для
удобства 5 сил синий там как правило этим ну не только в синих других системах традиционных
там тумбе и гуага они используют такой же описание это способ описания начальных этапов
усилий всевозможных генерирование сил усилий 8 сил тайцы 8 энергии там 3 грамм это тоже
способ описания На самом деле, в развернутых 72 описания, а то и больше этих энергий по
возрастанию.\hfil\break
\indent
А это такие вот самые простые, самые начальные, чтобы ввести тебя в систему. Ну и плюс дань
традиции, что ли. Люди любят все классифицировать, но в начальном этапе подсчет упражнений
не нужен.\hfil\break
\indent
Почему 9, 18, опять же традиционные числа такие, китайские, Ну и считается, что там 9, некоторые
делают там 4, по 6, по 3 раза в каждую сторону, нечетное число, вроде как нечетные это янские
числа, 12 раз повторяют некоторые упражнения, разная система по-разному, ну как правило, да,
кратно 3, кратно~9.

\A
Здесь уже кстати кстати может еще такое быть объяснение что китайское мировоззрение оно
построено на двоичной и шестнадцатеричной системах подсчета то есть у нас десятичная
система у них 2 и 16 то есть вообще у них там трота такая может быть троичная может быть, но
просто возможно с этим связано что типа 3 --- это то, что тебе непривычно, и ты должен
специально концентрироваться, чтобы считать до 3-х, разные варианты.

\T
На самом деле сложно делать, если мы делаем какое-то движение боевое, то есть до 3-х повторить
достаточно легко телом, а вот сделать сразу 4-е, 5-е --- уже сложно. То есть на тройке тело
начинает застревать, и повторить нам делают усилия, и повторить уже... Я могу хаотично делать
там хоть 10 раз, но это не то.
Я имею ввиду когда ты осознанно генерируешь силу, осмысление и все прочее, особенно в
применении парах ты делаешь.

\A
Можно ли сказать, что минимум тогда лучше четыре раза делать?

\T
Ну 4--5 опять же больше трёх, если мы тренируем применение, а осознанный ответ во
вразумительный, разумный, ну, я имею ввиду, ответный хатак. А так вообще в этом плане есть
такой Девятов, по-моему, да, он писал 90-е книги, он вроде как был агентом наших, там, Китая,
написал много книг о китайской философии.
В 90-х годах, ну, сейчас на ютубе есть, но он, как правило, такой пьяненький там делает, Вот,
выступает, иногда чушь несёт, он обратил внимание на то, в 90-х годах, что в этом плане
китайская философия --- это всё-таки тройка. То есть, инь-ян и объединение, земля-небо-человек.
Триада. Да, и в этом плане похоже на православную традицию русскую, древнюю.\hfil\break
\indent
Бог-отец, Бог-сын, Бог-дух-святой. Вот в этом плане вот тоже троица единая система то синий ян
и их объединение да то есть получаем мы великий предел это объединение имя не оно работает
поэтому тут я не знаю там двойка двойка это скорее как раз европейский взгляд основанный на
европейский то есть 1 2 понятно добро зло бог дьявол и как бы вот этого нет там тема свет
православные и как бы в китайский там есть нюансы взаимодействие с ними вместе да то есть ну
как бы в этой философии борьба до и взаимодействие противоположности ну вот вроде как можно
вот на это свестись такой телен.

\null
\centerline{\bf Полезны ли эмоции в практике цигун?}
\null

\A
Вот раз мы заговорили про восемь сил... Тут такой вопрос у людей, кто практикует уже тайчи.
Значит, в тайчи у нас есть пэн - отражение; есть люй... и у людей возникают такие ассоциации что
пэн это какая-то агрессия, а люй это значит какая-то мягкость, уступчивость... Вопрос:
насколько полезны эмоциональные привязки?
И вообще, насколько полезны эмоции проживать и испытывать, когда мы отрабатываем эти
состояния - или лучше эмоции отодвинуть?

\T
С эмоциями надо быть осторожно, еще раз, потому что любая эмоция --- это волна в сознании, она
чрезмерная, особенно пока неконтролируемая эмоция, она может как помочь движениям и делам
эмоционально, знаменитый этот самый фильм «Остров дракона», где Брюс Ли там ученику,
мальчику, и говорит дело эмоционально, это да, это всё правильно, наполняешь эмоцией, но ты
эмоции наполняешь, когда движение твое освоено, и тело ты полностью, тело и сознание
контролируешь, потому что если ты раньше это эмоции, оно будет туманить мозги, и ты в
результате не увидишь реального действия, и просто не сможешь действия выбрать. Поэтому
первый этап --- это всегда спокойствие. По поводу восьми сил --- это деление такое условное,
чтобы показать разную направленность и разную работу сил.\hfil\break
\indent
Основа всегда --- это {\it пэн}. {\it Пэн\/} не как техническое действие, а {\it пэн\/}
как наполненность, {\it пэн\/} как...
Романов, когда пытался переводить книгу по Чэнь Тайчи, со слов Мастера Вансяна, писал так:
это как вот вода, которая держит корабль. Она качается и в то же время постоянно его
подпирает.\hfil\break
\indent
Это некая такая упругость. Как техническое действие --- это направление силы от себя вперед
вверх. То есть направленная вверх. Она может быть как мягким, может быть и как и жестким. ЛЮЙ
--- это сила пропускания. То есть мы пропускаем силу вокруг себя, не впускаем ее внутрь, а
уводим вокруг себя.
Это тоже быть может быть как мягким так и жестким то есть я могу очень жестко вытянуть
человека и рукой ему сломать локоть и закрутить ну и также со всеми силами.

\A
Могут использоваться как в агрессивном режиме как известно человек может умереть от того
что его водоворот просто затянет на дно. Вроде бы мягкая сила но...

\T
Смертельное вот так же как достаточно и оно может состоять из двух действий как мы выбираем
себя и выпускаем силу вниз как правило сейчас описывают это раз и толчок туда же входит в
движение, но на самом деле они это себя вперед вниз он тоже может быть как мягким массаж ты
делаешь массаж там в системе ты массируешь То есть ты как раз применяешь очень часто прием
{\it ань}, когда не руками, а всем телом --- это {\it ань\/} как раз.\hfil\break
\indent
Но он может быть и жёстким, очень серьёзным таким делом. Вот, а толчок --- это уже несколько
другое действие. Вот, поэтому каждая сила может быть использоваться и так, и так. Понятно,
вот. Поэтому... Но оно всё равно вот в этом состоянии наполненности {\it пэн}, ты не смятый, ты как
бы вот наполнен вот этой упругости и этой мягкости, подвижности.\hfil\break
\indent
Ну, скажем так, если брать это к цигуну конкретно, ты как бы позволяешь энергии Вселенной, то
есть ты расширяешься до энергии Вселенной, а потом ее впускаешь, и ты объединяешься с
энергией Вселенной.\hfil\break
\indent
И у тебя дыхательная практика, она становится, ты как бы всё время высвобождаешься, ты всё
время развязываешься. Узлы, которые в тебе, в силу там, что мы живём в этом мире, они каждый
день там завязываются, какие-то эмоции и всё прочее. Если ты достаточно уже контролируешь
своё тело и сознание, то эмоции, они могут помочь, если они контролируемы в слабом виде, они
помогают.\hfil\break
\indent
Опять же, ты как психолог, лучше меня это знаешь, как слабый стресс. То есть, что значит слабый
стресс? Когда человек может с ним справиться, он наоборот положительный. А если он
чрезмерный, он разрушает. Это прекрасно понимали еще в хатха-йоге Прадибхики, это 15 век нашей
эры, там такие слова написаны, что Хатха-йога подобна сети, которую сильный зверь разрывает и
становится ещё сильнее, а слабый погибает в этих сетях.
Так вот это жёсткая практика хатха-йоги ну в общем мы говорили что это
вот надо расслабиться и это не если человек не готов --- он погибнет.

\bye
