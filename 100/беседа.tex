%&12pt
\pdfpagewidth=297mm
\pdfpageheight=210mm
\pdfhorigin=1in
\pdfvorigin=0pt
\input QUIRE
\shhtotal=\pdfpagewidth
\htotal=.5\shhtotal
\vtotal=\pdfpageheight
\shoutline=0pt
\shstaplewidth=0pt
\shcrop=0pt
\shfootline={}
\shthickness=.27mm
\quire{20}

\horigin=9mm
\hsize=\htotal \advance\hsize by-2\horigin
\advance\hsize by-\QUIRE
\output={\ifodd\pageno\else\hoffset=\QUIRE\fi \plainoutput}

\vorigin=3.8mm
\vsize=\topskip \advance\vsize by37\baselineskip

\footline={\raise1pt\line{\hss\tenrm\folio\hss}}

\font\speakerF=omssbx10
\def\A{\item{\speakerF А.}} % Андрей Ширай (Андрей)
\def\T{\item{\speakerF У.}} % Анатолий Григорьев (Учитель)
\setbox0=\hbox{\speakerF А.\enskip}
\parindent=\wd0

\A
Почему именно 100 дней практики? Откуда это пошло?

\T
Сразу же начнём с последнего пункта. Если мы чем-то занимаемся, должно пройти какое-то время,
чтобы в теле, в
психике начались какие-то изменения. Есть разные... Православные традиции --- это, как обычно
40 дней. Более
короткий путь, там, в китайской традиции, это может быть и 50, и 70 дней.\hfil\break
\indent
То есть, должно пройти какое-то время, чтобы в теле начались... Под телом я подразумеваю
совокупность
психофизических факторов человека, а не только просто физическое тело. Должно пройти время
изменения. Почему 100
дней, я пытался выяснить, но этот вопрос достаточно тёмный.\hfil\break
\indent
По старым писаниям, это привязано к лунному циклу, как раз там 4 примерно лунных месяца,
отсюда такое время,
наиболее такой меньший вариант, что ли, плюс 100 дней тебе дают крепкий фундамент, ты
начинаешь... Это достаточно
длительное время для человека, делать одно и то же 100 дней, это требует некоторых серьёзных
усилий для людей,
которые не занимались, некоей стабильности. Человек начинает делать одно и то же сто
дней.
\hfil\break
\indent
Сам термин «гунфу» --- мастерство, полученное в результате... Если расшифровывать иероглифы,
мастерство, полученное в
результате затраты сил, усилий достаточно больших и серьёзных. Ну, в плане для каждого
человека эта серьезность и
большие усилия, они разные, естественно. Для одного, там, постоять в стойке 5 минут это уже
проблема, а для другого час
что-то поделать никакого труда не составляет. Но здесь серьёзность в том, что мы это делаем
каждый день в течение
трёх с половиной месяцев, да, если 100 дней брать. Тогда можем наблюдать какие-то
изменения в теле, но
приравнивать 100 дней... В общем, это нумерология китайская. Серьёзных работ у нас по этому поводу
не было. Писал там
Богачихин, переводил, на эту тему, нумерологию. В 90-е годы выпускал брошюры когда всё переводили.
Но когда мой
учитель по Ян увидел что я читаю он спросил через переводчика, что он читает, перевели, он
сказал, что в самом Китае
мало кто это понимает.\hfil\break
\indent
Ну, традиционно 100 дней. Есть ещё, там, Бодхидхарма 9 лет сидел, пялясь на стену, понимаешь, и
такие цифры. Суть в
том, что «гунфу» --- это мастерство, полученное в результате затрат сил и времени.
Поэтому первое, что мы
должны понимать: чтобы чем-либо заниматься, мы должны иметь правильный метод.\hfil\break
\indent
Затрат на освоение этого метода и практике этого метода --- силы, и немаловажный вопрос ---
время. Если у нас этот
фактор есть, мы можем получить гунфу. Поэтому все эти байки про освоение боевых искусств
крестьянами, которые
должны там в полях работать были каждый день и добывать хлеб на пропитание --- это
байки.\hfil\break
\indent
Хотя исключения, конечно, были в истории Китая.

\A
Как раз потому, что у крестьян времени на это не было, да?

\T
Да, да, да. У меня, помню, ученица, когда первый раз побывала в Индии, которая у нас считается
религиозной такой, она
говорит, да, религиозная, все там школьники бегают, но это на праздниках они бегут, а там у них
нет времени у простого
народа отдаваться религии и духовному росту. Поэтому это требует времени и плюс правильный
метод, то есть знать как,
знать что, когда, знать какие силы мы должны приложить. И время, это немаловажный
фактор.\hfil\break
\indent
И очень часто, допустим, если так отклоняться, китайская медицина или традиционная, та же
индусская медицина --- Аюрведа, где основной постулат это
лечение посредством диеты. Но
это требует времени.\hfil\break
\indent
Вот правильный метод и время. Естественно если там, на поле боя, как прописано в древних
текстах аж там, 4 века до
нашей эры, ранили кого-то стрелой --- там нет времени его лечить диетами и всё прочее. Надо
оперативным путём
вытащить стрелу, остановить кровотечение.\hfil\break
\indent
Сейчас,
поскольку времена изменились,
у нас тоже нехватка времени.
Отсюда практика, которая противоречит вообще восточным практикам. Бегун бежит, вроде
наматывает круги, тут же у него смартфон, наушники в ушах, и он слушает там какую-то музыку
или развлечение, может ещё английский язык учит.\hfil\break
\indent
Но это противоречит полностью всем восточным практикам, да и не только восточным --- христианским
духовным практикам, где полная отдача внимания и собранность на том, что ты
делаешь. Физически. То
есть единение тела, ума и внутренней твоей энергетики. Под этим мы подразумеваем
все
составляющие твоих полей
гравитационных, электромагнитных и все прочеё.\hfil\break
\indent
То, что в Китае описывается словом «ци» у нас неправильно
переводят как «энергия». Ци ---
это совокупность всевозможных психофизических составляющих полей: дыхания,
движения крови, какие-то обменные процессы, метаболизм.\hfil\break
\indent
Это требует времени.
Традиционно в Китае так сложилось, что минимум времени, которое необходимо затратить для
освоения какой-то практики --- 100 дней. Это тот минимум, который
позволяет увидеть
какие-то результаты в теле и какие-то изменения.\hfil\break
\indent
Причём они часто, из моего опыта, ты практикуешь, 100 дней делаешь и вроде
ничего не происходит, а окружающие, особенно которые тебя видели до этой практики и после,
они замечают какие-то изменения.
Поскольку это во времени так растянуто, изменения могут быть для практикующего и не
очень очевидны, но со
временем потом вдруг --- раньше он не мог это сделать, а потом вдруг почему-то делает.
Это
как стрелка на часах:
вроде смотришь --- она не двигается. Отвлёкся --- а она уже на полчаса куда-то пробежала, время
поменялось. Поэтому сто дней это немаловажный такой фактор --- это именно время.
Оно выступает вот той
четвертой составляющей
в описании пространства и играет большую
роль.\hfil\break
\indent
В йоге немножко по-другому меряют. В йоге не сто дней длятся практики овладения,
хотя там знают о таких временных промежутках. В традиционной
йоге, как духовной
дисциплине, когда осваивают какие-то асаны, то жгутся ароматические палочки, свечи.
Как правило, ты
принимаешь какую-то
позицию и в течение того времени, пока горит эта свечка, примерно два с половиной часа
ты должен в этой
позе находиться.
И когда ты эти два с половиной часа... Понятно, что там без раскрепощения, расслабления,
подготовки ты не выдержишь,
не сделаешь это. Но если ты это делаешь, то считается позиция освоена и ты больше её не
практикуешь. Ты её
практикуешь только тогда, когда необходимо какое-то воздействие на психо-физику, там,
лечение, ещё что-то там
усилить, углубить и так далее. Но она остаётся с тобой уже на всю жизнь. Поэтому если ты
проходишь правильную практику в течение
ста дней и закрепляешь результат, то ты больше её не делаешь.\hfil\break
\indent
Понятно, что если ты лет десять не в теме, то тебе понадобится какое-то время, чтобы выйти на
тот
же самый уровень, но
оно будет в разы короче, чем пока ты осваиваешь. А если ты ведёшь нормальный образ жизни,
продолжаешь
практиковать, то ты не теряешь этот навык.\hfil\break
\indent
Поэтому время --- очень важный фактор. Одни за сто дней могут чего-то достигнуть, другим
может потребоваться больше.
Это время, которое необходимо, чтобы необратимые изменения, в данном случае
положительные изменения, которые
система требует, они начали работать.\hfil\break
\indent
Наверное, есть какие-то гении, которые в других жизнях практиковали. Им достаточно метод
попробовать, и у них всё сразу получается. Это гении. Но в расчёт это брать не следует.
Моцартом можно
восхищаться, но
обычный музыкант, конечно, должен трудиться и работать, работать, пока не выйдет на
какой-то уровень.

\A
Понятно. Можно ли сказать, что вот с точки зрения такой практической, сто дней это во многом
требование для
самодисциплины? Понять что-то можно и за день, и за неделю, но самодисциплину,
привычку что-то делать, нужно вырабатывать.

\T
Да, понять можно и за пять минут, а освоить уже потребуется больше времени --- час-два.
Но чтобы наступили
изменения, то есть чтобы это стало твоим устойчивым навыком, на уровне
уже рефлексов,
потребуется длительное время. То есть когда это уже твоей сутью становится.
Кому-то 50--40 дней, кому-то 100 дней, а
кому-то и больше, не
обязательно 100 дней. Сто дней, это если используется правильная методика, правильная тренировка,
правильное понимание.
Без этого никак.

\A
Есть такая шутка среди преподавателей. Называется «Ещё один всё понял».

\T
Понять-то можно, а вот ввести это в тело, чтобы это стало работать, требуется
время. Иногда очень
большое время, и ста дней может не хватить. Особенно, если сам изначально метод, он такой,
не очень понятный. А
даосская методика, она как раз, если брать вот такие традиционные методики, они как раз очень
расплывчатые. Для нашего
мышления очень сложно зацепиться там за что-то. Мы привыкли к каким-то опорам, нам должно
логически быть понятно.
Ну, вот в такой парадигме мы выросли, а там этого нет.\hfil\break
\indent
Вроде как так, а вроде как и не так. Вроде как должно быть жарко, а вроде как и холодно.
Вот в этом вся сложность. Очень противоречивые вещи. Поэтому
сегодня так, а завтра
по-другому. И вот как тут...?\hfil\break
\indent
Когда учитель есть, он объясняет. И даже очень часто ничего не объясняет --- подразумевается, что
учитель это сильная личность, которую можно просто копировать
на подсознательном уровне, живя рядом с ним.
Ты как-бы копируешь его
модель поведения.\hfil\break
\indent
В древней Индии ррахманы обучали чтению гимна именно вот таким
образом. То есть все
знали в данной деревне, в данном городе вот такой есть мастер брахман, который считается всеми
признанным духовным
лидером, учителем с большой буквы.
Ему отдавали детей соответствующие сословия на 5 лет, на 25 лет. И ученик просто жил как
слуга в доме, как член
семьи, в доме учителя. Об этом вот китайская школа, да, это вот именно об этом. Это семья.
И человек там
вроде жил, делал какие-то задания... Кстати, такая же система была в эпоху возрождения.
Всякие мастерские художественные. Все эти великие художники. Такая же идея.
Или изготовление оружия.
Или физики XX века. Выполняли работу, копировали
мастеров, а потом
в нужный момент, в нужное время когда учитель считал нужным, он давал ученику какие-то
небольшие, но секретные,
условно говоря, наставления, и ученик осваивал эту всю тему.\hfil\break
\indent
Очень часто они такие неуловимые. По цигун как правило приводят
такой пример, Neoglory приводит в одной из программ такой пример: изготовление статуэток.
Типа терракотовых воинов и прочих статуэток глиняных.
Прежде чем их обжечь... То есть, там вылеплено всё правильно, не сложно научиться относительно
скопировать, но чтобы
придать живой вид им... Там мастер показывал ученику пример --- похлопать надо по щекам
лёгкими хлопками, сделать
чуть-чуть такую маленькую улыбку и только потом в печь отправить, и фигурка уже выглядела
другой, она была живой.
Вот такого рода секреты. До них можно дойти и в наших условиях, не имея такого рода учителей,
но это требует
определённой умственной работы, то есть собирание разных практик и попытка вычленить что-то
одно.\hfil\break
\indent
Но это опять же требует времени, и достаточно длительного. Как раз 100 дней --- это тот
минимум. Всё-таки на это
лучше ориентироваться.\hfil\break
\indent
На самом деле, вот этот временной промежуток, он известен.
Ты, как
психолог, это знаешь
прекрасно. Освоить какую-либо профессию или стать даже чемпионом мира в каком-то виде
спорта, можно при...\ там,
звёзды если сошлись правильно, и за три года подготовки ты можешь стать.
Но стать действительно настоящим мастером всё-равно требуется 10 лет примерно. Так же как и в
освоении любой
специальности. Или в науке. Это хорошо знают те же самые медики. Когда этот интерн
обучается... То есть всё равно
институт, плюс практика, аспирантура, интернатура, все равно 10 лет примерно. Тогда ты можешь стать
врачом.\hfil\break
\indent
Также и в освоении китайской медицины. Как нам рассказывал доктор Цзян, если у тебя есть
какой-то учитель, как раньше это было хорошо показано в фильме Куросавы «Красная борода».
Какой-то такой учитель и приезжает к нему
ученик, получивший европейское образование. И он с этим учителем общается. Фильм такой тяжелый,
но его стоит
посмотреть. Если есть такой учитель, то да, там всё быстрее, а если такого учителя
нет... Как обучаются
китайской медицине нормальные врачи, кто хочет освоить. Как нам говорил Цзян, ты закончил
институт, начинаешь
практику. К тебе приходит больной и ты не знаешь что с ним делать. Ты записываешься на
курсы повышения
квалификации, где разбирается этот вопрос. Осваиваешь его, лечишь
больного. Приходит следующий
с непонятным... И так 10 лет. Тогда есть у тебя шанс стать нормальным врачом китайской
медицины.\hfil\break
\indent
Традиционно, имеется в виду, но на Западе я понимаю примерно то же самое.

\A
То же самое всё, да.

\T
Поэтому это такая вещь, она известная везде.
Ну сложно конечно сто дней соблюдать всевозможные правила и
ограничения.
Можно сначала допустим хотя бы поделать месяц, каждый день. Ключевые
слова --- {\it каждый день}. Я постоянно в этом потоке.
Заниматься надо не
сразу час, а потом 2--3 дня отходить от этой практики. 5 минут, но каждый день --- в этом
ключ. День пропустил, значит добавляешь 3 дня, 2 дня пропускаешь, добавляешь 10 дней, 3 дня
пропускаешь, значит
всё по-новой начинаешь. Вот такие условия. И это тебя вгоняет в рамки --- то, что ты назвал
самодисциплиной.\hfil\break
\indent
Это очень важно. Потому что по-разному осваиваются все эти методики. У кого-то очень быстро и
легко получается, и в этом
засада. У меня была эта проблема. То есть, ты легко осваиваешь, и потом тебе
становится неинтересно, хочется
дальше, а это практиковать, рутину, одно и то же. И в результате ты топчешься на месте, ты
всё новое схватываешь,
осваиваешь, а старое не прорабатывается.\hfil\break
\indent
Опять же, времени не хватает. А когда надо делать одно и то же длительное время, то тогда ты
приучаешь себя, и в этом
ключ к мастерству --- то, что китайцы называют «гунфу».

\null
\centerline{\bf Вопрос про Увэй}
\null

\A
Вот смотри, про увэй люди много читали, про увэй сейчас, как говорится, из каждого
«матюгальника» звучит. Недеяние,
пустотность и так далее.
Основной вопрос, чтобы на практике: где это проявляется, как понять есть увэй, нет увэя?
Как
понять это в тренировках?
Потому что у нас люди не только цигун занимаются. Кто-то тайчи занимается, кто-то футбол/хоккей.
Ну то есть
разные виды, так сказать, телесных практик, в том числе и спорт. И вне тренировок где это тоже
проявляется? Не знаю, в
еде, в общении с людьми... Что такое увэй на практике? Не как философия, а прям вот в жизни.

\T
Значит, в жизни самый простой пример, который мне близок: ты садишься в лодку, тебе надо из
точки А приплыть в
точку Б, условно говоря, через какое-то время. Течение тебя несёт, и ты не сопротивляешься этому
течению. Ты отдался этому
течению, но периодически ты веслом корректируешь своё направление. Ты знаешь, что течение
туда донесёт, но ты
периодически веслом регулируешь направление, чтобы тебя на камни куда-то не закрутило,
куда-то не прибило к
берегу.
Ты выравниваешь, контролируешь. И всё, что тебе надо делать --- это не сопротивляться этому
течению. Вот это и есть увэй в
моём понимании. В плане цигун или прочее, ты делаешь, создаёшь условия для
правильной практики,
соблюдаешь детали, но не ожидаешь какого-то явного результата.
Ты его не планируешь,
не моделируешь себе что `Я должен получить то-то, то-то, то-то.'\hfil\break
\indent
И подгоняешь в результате своё...\ сознание же у нас мощно работает, да, это только кажется, что
так оно. Если ты
планируешь результат, а потом ты его вдруг не получаешь, тот результат, который тебе сказали
или который ты ожидал, а
получаешь совсем другое.
Если ты очень жёсток, то есть нет в тебе этого непротивления, то ты в результате либо не
получаешь практики той, которая
должна быть, либо не ловишь то, что нужно.
Поэтому ты создаёшь условия для правильной практики,
наблюдаешь и
отслеживаешь детали, но не планируешь результат. Ты не знаешь, что ты получишь в
результате.\hfil\break
\indent
Второй
пример самый простой. Ты создаёшь условия --- сажаешь дерево или семя какое-то. Взрыхляешь
почву, поливаешь
его, но не тянешь этот росток, не ожидаешь что там вырастет.
Ты можешь не знать, какое семья ты посадил, в какой день у тебя там вырастет. Ты просто
создаёшь условия, и
оно растёт само по себе.\hfil\break
\indent
Поэтому самая основная задача в цигун и практике вот этих вещей, это
создать правильное
намерение, создать условия, а дальше ты только наблюдаешь и не вмешиваешься.
И изредка корректируешь вот эти условия, потому что...\ ну, мы в мире находимся.
Влияние внешней среды может
как-бы создавать дискомфорт особенно, если мы пока не привыкли к этим вещам.

\A
Очень яркая метафора про лодку, а можно ли так сказать, что если река, на которой находится
твоя лодка, течёт не в ту
сторону, тогда ищи другую реку? Не надо пытаться грести, умирать от усталости...

\T
Да. Это, кстати, мне в своё время, мне было 17 лет, мой знакомый, сильно повлиявший на меня, очень
умный, известный в
городе каратист был, он мне так объяснил: ``Есть направление. Тебе надо попасть...\ ну почему-то
ему в Мурманск
приспичило, и тебе надо идти и прийти из Ленинграда в Мурманск, но ты сидишь и
ничего не делаешь и никуда
не идёшь. Я ничего не могу сделать --- ты сидишь. Вот ты пошёл, и вдруг ты пошёл не в Мурманск, а
пошёл в Москву.
Я могу догнать тебя тогда и палкой тебя загнать и повернуть в Мурманск. Я тогда могу
поправить тебя. А если ты
ничего не делаешь --- нет. Поэтому если ты начал плыть по реке, а она течёт в другую сторону,
это всё
равно уже начало
работы. Тогда человеку можно объяснить, что она не туда течёт,
что ты не попадёшь в пункт B, а ты попадёшь в пункт С, а тебе туда не надо. Ты решил другую вещь.
Поэтому надо
пересесть либо в другую реку, либо корректировать и смотреть какие-то пути.\hfil\break
\indent
Правильный метод, это вот как раз об
этом. Надо сесть именно в ту реку, которая тебя ведёт. Ну
это если это твоя река,
если это тебе надо.
Кто-то хочет стать мастером там, я не знаю, тайцзицюаня, а кто-то хочет супер-пупер здорово
играть на скрипке.
Методики могут быть похожи по внутреннему содержанию, но сам метод --- он разный --- обучения и
метод работы с телом, и так
далее, и так далее.
Хотя общее можно найти, безусловно.

\null
\centerline{\bf Вопрос про баланс тела и ума}
\null

\A
Следующий вопрос связан с тем, что у нас в основном обучаются люди,
которые являются работниками
интеллектуального труда, то есть люди, которые привыкли, что в жизни главное это мозги, что
мозгами деньги
зарабатываются, мозги нужно тренировать и мозги всегда главные.
То есть тело как бы
вторично. Не в смысле что оно
плохое, просто оно должно за мозгами следовать.\hfil\break
\indent
И вопрос такой... От практик, которые люди изучают, внезапно тело начинает брать на себя
лидерство, то есть тело
начинает совершать какие-то движения, а мозг оказывается не главный, мозг только наблюдает
за этим. От этого у
людей возникает, там, у кого-то паника, у кого-то напряжение и так далее.
Вот вопрос: нужно ли чтобы всегда был баланс тела и ума или где-то тело может опережать,
где-то ум может опережать?

\T
Здесь естественно должно быть сотрудничество, но когда мы начинаем обращать
наши мозги...\ особенно у
очень умных, интеллигентных людей мозги обращают внимание на тело, тело естественно
оживает, оно начинает жить
своей жизнью. Оно и так живёт своей жизнью, как бы там всякие умные люди не
кричали о том, что они, вот --- всё у них от мозгов. Они во-первых спят, во-вторых они ходят в
туалет, в-третьих им
требуется еда, в-четвёртых какие-то проблемы со здоровьем. И когда жрать нечего, особо там не
до духовности, когда ты
голодаешь. Это известный факт.\hfil\break
\indent
Когда мы начинаем
какие-то практики делать и позволяем телу стать свободным, оно естественно проявляет свой
характер. Кстати, работа с настоящим оружием в ушу, она как раз в какой-то степени
символизирует сотрудничество: с
одной стороны мы не позволяем, допустим взяли в руки меч, мы не позволяем мечу управлять нами,
но с другой стороны
мы позволяем мечу жить своей жизнью.\hfil\break
\indent
То, что я хочу сказать, лучше всего выражается следующей метафорой.
Есть экипаж --- запряжены лошадь и карета.
Карета --- это тело. Лошадь олицетворяет желания, страсти, эмоции и жизненную энергию.
Кучер представляет собой разум, интеллект, ум, который управляет лошадью.
А есть главный, кто сидит внутри кареты, невидим для стороннего
наблюдателя, управляет всем через кучера, кто
направляет и даёт
задание куда ехать. Этот главный символизирует истинное «Я», Душу, Дух.\hfil\break
\indent
Но баланс он необходим, поэтому если мы очень сильно ударяемся в мозги, то тело начинает
разрушаться.
Соответственно, Кастанеда в своих книгах это сравнивал словами Дона Хуана с луковицей.
Луковица начинает
рассыпаться, она расклеивается. И в йоге этот фактор известный: когда кундалини
поднимается, жалит мозг, самадхи
наступает. Если спустя 40 дней ты не возвращаешься на землю, ты уходишь из
этой жизни. Ну, ты
просветлился, и что тебе здесь делать? А тело разрушается. А что там, мы пока не знаем. На
этот вопрос мало кто
ответит.\hfil\break
\indent
Ну, мастерам боевых искусств проще. Тот же Уэсиба, когда  испытал просветление, тот же
Гагеншмидт, один из силачей,
он испытал просветление благодаря гирям. Но они возвращались к обычной жизни. Этот
Гагеншмидт продолжал гири
тягать, а Уэсиба --- выступать и обучать боевому искусству.
Это очень важный фактор. А если ты туда ушёл, ну и чего? Ну ушёл ты туда, а что ты
для мира, кто? И кем ты
туда ушёл? Здесь надо учитывать тот факт, что --- по восточной философии --- какой-то момент
наступает, когда ты должен покинуть тело, перейти на новый этап.
Главное, чтобы не возникало напряжение. Уходить от любых панических каких-то настроений и от
восторженных настроений тоже.\hfil\break
\indent
Это {\it увэй}, мы не должны ожидать. Сегодня у меня тело затряслось,
заколебалось. Если я завтра
начинаю практиковать и ждать снова --- а вот тело должно... Оно же у меня вчера вибрировало, а
почему оно сегодня не
вибрирует? Или вчера у меня Даньтянь горел, а я сделал всё то же самое, а у меня сегодня
Даньтянь холодный.
Вот это
как раз уже выход из {\it увэй}. То есть, я планирую результаты, ожидания, а они могут не
соответствовать моему
психотипу и состоянию на данный момент. Конечно, если рядом великий какой-нибудь учитель
Будда Бодхидхарма,
может быть он может объяснить мои проблемы и дать совет, но на таких не каждому везёт, таких вроде
как рядом нет. Поэтому
очень такой немаловажный фактор --- ничего не ждать.
Но самое главное --- не вдаваться в крайности.
Как восторга, так и радости~---
крайних проявлений эмоций, надо избегать.\hfil\break
\indent
Это нормально: мы начали
заниматься телом --- естественно
оно начинает оживать. Вчера я не чувствовал как я хожу и какой у меня позвоночник, сегодня я
почувствовал и испытываю по
этому поводу радость. Эмоции должны быть обязательно положительные, но не планируемые, не
ожидаемые. Потому
что сегодня у кого-то будет тепло, а у кого-то наоборот холод.
Важно, чтобы были какие-то ощущения. И даже если их нет, тоже ничего страшного. Рано или
поздно они какие-то
появятся. Но тело --- это тот инструмент, который позволяет нам быть в реальности, а не уходить
куда-то в иллюзии,
потому что ум любит скакать, планировать.

\A
Как китайские традиции относятся к сновидениям?
Насколько это связано с цигуном или это вообще не про цигун?

\T
Сновидения тоже могут меняться. Здесь я на вопрос не отвечу. Во-первых, в китайской
традиции есть роман, по
изучению которого целый институт работает: «Сон в красном тереме». Как раз об
этом.\hfil\break
\indent
А вообще-то в
даосской традиции есть знаменитое выражение: «Чжуанцзы однажды увидел себя во сне порхающей
счастливой бабочкой, которая не знала, что она Чжуанцзы. Проснувшись, он не мог понять: то ли он
Чжуанцзы, которому приснилось, что он~--- бабочка, то ли бабочка, которой приснилось, что она
--- Чжуанцзы».\hfil\break
\indent
Поскольку это такая тема, я не уверен, что в разных культурных
традициях сны одинаково
трактуются. То есть всё-таки культура --- это фон, он влияет на мир, на человека. И поэтому я не
уверен, что трактовка
снов у Юнга и трактовка снов какого-нибудь православного человека или даже не
православного, а советского или русского человека, она будет одна и та же.\hfil\break
\indent
Поэтому тема такая тёмная. То, что сновидения могут быть и влияют... Ну, поскольку я с ними особо
не работал, я бы к ним
относился с такой осторожностью, потому что фаза глубокого сна и серьезного
такого уединения в самадхи,
даже в йоге это известно, описано --- это сон без сновидений.
Единственное, что с точки зрения сна надо ложиться спать до 23 часов по местному времени.
С 23 до где-то часу ---
вот это время когда надо...

\A
Ну, это золотые часы сна. Мы это в книжке нашей обсуждаем.

\T
Если ты не владеешь практиками серьёзными, практиками сидения и изменения
состояния, потому что
эти часы... Этот переход инь-ян, то есть 23--1, 5--7, 11--13 и 17--19 часов. Если
ты владеешь практиками, понимаешь что надо делать, то тогда ты можешь и не спать.
Это серьёзные практики для изменения. Но вот с 23 до 1 --- это как раз тот час, когда мы можем
поменять...\ очень
серьёзно продвинуться в результатах.
Но это если ты владеешь, ты можешь как бы заменить сон.
Поскольку если серьёзно практикуешь,
то
наступает момент когда сон сокращается.
И сновидения могут быть как яркими, так их может и вообще не быть.
Но с работой со снами я не очень
компетентен поскольку на такую тему
серьезных исследований китайских именно я не встречал.

\null
\centerline{\bf Питание и цигун}
\null

\A
Ладно, со сном понятно, или не совсем понятно, а что с едой? Вот много рекомендаций люди
читают в литературе по
еде, по поводу того, как связана еда с цигуном. Опять, мы в нашей книге про это говорим, но вот
твоё мнение коротко. Думать про еду, не думать...

\T
Если ты практикуешь серьёзный активный цигун, который
серьёзно начинает
запускать и ускорять твой метаболизм, т.е.\ движение крови и ци, как мы
практиковали --- я, Рома и прочие,
то избегаешь всего, опять же, всех крайностей, избегаешь острого, очень острого, очень
солёного, очень, очень-очень-очень.
То есть стараешься проходить посередине. Так же как и в эмоциях, ты
избегаешь крайностей. Ну, а
если выработано...\ уже ты серьёзно практикуешь, как-то продвинулся в цигун
и сотрудничаешь с телом, то
если
запущено чувство тела, то ты ешь то, что хочется. То, что хочется, но опять же разумно.
Если тебе хочется
чипсов или пива выпить, ну наверное можно сделать вывод, что тебе, наверное, может быть на
данный момент это
надо, если ты считаешь, что ты с телом работаешь.
Но если тебе каждый день начинает хотеться пива, то, наверное, надо уже подумать, что тут
что-то не то.

\A
Или сладкого каждый день хочется.

\T
Да-да. Опять же, избегать прежде всего рафинированных, очень обработанных продуктов. К
сожалению, современное
производство продуктов, оно немножко не туда идёт, неестественное. И поэтому говорить
сложно. Поэтому надо
стараться выбирать продукты более-менее естественные, без каких-то сильных обработок, без
добавок «Е», всевозможных усилителей вкуса и всего прочего. Вот это всё надо избегать.
Да, это
сейчас сложно выбрать, но всё равно надо читать из чего что состоит.
Хотя понятно, что
и обманывают и всё
прочее, но всё-таки как-то контролировать. И доверять своему вкусу.
Более-менее натуральная
еда должна быть.

\A
Обманывать проще человека, который не разбирается и даже не пытается разобраться.

\T
Ну я проще, там берёшь тот же хлеб --- если там много всевозможных
добавок, ну его нафиг.
Берёшь там просто есть рожь, соль, вода, то нормально. Хотя сейчас такого сложно найти,
понятно почему. Сейчас
промышленность ускоряется, надо больше продавать, естественно добавляют всевозможные...
Если ты
покупаешь например йогурт, даже детский, проще сделать его самому.
Купить цельного
молока, добавить каких-то
палочек и всё прочее.\hfil\break
\indent
Если углублён уже в практике
цигун, то есть мы
говорим о ста днях, ты сильно погружен в практику, она достаточно сложная чисто психологически,
ты 100
дней делаешь одно и то
же, как правило, дважды в день, и добавка каких-то возбудителей, типа алкоголя, много чеснока,
допустим, или лука, того
же самого, это чрезмерная нагрузка на нервную систему, на возбудимость, и ты просто можешь
вылететь из этой
практики.\hfil\break
\indent
Нужно обращать внимание на то, чтобы эта еда была, с моей точки зрения,
как можно меньше подвержена
всевозможным
обработкам.
Ну, естественно не есть фастфуд, там чипсы и всё прочее.
Но если очень
хочется, то можно, потому что
это иногда организм может потребовать.
Не надо себя вгонять в
очень жёсткие рамки, но
всё-таки следить, то есть не наедаться на ночь варенья, печенья и всего прочего, и заменять
еду одними чипсами. Я конечно
утрирую.

\A
Но точно так же вот как например есть вот эти все так называемые жёсткие требования.
Например,
левая рука должна быть
поверх правой, а потом смотришь на китайского мастера, а он наоборот делает.
Спрашиваешь его
почему у вас наоборот, он
говорит ``Мне так захотелось.''

\T
На самом деле всё достаточно просто. Требования такие есть. Ну, допустим, в современном цигуне
у мужчин,
как правило, правая рука
сверху. Если мы берём алхимию, более древние источники, то у мужчин левая рука сверху. Моё мнение
--- всё очень просто:
ты не думая, находясь в состоянии {\it увэй}, сложил руки, у тебя ляжет та рука, которая нужна
на данный момент. Ну
вот и всё. Поэтому всё достаточно просто. Также как есть...\ мы начинаем живот растирать вроде
как сначала против
часовой стрелки, а другие говорят, что надо сначала по часовой стрелке.\hfil\break
\indent
Tсли ты, опять, в неком состоянии {\it увэй}, то куда оно повернулось туда и
повернулось.
Потому что есть методики, например, по кишечнику сначала по часовой стрелке и потом против, а потом
опять по часовой
стрелке. Поскольку по часовой стрелке у нас лучше заканчивать это движение кишечника, проще
там освободить.\hfil\break
\indent
И опять же разные... Допустим у индусов правая рука считается нормальной,
а левая нечистой рукой. А у
китайцев может быть это наоборот.
И так далее. И опять же вот в эти
все методики как говорил
мой учитель не надо лезть; сами китайцы в этом не очень хорошо разбираются.
Найти учителя по
движению вот этой
энергии как по фэншую в Китае (это 92 год) было достаточно сложно, кто в этом серьёзно
разбирается. А в современном Китае сейчас, наверно, найти совсем это очень
сложно, поскольку
современный Китай семимильными шагами идёт к товарно-денежным отношениям. И в этих
делах всё продаётся,
всё покупается.\hfil\break
\indent
Я не говорю, что этого нет. Есть. И найти можно, но очень сложно. Как говорил Мастер Ян Фанши,
найти боевое тайцзи,
настоящее тайцзи в современном Китае, это десятый год, очень трудно. Можно, но очень сложно.

\null
\centerline{\bf На какую математику опирается практика цигун}
\null

\A
Мы постоянно, когда объясняем упражнения, говорим, например `повторяем 9 раз,'
или значит там 5 стихий, 8 сил.
Эта математика, к ней нужно привязываться или условно сделали 9 раз, но
можно и 9 и 8 и 10 и ничего страшного? Насколько важна вот эта математика выполнения упражнений?

\T
Ну во первых сразу могу сказать, ничего
страшного, если ты не делаешь серьезной практики. Потому что сам подсчёт количества
упражнений --- это контроль сознания. Ты считаешь количество повторов, ты начинаешь себя не
контролировать, а осознавать что ты делаешь. И сам подсчёт концентрирует, собирает сознание.
Ты осознан.
И поэтому ты делаешь, а вроде...\ а сколько я сделал? Да не знаю. Это такой удел мастера вообще-то
который делает сколько ему хочется, то есть это подвинутый. Когда начинающий тоже так же
себя ведёт, это скорее он не понимает что делает, он не осознан. Поэтому подсчёт упражнений
важен.\hfil\break
\indent
5 сил, 8 триграмм и всё прочее --- это такие философские термины, так сложилось для
удобства описания систем.
Например, у нас в европейской $x, y, z$. Почему так принято? Ну, вот принято так и всё.
Были бы другие
буквы, ну ничего страшного. Но приняты $x, y, z$. У них вот такая классификация.
Она имеет
философский подтекст, всё объяснено. Даосы...\ одно рождает 2, 2 рождает 3 и так далее...
Всё
объяснено.
Углубляться не будем в эти философские... Ну, вот так. Это способ описания, на начальных этапах,
усилий всевозможных, генерирования силы.\hfil\break
\indent
8 сил тайцзи, 8 энергий, триграмм это тоже
способ описания. На самом деле, в развернутых 72 описания, а то и больше этих энергий, по
возрастанию.
А это такие вот самые простые, самые начальные, чтобы ввести тебя в систему. Ну и плюс дань
традиции, что ли. Люди любят всё классифицировать, но в начальном этапе подсчёт упражнений
нужен.\hfil\break
\indent
Почему 9, 18 --- опять же, традиционные числа такие, китайские.
Ну и считается, что 9...\ некоторые
делают по 4, по 6, по 3 раза в каждую сторону, нечётное число. Вроде как нечётные это янские
числа. 12 раз повторяют некоторые упражнения. Разные системы по-разному.
Как правило,
кратно 3, кратно~9.

\A
Кстати может ещё такое быть объяснение, что китайское мировоззрение оно
построено на двоичной и шестнадцатеричной системах подсчёта. То есть у нас десятичная
система, у них двоичная и шестнадцатеричная.

\T
Вообще у них там может быть троичная...

\A
Может быть, но
просто возможно с этим связано что типа 3 --- это то, что тебе непривычно, и ты должен
специально концентрироваться, чтобы считать до 3-х. Есть разные варианты.

\T
На самом деле сложно делать, если мы делаем какое-то движение боевое, то есть до 3-х повторить
достаточно легко телом, а вот сделать сразу 4-е, 5-е --- уже сложно. То есть на тройке тело
начинает застревать, и чтобы повторить надо делать усилие, и повторить уже...
Я могу хаотично делать
хоть 10 раз, но это не то.
Я имею ввиду когда ты осознанно генерируешь силу, осмысление и все прочее, особенно в
применении в парах ты делаешь.

\A
Можно ли сказать, что минимум тогда лучше четыре раза делать?

\T
Ну 4--5 опять же больше трёх. Если мы тренируем применение, осознанный ответ
вразумительный, разумный, ну, я имею ввиду, ответных атак. А так вообще в этом плане есть
такой Девятов, по-моему, да, он писал в 90-е книги, он вроде как был агентом наших в Китае,
написал много книг о китайской философии в 90-х годах.
Он обратил внимание на то, что в этом плане
китайская философия --- это всё-таки тройка. То есть, инь, ян и объединение.
Земля, небо, человек. Триада. И в этом плане похоже на православную традицию русскую,
древнюю.
Бог-отец, Бог-сын, Бог-дух-святой. Тоже троица. Единая система.
Инь, янь
и их объединение. То есть получаем: великий предел это объединение инь-янь, и оно работает.
Поэтому тут я не знаю... Двойка это скорее как раз европейский взгляд, основанный на...\
то есть 1, 2. Понятно, добро-зло, бог-дьявол, тьма-свет. А
православные и в китайской...\ там есть нюансы.
Как бы в этой философии борьба и взаимодействие противоположностей.

\null
\centerline{\bf Полезны ли эмоции в практике цигун}
\null

\A
Вот раз мы заговорили про восемь сил... Тут такой вопрос у людей, кто практикует уже тайчи.
Значит, в тайчи у нас есть пэн - отражение; есть люй... и у людей возникают такие ассоциации что
пэн это какая-то агрессия, а люй это значит какая-то мягкость, уступчивость... Вопрос:
насколько полезны эмоциональные привязки?
И вообще, насколько полезны эмоции проживать и испытывать, когда мы отрабатываем эти
состояния - или лучше эмоции отодвинуть?

\T
С эмоциями надо быть осторожно, еще раз, потому что любая эмоция --- это волна в сознании, она
чрезмерная, особенно пока неконтролируемая эмоция, она может как помочь движениям и делам
эмоционально, знаменитый этот самый фильм «Остров дракона», где Брюс Ли там ученику,
мальчику, и говорит дело эмоционально, это да, это всё правильно, наполняешь эмоцией, но ты
эмоции наполняешь, когда движение твое освоено, и тело ты полностью, тело и сознание
контролируешь, потому что если ты раньше это эмоции, оно будет туманить мозги, и ты в
результате не увидишь реального действия, и просто не сможешь действия выбрать. Поэтому
первый этап --- это всегда спокойствие. По поводу восьми сил --- это деление такое условное,
чтобы показать разную направленность и разную работу сил.\hfil\break
\indent
Основа всегда --- это {\it пэн}. {\it Пэн\/} не как техническое действие, а {\it пэн\/}
как наполненность, {\it пэн\/} как...
Романов, когда пытался переводить книгу по Чэнь Тайчи, со слов Мастера Вансяна, писал так:
это как вот вода, которая держит корабль. Она качается и в то же время постоянно его
подпирает.\hfil\break
\indent
Это некая такая упругость. Как техническое действие --- это направление силы от себя вперед
вверх. То есть направленная вверх. Она может быть как мягким, может быть и как и жестким.
{\it Люй\/}
--- это сила пропускания. То есть мы пропускаем силу вокруг себя, не впускаем ее внутрь, а
уводим вокруг себя.
Это тоже быть может быть как мягким так и жестким то есть я могу очень жестко вытянуть
человека и рукой ему сломать локоть и закрутить ну и также со всеми силами.

\A
Могут использоваться как в агрессивном режиме как известно человек может умереть от того
что его водоворот просто затянет на дно. Вроде бы мягкая сила но...

\T
Смертельное вот так же как достаточно и оно может состоять из двух действий как мы выбираем
себя и выпускаем силу вниз как правило сейчас описывают это раз и толчок туда же входит в
движение, но на самом деле они это себя вперед вниз он тоже может быть как мягким массаж ты
делаешь массаж там в системе ты массируешь То есть ты как раз применяешь очень часто прием
{\it ань}, когда не руками, а всем телом --- это {\it ань\/} как раз.\hfil\break
\indent
Но он может быть и жёстким, очень серьёзным таким делом. Вот, а толчок --- это уже несколько
другое действие. Вот, поэтому каждая сила может быть использоваться и так, и так. Понятно,
вот. Поэтому... Но оно всё равно вот в этом состоянии наполненности {\it пэн}, ты не смятый, ты как
бы вот наполнен вот этой упругости и этой мягкости, подвижности.\hfil\break
\indent
Ну, скажем так, если брать это к цигуну конкретно, ты как бы позволяешь энергии Вселенной, то
есть ты расширяешься до энергии Вселенной, а потом ее впускаешь, и ты объединяешься с
энергией Вселенной.\hfil\break
\indent
И у тебя дыхательная практика, она становится, ты как бы всё время высвобождаешься, ты всё
время развязываешься. Узлы, которые в тебе, в силу там, что мы живём в этом мире, они каждый
день там завязываются, какие-то эмоции и всё прочее. Если ты достаточно уже контролируешь
своё тело и сознание, то эмоции, они могут помочь, если они контролируемы в слабом виде, они
помогают.\hfil\break
\indent
Опять же, ты как психолог, лучше меня это знаешь, как слабый стресс. То есть, что значит слабый
стресс? Когда человек может с ним справиться, он наоборот положительный. А если он
чрезмерный, он разрушает. Это прекрасно понимали еще в хатха-йоге Прадибхики, это 15 век нашей
эры, там такие слова написаны, что Хатха-йога подобна сети, которую сильный зверь разрывает и
становится ещё сильнее, а слабый погибает в этих сетях.
Так вот это жёсткая практика хатха-йоги ну в общем мы говорили что это
вот надо расслабиться и это не если человек не готов --- он погибнет.

revise previous part and following-----------------------------

Что с воображением?
АШ:
Продолжая тему "жесткого ума". Что делать с воображением?
Оно на пользу, воображение, в ходе практики, - или скорее его нужно притормозить и чтобы оно не вмешивалось?
Учитель:
В свое время, и я с этим согласен, сказал Скалозуб на одной из лекций в 90-х годах... В современном мире воображение -
это такая очень сложная вещь. Почему? Потому что
очень много видеоконтента, мы все знаем как выглядит Чехия, мы знаем как выглядит то или иное там планета, нам все
показано, и поэтому это не идет изнутри, я сейчас объясню,
АШ:
Воображение это что-то, нам уже навязанное по сути?
Учитель:
Это кем-то создано и мы знаем: раньше пьющий человек видел черта. И как Его не рисовали, он понимал, что ему
является сущность той или иной традиции, он видел его. Я примерно такой же вопрос задал Ян Фан Шэну, я его спросил,
что более важно, воображение, образ или движение тела.
Он мне сказал, что образ более важен. Но, тогда я не сразу понял: ситуация такая - тебе дается какое-то действие, ты его
выполняешь.
С включением сознания, мыслей ты делаешь правильно, у тебя возникают разные образы в силу выполнения
упражнений, и дальше ты бежишь к учителю, и тебе он объясняет, что вот этот образ неправильный, а этот правильный...
В зависимости от того, чем ты занимаешься.
Поясню. Ну, делали вы вот это упражнение (показывается движение из комплекса "5 Драконов"), да. Понятно. Образ для
всех. Тёплое масло разливается по всему телу. Образ для здоровья. Он не для боевого искусства. А для боевого
искусства у тебя есть свернувшаяся змея, и она выползает.
Знаешь, это такой образ, к которому надо прийти. Поэтому с воображением надо очень осторожным быть. Но занимаясь
цигуном, Мы должны формировать положительные образы обязательно, начинать и заканчивать обязательно
положительными эмоциями. Обязательно. То есть я представляю, заканчивая цигун или занятие по боевому искусству,
что мастерство вошло в меня, да, ошибки я помню, я отмечаю, и в следующий раз я их исправляю, но я не фокусирую
внимание и образ правильный и положительный.
То есть Ци обильно в Даньтяне, в почках обильно, во всём теле оно обильно. Оно циркулирует легко и свободно. Здесь
каждый сам свой образ может придумать из этой циркуляции свободно. Нет никаких застоев и препятствий. Тело
здоровое, все боли уходят,
болезни трансформируются и исчезают. Тело становится молодым и сильным. И вспомнить себя, когда я действительно
был молодой и сильный, не словами, как в аутогенной практике, а именно почувствовать вот это более важно. Именно
образ почувствовать, не увидеть себя как на ленте в кино, молодым и здоровым. Хотя поначалу это может быть, а
вспомнить именно эти ощущения. И второй момент, связанный с воображением уже практики Цигун, что важно Мы это
делаем так, как будто это и есть на самом деле.
Вот это важный момент, что не делаем это, что это как-то оно вроде как есть и нет, то есть и здесь вот наступает момент
веры в цигун. Если мы делаем и не верим, что это работает, это бессмысленная работа. То есть момент веры должен
быть. Вот отсюда вот такой отличительный отбор учителей, отсюда тайна. Понятно, если человек приходит, пробившись
через кучу всего, он естественно верит и доверяет, уже тому, что ему говорят.
А если он не верит, это не значит, что у него нет сомнений или каких-то там самостоятельных действий, есть. Но он верит
в этот метод, а Дальше начинает копаться, он считает, что это да, это работает, эта школа-то здесь самая сильная, если
что-то не получается, что-то я неправильно делаю, а не из-за того, что эта школа убогая, а какой-то бокс сильнее.
Если он верит в бокс, он считает, что бокс самый сильный, и с этой точки зрения он начинает искать, как сделать его
сильнее, как себя приспособить. А если он не верит в бокс, - а вот это круче, а это лучше, а это так. Поначалу это
сомнения, и сомнения, они нормальные такие. То есть мы выбираем реку, по которой плыть. Ну, если уж мы сели в эту
реку, то есть надо плыть до конца, тогда я могу понять: подходит, не подходит. А если сел, полреки, прошел... "а что-то я
не туда".
Ну, может быть, и правильно вовремя вернуться. Ну, в Хагакурэ, вот в этом кодексе самураев Бусидо там записано. Когда
войны закончились, Сёгун Токугава, все прекрасно, можно было осмыслить этот боевой опыт. И было написано то, что у
нас называется Бусидо в Хагакурэ.
Там написано, что самурай, который ничего не умеет, и самурай, который прошел этот круг, стал боевым мастером, они
подобны друг другу, внешне сложно отличить. Но если ты встал на путь самурая, то все, возврата нет, ты должен его
пройти до конца. Только так можно вернуться, то есть назад ты уже не возвращаешься. То есть, действительно ли ты в
этой реке поплыл, и все, у тебя выбор только один - доплыть до этой точки, другого пути нет, назад ты вернуться не
можешь.
Вот, если встал, то тогда да. Поэтому воображение, мы не даем ему воли, а с другой стороны, от этого воображения, в
плане вот этих положительных эффектов, все должновосприниматься, что это есть на самом деле. И с третьей стороны, в данный момент с этим воображением мы должны
это все практиковать, как будто есть, как будто нет. То есть немножко пофигистски.
То есть это создает определенные жесткости. Ты говорил о жестком уме, когда мы очень погружаемся, а надо то, надо
результат, мы жестко этого хотим, получаем обратный результат. И очень часто достигается, скорее всего, человек,
который ну легко к этому относятся и это вот нормально.


\bye
