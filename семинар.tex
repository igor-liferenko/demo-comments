%&14pt
\pdfhorigin=15mm \hsize=\pdfpagewidth \advance\hsize by-2\pdfhorigin
\pdfvorigin=15mm \vsize=\pdfpageheight \advance\vsize by-2\pdfvorigin

{\bf Способы вытянуть позвоночник среди дня:}

\item{1.} Широкая стойка, стопы внутрь, копчик поджать, макушку вытянуть.

\item{2.} «Пробуждение силы» вытягиваемся всем телом вперед и вверх и на опускании
рук копчик поджать, макушку вытянуть.

\item{3.} Вдох в живот, в рёбра, в ключицы, с руками на затылок;
выдох с вытянутыми вперёд руками, оседанием, поджиманием копчика и
вытягиванием макушки.

\bigskip

\item{$\bullet$} Сразу после цигун не пить --- дать остыть.

\item{$\bullet$} Во время цигун хуэйинь зажать и язык к нёбу.

\item{$\bullet$} Во время вращения носок высоко задирать и привставать, в центре чуть оседать.

\item{$\bullet$} Чтобы не были холодные руки, жарить кунжут без масла (подсаливать).
Это янь. Смородина это инь.

\item{$\bullet$} Бить грушу снующим челноком.

\item{$\bullet$} Отрабатывать выброс (плеть внутренней/внешней стороной ладони, кулак-кистень) с под шагом задней ногой сбоку и в мабу с переносом веса на разноимённую ногу (есть видео в канале).

\item{$\bullet$} Проталкивающий удар в правой и левой стойке (макушка расслабленно тянется вверх).
Применяется в туйшоу: которую руку должны отбивать придержать рукой которой должны будем бить и отводящим движением под мышку ударить рукой которой должны отбивать и потом ударить второй рукой в грудь. Кулаки под углом: рука которая нога сзади, потом рука которая нога спереди, грудь растягивается в разные стороны.

\item{$\bullet$} Кулак руки впереди стоящей ноги бьёт в грудь запрокидывая, пятка ладони сзади стоящей ноги в солнечное сплетение отбрасывает когда человек пытается выпрямиться после первого удара.

\item{$\bullet$} Один локоть толкает выкручиваясь и сжимаясь в кулак, кулак второй руки подставляется к кулаку толкающей руки, локти в стороны, поворот и от ноги бьём вторым локтем.

\item{$\bullet$} Перед собой крутим шар --- удар сбоку головы, отбитие бьющей руки. Разбивается на отдельные этапы: отбитие выставленной руки вправо-влево, просто бить с неактивной второй рукой, но добавлять контроль локтем при развороте. Работает только поясница. Смена рук: рукой которая должна бить отмахиваем руку партнёра и делаем смену стойки. Можно делать одному в движении вперед-назад защита/удар/смена стойки. Отбивать лао-гуном ближе к локтю вниз.

\item{$\bullet$} Два способа защиты от прямого удара:
\itemitem{1.} Выворот плеча.
\itemitem{2.} Когда кидаем мешок картошки, по рёбрам бить задней рукой.
Застываем в каждом положении --- контроль устойчивости.

\item{$\bullet$} Ходить по кругу «беседовать» (которая рука бьёт разноименная нога зашагивает назад???)

\item{$\bullet$} Водить круг руками когда вставлена спичка.

\item{$\bullet$} Обворот вокруг впереди стоящей ноги наружу через стойку дракона со сменой рук, второй меняет руки но стойку не меняет, потом восстанавливает встречную стойку (когда второй наработает то ему можно одновременно).

\item{$\bullet$} При смене рук в туйшоу на месте или в круге какая рука вперёд, такая нога вперёд, какая рука назад, такая нога назад.

\item{$\bullet$} На штанах резинка вокруг голени чтоб штаны по полу не волочились и вытягивались когда удар ногой.

\item{$\bullet$} Эксплуатировать промахи в туйшоу:
\itemitem{1.} Локоть поднять вверх, заломать и с подшагом между ног толкнуть перпендикулярно линии ног.
\itemitem{2.} Руку увести вниз с отшагом и заворотом запястья и продолжая тянуть отшагом второй ногой вывернуть запястье в противоположную сторону приседая.

\item{$\bullet$} Пинки в туйшоу. Которой рукой бьём противоположная нога пинает ребром. У защищающегося ладонь, предплечье, колено в одной плоскости.

\item{$\bullet$} Меридианы ног:
\itemitem{А.} По внешней стороне ноги опускаемся, по внутренней поднимается, проходя по тыльной стороне ступней, внешняя ладонь сверху. На другой стороне меняем ладони. Спина прямая.
\itemitem{Б.} Обводим ноги по бокам сразу с обеих сторон вниз и собираем в треугольник на пупке и проходя по пояснице спускаемся опять. Спина прямая. (Потом в обратном направлении? Смотреть у Борисовой: «12 форм хуньюань» которую скачал с Ютуба и ещё одно платное видео которое записано на флэшку.)

\item{$\bullet$} Ягодицу напрягаем и нога совершает обхватывающее движение.
Где лодыжка спереди и сзади ноги должно быть натяжение --- стопа делается как арка за счёт напряжения ягодицы и вцепляется в землю. Постоянно на этом держать внимание.

\item{$\bullet$} Ладони держать скруглёнными за счёт стремления основания большого пальца и ребра ладони друг к другу.

\item{$\bullet$} Правило: ноги растут от ушей, руки от поясницы.

\item{$\bullet$} Встать к стене и бить кулаком, пяткой ладони и локтем или просто прислонить и держать. Кулак под углом --- руки уже, кулак тыльной стороной вверх --- руки шире. Локтем распирание от другого локтя --- бьющий как подставка; сначала ладонью на ладонь нажимать, потом ладони не касаются.

\bye
