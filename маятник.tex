%&17pt
\pdfhorigin=15mm \hsize=180mm
\pdfvorigin=15mm \setbox0=\hbox{T} \dimen0=\topskip \advance\dimen0 by-\ht0
  \advance\pdfvorigin by-\dimen0
\vsize=282mm \advance\vsize by-\pdfvorigin
\nopagenumbers
{\bf Упражнение «Маятник»}

Лучше всего выполнять это упражнение на песке или мягкой траве. Но если такой возможности нет, пусть песок или траву вам заменит ковер или одеяло, которое вы расстелите на полу. Проследите также, чтобы вокруг вас было достаточно свободного пространства, а поблизости не было твердых или тем более острых предметов --- на случай, если, выполняя
упражнение, вы потеряете равновесие и качнётесь
в сторону, вперёд или назад.

Упражнение нужно выполнять босиком (или по крайней мере без обуви), в свободной, не сковывающей движений одежде.

Встаньте на мягкую поверхность в центре свободного пространства. Ноги поставьте на ширину плеч, стопы параллельно друг другу. Колени чуть-чуть (едва заметно) согнуты --- ровно настолько, чтобы ноги
могли мягко пружинить.

Закройте глаза и начните медленно покачиваться из стороны в сторону, не отрывая стопы от поверхности, на которой вы стоите. Сначала лишь немного отклоняйтесь от вертикальной оси, затем увеличивайте амплитуду до максимально возможной, стараясь не терять равновесие. Сделав несколько покачиваний на максимальной амплитуде, снова уменьшайте её и наконец остановитесь в вертикальном положении.

Откройте глаза. Сделайте несколько встряхивающих движений стопами и кистями рук. Затем передёрните несколько раз плечами, встряхивая их. Сделайте круговое движение плечами назад и свободно
опустите плечи вниз.

Снова закройте глаза и начинайте делать покачивания всем телом вперед-назад, не отрывая стопы от пола. Тело остается прямым, не сгибается в талии. Начинайте с малых амплитуд, постепенно их увеличивая. На стопе можно перекатываться, перенося центр тяжести то к пятке, то к пальцам, но не отрывать стопы от пола. Раскачивайтесь, максимально отклоняясь вперед-назад, следя, чтобы тело было
прямым, и стараясь не терять равновесие.

Затем медленно уменьшайте амплитуду раскачки, пока тело не замрёт на месте. Откройте глаза. Заметьте, как вы стоите. Сейчас у вас свободная спина и правильная осанка. Запомните эти ощущения.

Затем закрепите их. Для этого снова закройте глаза и начинайте раскачиваться вперед-назад с прямым телом, не отрывая стопы от пола. Через некоторое время вы почувствуете, что тело освободилось и начинает двигаться уже самостоятельно. Дайте ему свободу движения, но по-прежнему не отрывая стопы от пола. Возможно, вам захочется делать легкие наклоны вперед-назад, сгибая тело на уровне талии, или вращательные движения бедрами или верхней частью туловища. Сделайте несколько таких свободных движений (все это также с закрытыми глазами). Затем перейдите снова к плавным раскачиваниям вперед-назад с прямым телом. Уменьшайте амплитуду раскачки, пока не остановитесь.

Откройте глаза и медленно, плавно опуститесь на пол, сядьте, затем лягте на спину. Руки вытяните за голову. Закройте глаза. Начинайте медленно растягивать тело, представляя себе, что вас тянут в разные стороны за руки и за ноги. Затем как будто тянут за правую руку и за правую ногу. Затем --- за левую руку и левую ногу. Повторите несколько раз. Вытянув тело как следует, почувствовав растяжение в каждой мышце, расслабьтесь и минуту-другую лежите спокойно. Дышите диафрагмой, для этого положите ладони на живот и дышите, следя, чтобы живот поднимался и опускался. Делайте акцент на выдохе, выдыхайте весь воздух до конца, тогда вдох будет получаться спонтанно, сам собой. Подышите так несколько минут, чувствуя, как тело расслабляется
и вдавливается в поверхность, на которой вы лежите.
Между вашим телом и поверхностью не должно быть свободного пространства. Поясница полностью прилегает к полу.

Затем снова растяните тело и медленно встаньте. Заметьте, как вы стоите. Ваша спина сейчас свободна, а осанка стала наиболее естественной для вас.

Повторяйте это упражнение ежедневно утром или вечером.
\bye
../Downloads/на_ниточке.tex
../Downloads/лёгкость_рук.tex
%&17pt
\pdfhorigin=14mm \hsize=\pdfpagewidth \advance\hsize by-2\pdfhorigin
\pdfvorigin=12mm \setbox0=\hbox{a} \dimen0=\topskip \advance\dimen0 by-\ht0
  \advance\pdfvorigin by-\dimen0
\vsize=285mm \advance\vsize by-\pdfvorigin
\nopagenumbers
{\bf Упражнение «На ниточке»}

Встаньте прямо, стопы прочно упираются в пол, спина прямая. Сделайте круговое движение плечами
назад и свободно опустите плечи. Закройте глаза. Руки, согнутые в локтях, заведите за спину и
упритесь обоими большими пальцами в позвоночник на уровне пояса. Остальные пальцы подогните.
Сосредоточьтесь на ощущении в той точке позвоночника, куда упираются ваши большие пальцы.
Представьте себе, что в этой точке ваш позвоночник прочно закреплен словно бы воображаемым винтом. Весь позвоночник выше этого «винта» держится на нем как на опоре и полностью свободен выше этой опоры.

Опустите руки вдоль тела, сохранив ощущение опоры позвоночника на точке по центру спины. Ощутите
свободно держащийся на этой точке позвоночник. Теперь представьте себе, что ваша голова --- это
воздушный шарик, а шея --- ниточка, на которой он держится. Сделайте несколько легких покачиваний
головой в разные стороны, не теряя этого ощущения. Голова, как воздушный шарик, тянется вверх, при этом она свободна и подвижна благодаря гибкой невесомой «ниточке». Делайте очень легкие, небольшие,
может, даже чуть заметные покачивания головой в разные стороны, чтобы прийти к ощущению легкости и освобожденности шеи.

Теперь опустите подбородок на грудь, свободно уронив голову. Почувствуйте, как растягивается задняя поверхность шеи. Это приятное ощущение, некоторое время оставайтесь в такой позе, чувствуя, как освобождаются и растягиваются мышцы шеи. Затем начинайте медленно перемещать подбородок по груди сначала вправо, чтобы ухо максимально приближалось к правому плечу, затем снова к центру груди, затем влево, приближая ухо к левому плечу, затем повторите еще несколько раз. Назад голову закидывать не надо, движение идет полукругом от левого плеча к правому.

Поднимите голову, сделайте круговое движение плечами назад, руки вдоль тела. Теперь разведите руки в стороны параллельно полу и снова сделайте круговое движение назад. Теперь вытяните руки вверх и сделайте руками полукруг назад. Ощутите, как растягиваются и раскрываются мышцы груди, как освобождаются плечи.

Встаньте прямо, руки вдоль тела. Плечи развернуты и свободно опущены. Почувствуйте, что плечи, шея и затылок освободились.

Если выполнять это упражнение ежедневно, результат будет улучшаться с каждым днём.
\bye
%&17pt
\pdfhorigin=13mm \hsize=\pdfpagewidth \advance\hsize by-2\pdfhorigin
\pdfvorigin=15mm \setbox0=\hbox{T} \dimen0=\topskip \advance\dimen0 by-\ht0
  \advance\pdfvorigin by-\dimen0
\vsize=282mm \advance\vsize by-\pdfvorigin
\nopagenumbers
{\bf Упражнение «Легкость рук»}

Встаньте прямо, руки опустите вдоль тела. Дышите диафрагмой, следя, чтобы при выдохе легкие полностью освобождались от воздуха, тогда вдох будет получаться сам собой, легко и спонтанно.

Несколько раз интенсивно встряхните руками, опущенными вдоль тела, так, чтобы встряхивание охватило всю руку от плеча до кисти.

Затем вытяните руки в стороны и снова интенсивно встряхните. Теперь встряхивается в основном часть руки от локтя до кисти.

Теперь поднимите руки вертикально вверх и интенсивно встряхивайте кисти движениями вперед-назад и вправо-влево.

Свободно уроните руки вниз и снова встряхните.

Несколько раз сильно сожмите и разожмите кулаки и опять встряхните кисти.

Согнутые в локтях руки поставьте перед собой и снова встряхните кисти, чувствуя, как к кончикам пальцев приливает тепло.

Свободно пошевелите пальцами обеих рук одновременно. Сделайте движение, как будто вы играете на
музыкальном инструменте --- только без напряжения, свободно и легко. Выполните любые другие легкие и свободные движения пальцами.

Потрите ладони друг о друга, еще два раза свобод но встряхните кистями и опустите руки вдоль тела.

Вы можете ощутить тепло, тяжесть, приятное покалывание в руках --- это значит, в них началась
нормальная циркуляция энергии.

Выполняйте это упражнение ежедневно, утром или вечером, вместе с двумя предыдущими. После всего комплекса рекомендуется 2-3 минуты полежать на спине на твердой поверхности, следя за дыханием и расслабляя мышцы.
\bye
