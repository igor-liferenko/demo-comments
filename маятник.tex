%&17pt
\pdfhorigin=15mm \hsize=180mm
\pdfvorigin=15mm \setbox0=\hbox{T} \dimen0=\topskip \advance\dimen0 by-\ht0
  \advance\pdfvorigin by-\dimen0
\vsize=282mm \advance\vsize by-\pdfvorigin
\nopagenumbers
{\bf Упражнение «Маятник»}

Лучше всего выполнять это упражнение на песке или мягкой траве. Но если такой возможности нет, пусть песок или траву вам заменит ковер или одеяло, которое вы расстелите на полу. Проследите также, чтобы вокруг вас было достаточно свободного пространства, а поблизости не было твердых или тем более острых предметов --- на случай, если, выполняя
упражнение, вы потеряете равновесие и качнётесь
в сторону, вперёд или назад.

Упражнение нужно выполнять босиком (или по крайней мере без обуви), в свободной, не сковывающей движений одежде.

Встаньте на мягкую поверхность в центре свободного пространства. Ноги поставьте на ширину плеч, стопы параллельно друг другу. Колени чуть-чуть (едва заметно) согнуты --- ровно настолько, чтобы ноги
могли мягко пружинить.

Закройте глаза и начните медленно покачиваться из стороны в сторону, не отрывая стопы от поверхности, на которой вы стоите. Сначала лишь немного отклоняйтесь от вертикальной оси, затем увеличивайте амплитуду до максимально возможной, стараясь не терять равновесие. Сделав несколько покачиваний на максимальной амплитуде, снова уменьшайте её и наконец остановитесь в вертикальном положении.

Откройте глаза. Сделайте несколько встряхивающих движений стопами и кистями рук. Затем передёрните несколько раз плечами, встряхивая их. Сделайте круговое движение плечами назад и свободно
опустите плечи вниз.

Снова закройте глаза и начинайте делать покачивания всем телом вперед-назад, не отрывая стопы от пола. Тело остается прямым, не сгибается в талии. Начинайте с малых амплитуд, постепенно их увеличивая. На стопе можно перекатываться, перенося центр тяжести то к пятке, то к пальцам, но не отрывать стопы от пола. Раскачивайтесь, максимально отклоняясь вперед-назад, следя, чтобы тело было
прямым, и стараясь не терять равновесие.

Затем медленно уменьшайте амплитуду раскачки, пока тело не замрёт на месте. Откройте глаза. Заметьте, как вы стоите. Сейчас у вас свободная спина и правильная осанка. Запомните эти ощущения.

Затем закрепите их. Для этого снова закройте глаза и начинайте раскачиваться вперед-назад с прямым телом, не отрывая стопы от пола. Через некоторое время вы почувствуете, что тело освободилось и начинает двигаться уже самостоятельно. Дайте ему свободу движения, но по-прежнему не отрывая стопы от пола. Возможно, вам захочется делать легкие наклоны вперед-назад, сгибая тело на уровне талии, или вращательные движения бедрами или верхней частью туловища. Сделайте несколько таких свободных движений (все это также с закрытыми глазами). Затем перейдите снова к плавным раскачиваниям вперед-назад с прямым телом. Уменьшайте амплитуду раскачки, пока не остановитесь.

Откройте глаза и медленно, плавно опуститесь на пол, сядьте, затем лягте на спину. Руки вытяните за голову. Закройте глаза. Начинайте медленно растягивать тело, представляя себе, что вас тянут в разные стороны за руки и за ноги. Затем как будто тянут за правую руку и за правую ногу. Затем --- за левую руку и левую ногу. Повторите несколько раз. Вытянув тело как следует, почувствовав растяжение в каждой мышце, расслабьтесь и минуту-другую лежите спокойно. Дышите диафрагмой, для этого положите ладони на живот и дышите, следя, чтобы живот поднимался и опускался. Делайте акцент на выдохе, выдыхайте весь воздух до конца, тогда вдох будет получаться спонтанно, сам собой. Подышите так несколько минут, чувствуя, как тело расслабляется
и вдавливается в поверхность, на которой вы лежите.
Между вашим телом и поверхностью не должно быть свободного пространства. Поясница полностью прилегает к полу.

Затем снова растяните тело и медленно встаньте. Заметьте, как вы стоите. Ваша спина сейчас свободна, а осанка стала наиболее естественной для вас.

Повторяйте это упражнение ежедневно утром или вечером.
\bye
