%&17pt
\pdfhorigin=15mm \hsize=180mm
\pdfvorigin=15mm \vsize=267mm
\nopagenumbers
\font\OM=./OM at17pt
\def\cycle{
  Глубокий вдох, глубокий выдох \par
  Глубокий вдох, глубокий выдох \par
  Вдох и выдох и выдох и выдох и выдох $\ldots$ и {\bf выдох} (3 раза) \par
  Глубокий вдох, глубокий выдох \par
  Глубокий вдох, глубокий выдох \par
  Вдох на три четверти лёгких, задержали дыхание* (30 секунд)\par
  {\OM?} выдох \par}
\vfootnote{*}{сознание в области межбровья, тело расслаблено}
\parindent=0pt
\parskip=7pt
\centerline{\bf КАПАЛАБХАТИ}
\bigskip
\cycle
\cycle
\cycle
Ровное спокойное дыхание, можно расслабить ноги, можно лечь на спину.
Не открываем глаза и чувствуем себя внутри, останавливаем мысли и не думаем ни о чём.
Лежим 1.5--2 минуты. \par
Потянулись. Хорошо потянулись. Встаём через правый бок.
\vfil\eject
\parskip=0pt
Расширяем ноздри, чтобы увеличить прохождение воздуха. С практикой большинство людей могут
контролировать расширение ноздрей. На выдохе ощущение прохождения воздуха должно быть глубоко
внутри ноздрей, а не только на выходе из носа.

Диафрагма должна быть расслаблена как на вдохе, так и на выдохе.

После последнего выдоха глубоко вдыхаем через нос и быстро выдыхаем.

На задержке дыхания выполняем джаландхара-бандху, мула-бандху и уддияна-бандху
(втягивание живота) --- именно в этом порядке.

Удерживаем задержку дыхания и бандхи комфортное время.

Перед вдохом расслабляем мула-бандху, уддияна-бандху, и джаландха\-ра-бандху --- именно в
этом порядке.

Когда голова поднята, медленно вдыхаем через нос.

\bye
