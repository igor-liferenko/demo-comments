%&12pt
\nopagenumbers
\pdfhorigin=15mm \hsize=\pdfpagewidth \advance\hsize by-2\pdfhorigin
\pdfvorigin=14mm \vsize=\pdfpageheight \advance\vsize by-2\pdfvorigin

Мы будем исследовать и тренировать навык пошагового расслабления различных частей тела. Тема
расслабления большая. И если мы хотим пойти глубже
привычных стереотипных «сядь, выдохни, ни о чём не думай» и прочее-прочее, то нужны
определённые приёмы, потому что наше сознание
не очень приспособлено к тому, чтобы существовать в действительно
глубоко расслабленном теле. То есть самый стандартный вариант, когда мы расслабляемся чуть-чуть
больше обычного, мы просто засыпаем. Задачей будет тренировать расслабление, но при
этом по возможности оставаться в бодром состоянии. И это возможно за счет метода пошагового
расслабления.

Например, у нас есть голова. Я сейчас сижу, откинувшись назад, опираюсь спиной на
стенку, можно сидеть на стуле, неважно. Я осознаю положение своей головы,

как напряжение плечи
и затем я просто кладу голову назад наблюдаю что меняется пробую немножко подвигать головой
запрокинуть голову назад больше и наоборот опустить подбородок как бы растянуть шею сзади вот
задача какая задача прочувствовать как разные участки головы разные точки на черепе разные
точки в районе ушей, челюсть и отдельно горло и мышцы шеи как они по-разному ведут себя и
расслабляются когда голова получает опору и в какой-то момент попробуйте самый главный прием
который мы сегодня будем делать, провалиться головой в эту опору. Почувствуйте, что внезапно
голова начинает проваливаться, как будто опора стала мягкой. До этого была твердая стенка, и
как будто голова прямо туда погружается, в эту стенку. Не надо как-то сознательно себе это
визуализировать. Например, ощущение, когда вы ложитесь на подушку головой, и голова
проваливается в подушку. Или когда, например, на песке лежите, голова проваливается в песок. Вы
поймаете вот это состояние мягкого проваливания, то есть опора есть, но она не жесткая. Кстати,
некоторым помогает такое странное положение тела, когда опираем голову на ладонь кажется, что
это неудобное положение потому что рука поднята вверх но странным образом для головы и для шеи
это положение может быть удобным или например вот так даем дополнительную опору, поддержку
рукой рука падает вперед и голова опирается, повисает такое вот это чувство мягкой упругой
опоры значит еще один вспомогательный прием он не основной основным будет все-таки прием
связанный с этим проваливанием вспомогательный прием попробуйте надавить головой на опору то
есть мы просто положили голову и чуть-чуть вдавили её в опору.

Естественно, делаем это
аккуратно, чтобы не было какого-то дискомфорта. Специально, сознательно не провоцируем боль. Мы
об этом тоже вчера в лекции говорили. С болью работаем осторожно. Она так или иначе будет, но
мы ее не провоцируем. Чуть-чуть надавили головой. Надавили и сбросили напряжение. Это
называется метод постизометрии. То есть дали нагрузку и потом убрали ее. Или еще по-другому
прогрессивное расслабление.

В какой-то момент у вас начнется такое растекание чего-то тёплого,
которым смазывается ваше тело или теплая вода которая мягко струей течет по вашему телу.
Некоторое время поработали с головой.

Дальше у нас руки. Рукам нужно найти опору. Хороший вариант
опоры для рук это ноги. То есть вот я сейчас поставил ногу на колено в смысле поставил ногу на
стопу и колено вверх. И вот я положил руку и я исследую как сделать так чтобы опять-таки рука
максимально растеклась и провалилась в опору. Возможно потребуется как-нибудь там повернуть
локоть. Обратите внимание, что кисть свободно свисает. Возможно потребуется как-нибудь наоборот
положить кисть или сделать какой-нибудь вот такой вариант, то есть когда локоть опирается на
колено а кисть ложится куда-то в район плеча. Самое главное, что вы постепенно будете ловить вот
это поначалу непривычное но постепенно все более узнаваемое чувство растекания тела, то есть
рука постепенно растекается. Возможно, она субъективно становится больше, возможно, она
становится, скажем так, мягче. Разные ощущения, разные описания. Но вот рука растекается, рука
становится какой-то более комфортной, более уютной. И то же самое дальше делаем со второй.

Вот у нас рука. Опять пробуем
разные варианты. Кстати, хороший вариант
расслабляться сидя на солнышке. Это прям полезно. Это называется янская
погода, медитация на солнце. Это прямо такая полезная для здоровья форма медитации. Что-то
получилось. Возможно, одна рука расслабится больше, другая меньше. Это нормально. Руки разные.
Полной симметрии у нас нет. Поэтому наблюдаем. И может быть сложным, но очень интересным
заданием расслабить одновременно обе руки. Даосские наставники подчеркивают, что одновременное
расслабление двух рук---это как бы задача не вдвое более сложная, а условно в несколько раз более
сложная. Почему? Потому что расслабить обе руки, это значит одновременно начать расслаблять все
тело, а оно при этом начинает ломаться, оно начинает терять структуру, оно начинает куда-то
падать и прочее, прочее. Потому что расслабление рук сразу ведет за собой перестройку всего
тела. Поэтому еще раз пробуйте с одной рукой. И только когда вы уверенно будете каждую руку
расслаблять, соединяйте руки. Делайте практику расслабления на две руки. Тот же самый момент.
Напоминаю. Два основных способа. Основной и вспомогательной. Принцип рука проваливается в
опору. Дайте руке растечься. Рука это не палка. Рука она мягкая. Почувствуйте, как ткани руки
растекаются по опоре. Почувствуйте, как опора входит в мышцы. Как сухожилия перестраиваются.
Как сосуды начинают по-другому себя вести. Возможно, там какая-то пульсация появляется. То есть
вот это все прочувствуете. Глубокое проваливание руки в опору. И второй вспомогательный метод ---
это надавить на опору. Особенно если проваливания не чувствуете, надавили на опору и потом
расслабили. Надавили и расслабили. То же самое. Надавили немножко и расслабили. Надавили и
расслабили. С руками разобрались.

Аналогично поступаем с корпусом. Или давайте ноги, корпус
оставим в последнюю очередь. Все то же самое с ногами. Для ног нужно найти различные виды опор.
для ног сложная потому что мы на него постоянно каждый день опираемся и у нас уже
сформировалась определенное стереотипное напряжение определенный паттерн как бы как как ноги
стоят на полу это мешает ногам хорошо расслабиться поэтому очень рекомендую экспериментировать
положить ногу на подушку, пробуйте положить ногу на стенку. Мои близкие знают, что одна из моих
любимейших поз отдыха и расслабления это лежать у стенки так, чтобы ноги были вертикально
вверх, прижавшись к стене. Соответственно ноги, задняя поверхность бедер и ягодицы. 25 лет
практикую это упражнение. Такая поза глубокого расслабления и раскрепощения ног. То есть ищите
позы, которые будут удобны для вас. И даже если вы сидите, давайте так, вы в офисе, например,
делаете эту практику, то есть больше у вас нет возможности, но в офисе попробуйте, например,
если у вас стул на колесиках, одну ногу поставить на пол, а вторую ногу поставить на вот эту
стойку кресла, где колесо крепится. То есть хотя бы, чтобы одна нога была выше другой. И
насколько это повлияет на расслабление в ногах. Или может быть даже позволить себе ноги на
какую-нибудь тумбочку положить. на какую-нибудь коробку. Поискать, как ноге по-разному можно
расслабиться. Если вы работаете дома, то смело можете ногу и на стол класть, куда только ноги
не кладутся. Мы когда с семьей ездили на море на машине, мы в дороге были И, соответственно, в
день мы проезжали от 5 до 10 часов непрерывного. И в какой-то момент мы просто поняли, что
очень удобно ехать, положив ноги куда-нибудь на впереди стоящее кресло, на приборную панель. И
даже, что, конечно, очень небезопасно, высунуть ноги куда-нибудь в окно. Прямо на ходу машины я
всегда думал, какие идиоты летом ездят, высунув ноги в окно. Оказывается, это не идиоты, это
люди, которые, видимо, очень долго в машине едут, и вот они поняли, что на Как сказать,
естественный поиск человеком позы для комфортной релаксации разных частей тела. И то же самое
правило. Нога проваливается сквозь опору. Первое и второе. Нога надавливает на опору, а потом
просто расслабляется. У нас остался корпус. на что-то опирать но как правило только одним
способом ну то есть вот условно область лопаток область поясницы у нас на на что-то привыкла
операция вот теперь попробуйте опираться корпусом ну например в районе плеч стеките вот так вот
я знаю что некоторые люди так расслабляют а вот потренируйтесь как расслабиться в этом
положении прямо стеките на стенку или там на спинку кресла вот чтобы на опору давили только
ваши плечи и чуть-чуть лопатки почувствуйте как это или например значит попробуйте Может быть
там ребрами как-нибудь опереться. Или если у вас какая-нибудь большая подушка или кресло
какое-нибудь такое объемное, то прямо вот лягте боковой поверхностью корпуса на вот эту опору.
Ну, в конце концов, можно, конечно, это делать просто лежа на кровати или на полу. Именно на
боку. Прочувствуйте. Там может быть болезненно, там может быть некомфортно. Но вот поищите, как
при этом тело будет расслабляться. Это, кстати, нам отдельно понадобится, когда мы будем
практиковать сон дракона. Потому что сон дракона --- это позиция лёжа на боку. Так что уже можно
готовиться. Ну и, конечно, немаловажно расслабление лёжа на животе. Расслабление сидя. с
опорой, грудью, животом вперед. Кто практикует столбовые практики, знают, что очень полезная
тема подойти к дереву и обнять дерево. Обнять дерево, прижаться к дереву всей передней
поверхностью. Буквально от ключиц и до лобковой кости. Соответственно, вот это ощущение что вы
как бы ложитесь вперед всей передней поверхностью тела и опять-таки растекаетесь это первое что
мы делаем и надавливаете а потом отпускаете давление и просто расслабляетесь я думаю что
сегодняшняя практика это только начало даже не нашего марафона Это только начало многолетней
истории. И, скорее всего, вы каждый раз в ближайшие несколько лет, когда будете делать эту
практику, будете находить для себя какие-то новые моменты. Но если не каждый раз, то регулярно.
Поздравляю, мы отправляемся в это длинное путешествие внутрь себя.
\bye
