%&12pt
\nopagenumbers
\pdfhorigin=15mm \hsize=\pdfpagewidth \advance\hsize by-2\pdfhorigin
\pdfvorigin=14mm \vsize=\pdfpageheight \advance\vsize by-2\pdfvorigin

Мы на этом курсе будем говорить о том,
что позволяет перестроить качество жизни. В терминологии даосских мистиков соединить
сердце и судьбу, соединить син и мин.
То есть соединить что-то в себе, что изначально разделённое. Что-то, что по отдельности.
Надо это соединить, вернуться к изначальной целостности. Значит, если говорить на более
таком простом приземлённом языке, то есть если отказаться от метафор, то на простом языке самое
главное, что мы должны научиться делать ---
следуя вот этим требованиям даосских практик --- {\bf мы должны освоить технику глубокого,
послойного,
детального расслабления всего тела. Еще раз, глубокого, послойного, пошагового.}
То есть большинство методик, которые говорят о расслаблении, как правило, они предполагают...
чуть-чуть расслабиться. То есть вот человек напряженный чуть-чуть расслабился. Скорее всего,
когда человек напряженный чуть-чуть расслабился, его тянет в сон. И человек говорит себе
``О, нормально, я расслабился, немножко отдохнул, вот все, я понял, как работает даосская
практика.'' Ну или какая-нибудь другая медитация там или какой-то метод, неважно.

Вот это как раз проблема, что речь идет не об этом поверхностном расслаблении.
То есть, например, если мы хотим радикально перестроить качество сна, то одним из базовых
требований ко сну в даосских школах, где я проходил обучение, там в закрытых школах,
там, например, требование, что вы легли спать на 15 минут и
заснули, соответственно, почти мгновенно. Легли и через минуту уже спите. А через 15 минут
проснулись и опять бодрые, активные. Ну и занимаетесь необходимыми делами.
В даосских школах, в тех, которые последовательно практиковали управление состоянием
человека, это было базовым требованием. Мне эту практику дали буквально на второй год моих
занятий в закрытой школе у Люпаи. То есть, эта практика считалась обязательной к освоению
всеми учениками. Иначе просто дальше человек не сможет полноценно тренироваться. Потому
что если вы не можете быстро выспаться, то вы будете, то есть вам все время будет не
хватать сил, или вы будете, значит, вялыми, или вы будете перенапряженными.
Для некоторых практик нужно изначально расслабленное тело и так далее, и так далее.

То есть вот у нас получается есть такая первая задача, что
научиться быстро засыпать, качественно отдыхать, восстанавливаться за время этого короткого
сна и, соответственно, легко просыпаться и переключаться на какую-то активность. Соответственно,
возникает вопрос, как этого достичь. У нас будет отдельное занятие, посвященное правильной
технике сна.
Я прям покажу, как там надо ложиться, как выстраивать тело. Но это такая внешняя составляющая.
И опять-таки, внешнюю составляющую вот этого правильного сна, он называется сон дракона,
волунгун. Внешнюю составляющую можно освоить за неделю, за две.
То есть вы просто научитесь лежать в этой позе, выстраивать тело и в какой-то момент там что-то
в теле начнет происходить. Но если ваше сознание изначально перевозбуждено, если ваш ум
неспокойный, не умеет отдыхать, то всё сложно. Мы сейчас говорим именно про мозг, про сознание.

С телом проще. Телом можно научить отдыхать простыми методами. У нас был, например, марафон
столбовой практики.
Люди, которые стояли столбом хотя бы месяц, уже говорили, что они очень сильно расслабились.
Это физическое расслабление. Физическое расслабление доступно за месяц, за полгода практики.
Но научить отдыхать мозг --- это как раз работа, которая занимает несколько лет. И очень
сильно зависит от того, насколько раньше вы привыкли мозгом работать.
Насколько вы привыкли мозг включать и раскручивать его.

Парадоксально, но людям интеллектуально менее развитым даосские практики иногда даются проще.
То, что называется горе от ума. Но опять не все, но некоторые даосские практики действительно
людям с низким интеллектом на начальном этапе бывает легко освоить.
Потому что ум не мешает. Остальным надо {\bf учиться ум замедлять, останавливать.}
Совсем его выключить невозможно. Но как-то {\bf учиться управлять работой ума.}

Для всего вот этого, для того, чтобы мы могли полноценно приступить к практике правильного
сна, к практике контроля ума, {\bf необходимо довести навык чувствования тела до
определенного уровня.} В европейской традиции
если говорить более-менее научным языком, это называется соматическая интроспекция или
соматическая интероцепция. И она должна быть быстрой и детальной.

То есть нам необходимо с одной стороны {\bf развить в себе способность чувствовать
свое тело мгновенно}, то есть без какого-то длинного перехода.
В отношении своего ума мы считаем что это нормально, что 15 минут нужно, чтобы ум успокоился.
Вот задача научиться чувствовать тело сразу, как только вы поняли, что это необходимо,
как только возникла такая потребность. Это потребует определенной тренировки. Это с одной стороны.

А с другой стороны необходимо, чтобы вот это погружение в тело было детальным.
То есть, в идеале, опять-таки, в старых текстах говорилось о том, что вы должны научиться
чувствовать каждую клеточку, каждый сантиметр вашего тела. Каждый волосок, буквально каждый
ноготь и так далее. Не просто, например, почувствовать пульс, а почувствовать, как пульс
распределяется во всем теле.

И при желании можно даже {\bf научиться пульс контролировать, управляемо замедлять, ускорять,
выключать его в разных частях тела и так далее.}
Вопрос --- зачем это делать, но теоретически это возможно. Или, например, {\bf управлять
тонусом мышц.} То есть не просто расслабляться, а управлять расслаблением кончика пальца
на одной руке.
Или {\bf управлять расслаблением только мышц лица}, например, не расслабляя все остальное тело.

То есть вот требуется высокая степень детализации вот этого навыка расслабления и
осознавания тела. А для развития этого детального навыка расслабления и
осознавания тела необходимо {\bf составить детальную карту своего тела в сознании.}
Здесь, конечно, помогают разного рода атласы анатомические, разглядывание каких-нибудь
манекенов, но только отчасти. Если совсем нет представления об анатомии, наверное,
это будет полезно.
Но дальше все равно необходимо, развивать вот эту чистую интроспекцию, то есть навык
самонаблюдения, когда человек может несколько часов, например,
{\bf сидеть и наблюдать за тем, как у него пульс бьется, как у него волосы на теле шевелятся,
как у него зрачок там мигает, моргает, пальцы подергиваются и прочее.}

Необходимо {\bf научиться тренировать и развивать навык сосредоточения на физическом теле},
а не на чистом разуме.
Проблема в том, что на начальных этапах кажется, что совсем нет никакого прогресса.
То есть вы сидите, например, лежите, стоите и наблюдаете за телом. И какой будет прогресс
за пять минут? Никакого.
За час тоже не факт, что будет какой-то прогресс. Скорее вы просто устанете. Потому что еще
человеческое сознание не может, особенно у современного человека, долгое время наблюдать за
чем-то малоизменяющимся. Мы привыкли потреблять большое количество информации. Нам нужно,
чтобы что-то происходило.
Поэтому, когда в теле, ведь оно, конечно, происходит, но не очень явное. И нам трудно за
этим долго наблюдать.
Например, когда я лежу, я начинаю замечать распределение веса по телу. Я начинаю
замечать, как движутся кончики пальцев. То есть тело входит в какое-то такое
состояние невесомости и становится доступно очень много детальной информации о теле.

То есть ключевой момент здесь какой? Вы должны осознанно 40 минут
ничего не делать.
{\bf Ничего не делать. В том числе и не делать, в смысле не пытаться что-то тренировать.}
Потому что что там тренировать-то?
{\bf Вы просто наблюдаете за собой.} При этом нельзя просто отпустить ум, и чтобы он сам
собой что-то делал. Потому что ваш ум тут же начнет о чем-то думать. Вы начнете,
скорее всего, планировать что-то, крутить в голове какие-то мысли.
То есть, когда вы отпускаете ум, он не перестает работать. А здесь, вот в этой даосской такой,
ну это даже не практика, а как бы принцип, {\bf вы сосредотачиваете ум на теле,
но дальше ум ничего не делает. Он просто присутствует в теле.}

И вот это соматическое такое присутствие, телесное присутствие,
Когда вы наработаете десятки часов,
это телесное присутствие переходит в другое качество.
Опять-таки понятно, что это будет в разные периоды дня, это будет в разных ситуациях,
в разных позах. Это нужно делать и сидя, и лёжа, и стоя. Сложнее это делать на ходу.
{\bf Нужно удерживать сознание в теле.}

\bye

Потом я куда-то улетал, ну как минимум 20-25 минут я мог прямо вот
сосредоточиться только на теле и больше ни о чем не думать.
То есть фактически вопрос, как вы сможете эту практику реализовать. Нужно найти какое-то время в течение дня, нужно найти какие-то обстоятельства, где вы будете сосредоточены на себе. Вы не планируете ничего достичь за время этой практики, потому что любое достижение сразу ограничивает. Сразу ум начинает.
опять включаться, и тело до конца не сможет расслабиться. Вот. Значит, то есть вы не планируете ничего достичь, просто вот нарабатываете, нарабатываете, нарабатываете. Даосы говорят это, значит, ну так это и называется, наработка практики или наработка навыка. Вот.
То есть нужно вот это состояние наработать. Проводится аналогия с чаем. Чтобы чай был вкусным, он должен настояться. Чтобы практика давала эффект, она должна наработаться, как чай настояться. Да, понятно, что можно чай из пакетика быстро поболтать, но...
Так же и качество практики будет, что не очень эффективное. Значит, что еще? Ну, наверное, отношение, вот важный момент --- это отношение к боли. Вообще, тема боли --- это тема, так скажем, больная, да?
Такой каламбур. Значит, с одной стороны боль сопровождает наше человеческое существование. То есть, в каком-то смысле, пока мы живем, у нас что-то все время болит. Какие-то сигналы идут от разных частей тела, что что-то где-то не идеально.
С другой стороны, у нас постоянно вырабатывается то, что даосы называли здоровое ци, ну а в научном описании у нас вырабатываются эндорфины, например, вырабатываются вещества, которые боль отключают или делают незаметной. И тогда нам хорошо, нам радостно, нам спокойно. То есть кажется, что боли нет.
Но вот когда вы будете развивать вот эту соматическую интроспекцию, качество вашего восприятия тела повысится. И, возможно, это приведет к тому, что вы начнете чувствовать боль там, где ее раньше не было. И вот на этом моменте многие люди, ну, как бы ломаются, что ли, ну, типа...
Я не хочу терпеть боль. Вот я расслабился и у меня что-то заболело. Значит, эта практика не для меня. Это сложный выбор. Тут нельзя сказать терпи, например. Потому что терпеть тоже неправильно. Если вы будете терпеть, ничего хорошего не произойдет. Нужно найти какой-то компромисс. Возможно, делать вот эти...
погружение в тело не очень длинными. То есть если вы там чувствуете, что у вас тело заболело, то возможно стоит прервать практику и продолжить ее чуть позже. Или, например, вам придется увеличить количество часов физической нагрузки, потому что общеизвестно, что многие болевые синдромы связаны просто с гиподинамией. То есть вы недостаточно двигаетесь,
Причем не вообще, а в конкретных местах. Например, шея болит просто потому, что шея недостаточно получает различных нагрузок в течение дня. Она находится в одном положении, застывшем, и мышцы шеи не работают. Или плечи, если вы за компьютером, запястья бывают парализованные.
То есть вот по сути, когда вы начинаете развивать вот эту соматическую интроспекцию, параллельно потребуется добавить какую-то разминку. Ну, благо разминок мы уже выложили просто на тысячи часов, наверное, общей продолжительности. Может не тысячи, но сотни часов точно. Поэтому...
Можно выбирать любую разминку, которая вам нравится. У нас есть и закрытые материалы, и открытые. В общем, материалов полно. И кто-то из вас занимается в группах с тренером. То есть, можно самим. Но главное что? Что вы понимаете, что если, например, я сел и начал сейчас осознавать тело, то я в любой момент могу прервать вот эту...
практику осознавания для того, чтобы просто начать разминаться. У меня там спина затекает, я начинаю делать какое-то упражнение на спину. Причем я могу сказать, что когда я наблюдал за даоссским таким активом, что называется, люди, которые
энергично практикуют, которые сами инструктора, со всей России съезжалась тусовка. Я могу сказать, что людей, которые слушали мастера часовую лекцию неподвижно, было очень мало. Большинство людей постоянно что-то разминали себе. Потому что вот эта привычка не ждать, не терпеть, когда болит. Спина затекла, тут же разминаем спину.
шея затекла, тут же как-нибудь шевелим шею. Голова напрягается, тут же начинаем себе условно делать какой-нибудь точечный массаж на голове. То есть, да, опять-таки это требует привычки и каких-то навыков, потому что возникает вопрос, а что делать? А вот шея болит, а я не знаю, что с ней делать. Но это уже такой технический момент, это можно себе, не знаю,
блокнот завести и в блокноте писать. Выяснить, что делать, когда болит шея. То есть в какой-то момент вам станет ясно, что вам на самом деле не надо много знать. Вам нужно разобраться с каких-нибудь условно 7 вопросов, которые вам нужно решить в отношении своего физического тела, чтобы оно не мешало вот этой практике глубокого погружения. Понимаете? То есть еще раз
Дорога очень простая. Глубоко погрузиться в свое тело, почувствовать его и в какой-то момент начать управлять его состоянием. Но она простая только на словах. Потому что на начальных этапах объем практики, которую вы будете делать... У меня были годы, когда я делал практику 5-6 часов в день. Фактически...
грубо говоря, не было часа, чтобы я что-то не делал. Я там вел семинары, например, и просто постоянно все семинары я ввертелся, крутился, я на месте не сидел. Возможно, это даже производило какое-то странное впечатление, но это был мой выбор такой, что не хочу терпеть боль, не хочу терпеть усталость какую-то, еще что-то. То есть...
на каком-то этапе количество практики будет очень большим. Это будет отдельный вопрос, как это совместить с вашей жизнью, с вашей работой и так далее. Значит, у меня друзья есть, которые, например, программисты, и они говорят, что когда они там в офисе внезапно начинают практиковать какую-то даоссскую практику, сидя за экраном, то люди незнакомые с этим прям в ужасе разбегаются. Типа,
Странно выглядит, когда человек за компьютером внезапно начинает крутить головой, корчить рожи, дергать себя за уши и так далее. А человек при этом программист. Системный администратор. Должен сидеть спокойно и бить по клавишам. А нет, он не сидит спокойно. Вот этот момент. Либо договориться с окружающими. Либо как-то...
Делать это незаметно, не палиться перед санитарами. В общем, это социальный момент отдельный. И даосы на эту тему часто шутят, что главная даосская практика не палиться. То есть делать практику незаметно, чтобы не привлекать внимание. Потому что когда вы привлекаете внимание, вам уже не дают делать практику. Начинают вопросы всякие.
«Ой, а покажи, а мне тоже интересно», ну и так далее, и так далее. Или там осуждение какое-то, или еще что-то, там сектант и прочее. В общем, отдельный вопрос --- научиться это все совмещать с какой-то социальной жизнью. Потому что вряд ли вы хотите закрыться, грубо говоря, дома, и чтобы вам пиццу под дверь подсовывали, да?
Скорее всего, предполагается какое-то общение с людьми, выход в мир, на улицу и прочее. И вот чтобы это не мешало практике. Это вопрос, как совместить. Что еще нам нужно сказать? Сейчас пока все, что я объясняю, это такая преамбула. Я даю общую настройку. Чем мы будем...
заниматься и на каком фоне мы будем делать то, что будем делать. Дальше будут уже завтра, послезавтра конкретные задания, но эти конкретные задания очень важно будет выполнять с учетом того контекста, который я сегодня обрисовываю.
Так, сейчас что-то еще важное. Да, ну, общие моменты, которые я довольно много объяснял в своих там публичных лекциях и там на канале, и много где. Значит, что касается вот этого образа тела или карты тела. То есть, когда мы говорим о карте тела, почему недостаточно, там, например...
рассматривать анатомические атласы или там есть проект Google Body, когда в интернете можно послойно тело человека снимать кожу, мышцы и смотреть, где что находится. Потому что и недостаточно просто чувствовать себя. Закрыл глаза и что-то там чувствуешь. Потому что есть определенные
критичная картина мира, в которую должно встроиться ваше тело. То есть даосская картина мира, она так устроена, что человек не сам по себе. Ну и это на самом деле привлекательно, что мы начинаем, практикуя даосскую практику, мы начинаем чувствовать свою связь с миром. То есть буквально
значит, там птичка зачирикала, у вас внутри это отзывается. Там, значит, лес зашумел, и вот этот шум леса проникает в вас, и что-то в вас меняет. Ну и наоборот, услышали выстрелы, выстрелы тоже как-то внутри отражаются. Или там кто-то кричит от боли. То есть вот эти моменты, что вы становитесь таким...
Человеком чувствительным к происходящему вокруг. И соответственно, мы отдельно поговорим по поводу информационной безопасности и информационной гигиены. Но что важно? Важно то, что мы приводим внутреннее состояние в соответствии с универсальной космологией, универсальной картиной мира.
Значит, универсальная картина мира в даосской традиции предполагает три элемента. Значит, это знаменитая триада, великая триада. Значит, небо, земля, человек. То есть, всегда есть небо. Ну, сейчас там говорят космос.
Ну вот какой-то принцип. Небо это что? Небо это источник чего-то нового. Небо это источник перемен. Небо это определенные законы, принципы, требования. В человеке за небо отвечает условно голова. То есть вот голова как раз думает.
И это как раз проявление вот этого небесного начала. Соответственно, в идеале, поскольку голова отвечает за принцип неба, она должна быть пустой. То есть, если голова чем-то перегружена, она уже не может воспринимать новое. Но это вроде так логично звучит. Чтобы воспринять новое, надо отказаться от старого. Но это на словах.
Просто, а вот как этого добиться? С другой стороны, значит, голова должна быть пустой, чтобы как раз наблюдать за телом. То есть, когда голова тяжелая, будет сложно наблюдать за телом, потому что значительная часть внимания будет отвлекаться вот на эту тяжесть в голове. Ну, это особенность вот.
человеческого сознания, что то, что происходит с головой, это важно. Поэтому, когда у человека болит голова, то думать о том, что происходит где-то там в Дантиане или в сердце, не до этого. Головная боль все на себя забирает. Задача опустошать голову. Параллельно с практикой этого осознавания мы постоянно опустошаем голову.
мысленно выбрасываем из головы все. С помощью дыхания, с помощью каких-то микродвижений, встряхиваний и так далее. Задача опустошить голову. Ну и там всякие хитрости типа расслабить нижнюю челюсть, расслабить язык, расслабить глаза. То есть это очень сильно помогает в опустошении головы.
Значит, дальше. Противоположность неба --- это земля. Принцип земли --- это принцип опоры, основы, стабильности, спокойствия, чего-то неизменного, что-то, в чем вы всегда уверены. Соответственно, всегда есть земля, всегда есть какая-то опора.
Очень интересный вопрос, как даосские практики будут практиковаться в космосе, в невесомости. Я прямо жду с нетерпением этого момента. Пока никакой информации нет, хотя некоторые китайские космонавты пытались уже практиковать цигун на орбите, но информация такая, какая-то немножко непонятная. То ли ничего у них не получилось, то ли, не знаю. В общем, это вопрос вопрос.
что там будет с этой Землей в космосе, когда человек начнет жить в космосе, посмотрим. Но вот пока мы на Земле, то вот это ощущение стабильности, опоры, это принципиально. И, соответственно, в теле, что отвечает за вот этот принцип Земли, это ноги. Ноги и отчасти таз. То есть, ну, собственно...
тазобедренные суставы. Это то, что соединяет нас с землей и позволяет как раз почувствовать вот эту тяжесть, устойчивость, весомость и так далее. То есть, противоположность головы. Голова всегда пустая, а ноги всегда наполненные. Вы их всегда чувствуете, они всегда есть. И, соответственно, третий принцип.
Вот все, что между головой и ногами, то есть, собственно, корпус и руки включительно, это принцип человека. То есть триада небо, земля, человек. Человек --- это как бы такое соединение неба и земли. Он перерабатывает энергию неба в энергию земли, а энергию земли в энергию неба. Такой коннектор, шлюз, интерфейс, можно сказать.
Соответственно, поскольку между небом и землёй постоянно что-то меняется, в теле человека тоже постоянно меняется. Ну и условно говоря, плечи могут немножко подключиться к энергии неба, и в какой-то момент вы почувствуете, что плечи и руки лёгкие и воздушные. А, например, живот может в какой-то момент набрать в себя энергию земли,
Особенно если вы, например, поели, то живот тяжелый и вот это ощущение тяжести такой. Это как раз нормально. То есть в теле могут какие-то вещи меняться. Не очень нормально, когда, например, голова тяжелая, а ноги пустые. Вот это прям нехорошо. Ну или когда условно лицо горячее, а ноги холодные. Это...
Это указывает на какой-то сильный дисбаланс энергии. И по-хорошему, как только вы ловите...
вот это ощущение дисбаланса, то есть нарушение космического вот этого порядка, все бросаете, как говорится, и начинаете восстанавливать космический порядок. Это важно. То есть в этот момент ни о каком расслаблении речи быть не может. Пока ноги холодные, а голова горячая, нет смысла говорить о расслаблении. Даже если вы
будете расслабляться, это будет неправильное, вредное расслабление. Вот этот очень важный момент, что мы постоянно поддерживаем распределение энергии и в теле, и условно вокруг нас. Поэтому, например, не очень рекомендуется практиковать какие-либо практики в помещениях с низким потолком,
Можно сказать с точки зрения физиологии, что в помещениях с низким потолком воздух не очень хороший. А с точки зрения даосской традиции низкий потолок, значит неба не хватает. Вы будете постоянно чувствовать давление сверху. И соответственно то же самое не очень хорошо практиковать сидя на обрыве.
Ну, потому что под ногами пустота. Можно. Это как специальная такая тренировка будет. Такое делают. Но обычно все-таки вы должны чувствовать под ногами опору. Иначе вот вопрос, что там будет происходить. Вот. Что еще? Что еще? Что еще? Ну, наверное...
Основные моменты я рассказал. На этом фоне, на этом контексте мы будем выстраивать нашу практику на протяжении данного марафона. Я призываю делиться опытом. Как положительным, так и отрицательным. Когда у вас что-то получается...
поделитесь, напишите об этом в чате. Может быть, какой-то вопрос, уточнение. И если что-то принципиально не получается... Вот тоже я заметил, я веду различные занятия уже больше 30 лет, я заметил, что люди очень боятся говорить о проблемах. Я причем, когда спрашиваю, почему... Ну вот раньше это было...
вживую, а сейчас это все в интернет переместилось, и вот я спрашиваю людей, а почему вы не пишете, что у вас проблема? И какие-то очень странные объяснения. Не хочу мешать другим, не хочу портить людям настроение, еще что-то. Но вот постоянно говорю про это, что вы не можете помешать другим или испортить настроение. Это часть обучения. В конце концов...
Если кто-то хочет уйти в себя, можно просто в чат не заходить и не читать его. Или взять себе какой-нибудь индивидуальный пакет и общаться напрямую с мастером. Многие так делают. То есть тут выбор есть. Но чат --- это место, где люди общаются. И то, что вы рассказали, что у вас что-то не получается, это хорошо. Это некая...
Это некий опыт, который можно разобрать и на его основе дальше обучаться. Поэтому, хотя у нас называется марафон тишина, ну и как бы типа, что болтать не надо, но делиться опытом желательно. Это важно для вашей же собственной практики. Наверное, я...
Такой теоретический блок закончил, но практики у нас сегодня не будет, сегодня только теория. Может быть у вас возникли какие-то вопросы, можно их задать или поделиться ощущениями, что вы почувствовали, пока слушали меня.
Игорь, тебя не слышно. Мне сегодня такая мысль интересная пришла. Почему у людей существуют различные установки? Например, в это время...
Такой праздник в это время, такой праздник. То есть это как бы получается планирование в течение года. Почему многие культуры склонны к этому? Это психология как бы так человека побуждает всё это распределять, что вот, допустим, весной такой праздник, зимой такой праздник и так далее. То есть как это с жизнью? Почему человек склонен так всё разграничивать? Хороший вопрос. На самом деле это вопрос, имеющий отношение
к нашему марафону, потому что мы выходцы из традиционной культуры, а в традиционной культуре нельзя было радоваться и получать удовольствие каждый день. Это был естественный такой механизм, то есть нужно страдать, и на фоне страданий, опять в этом есть определенная мудрость, на фоне страданий моменты радости ---
переживались людьми острее и ярче. Понимаете? То есть вот эти знаменитые карнавалы венецианские или всенародные празднества, моменты ликования людей. Раньше, во времена старые, условно, 2000 лет назад, люди, например, могли в том же самом Китае
в рамках какого-то праздника всей деревни заняться групповым сексом. Ну потому что это же всем же хорошо, радостно, ну давайте спариваться. Потому что о чем еще заниматься, когда хорошо и радостно. То есть это был определенный момент регулирования. Мы весь год страдали, нам плохо, мы занимаемся физической работой, мы болеем, умираем, ну давайте порадуемся. То есть вот этот контраст. И на самом деле сейчас...
сложность в том, чтобы научиться как раз радоваться, когда тебе хочется. Не когда тебе говорят, вот праздник, давай радуйся. Я в этом смысле праздники не люблю. Они меня обязывают к чему-то. Я стараюсь их игнорировать, когда это возможно. Но опять-таки социальная жизнь не всегда позволяет игнорировать праздники. Это очень правильный вопрос. Здесь нужно опять найти баланс.
То есть ты можешь быть таким чудиком, который просто живет не в мире людей, а можешь быть полностью социальным. Но найти баланс --- это всегда такой личный выбор. Мы в этом году впервые за последние 5 или 7 лет будем праздновать китайский Новый год. То есть 7 лет не праздновали, но просто как-то
не видели смысла в этом, а решили в этом году отпраздновать. Потому что люди занимаются китайской культурой, вопросы задают. Китайский Новый год, ничего плохого в этом нет. Давайте отпразднуем. Надо понимать, что это некая социальная рамка, которая вас обязывает. И чтобы вы не становились рабом этой социальной рамки. Вот так вот.
Спасибо, понятно. И по лекции всё понятно, тоже очень хорошая лекция. Хорошо. Всем добрый вечер. Я просто хотела сказать, что я поняла, что я действительно туда пришла за тем, что мне нужно.
Буду дать с нетерпением практик и делиться со всеми. Хорошо, Наталья, спасибо большое. Отлично. В общем, если сейчас есть еще какие-то комментарии, выскажитесь. Но еще раз напоминаю, в любой момент,
То есть, грубо говоря, если вы ночью проснулись и вам захотелось что-то написать в чат, смело пишите в чат. На время марафона не будем считать это бессонницей. Еще раз, когда вы практикуете, знаете, иногда проснуться ночью и начать делать среди ночи практику --- это хорошо. Это не то, что у вас бессонница. Просто надо сделать практику. У меня такое было много раз.
когда окружающие такие, что ты не спишь, домашние, близкие, ну, а я пошел тренироваться ночью или там дышать, или там какие-нибудь...
Восьмерки крутить или еще что-то. Тело и мозги так потребовали. На время марафона, если ночью проснулись, можете написать об этом в чате. Это нормально. Как-то так. Что еще? Записи заданий я буду выкладывать...
Во второй половине дня по Новосибирску, то есть завтра, ну и во все дни, где-то примерно после четырех по Новосибирску будет задание следующего дня. Чтобы у вас осталось время и в этот день, и в следующий. Соответственно, опять-таки, если...
вы уже так заряжены и хотите чем-нибудь заняться, за время, которое осталось до первого задания практического, вы можете просто сделать любую практику вам доступную. Столбом постоять или посидеть в тихом сидении, все что угодно. И попробуйте применить этот принцип глубокого послойного погружения в тело и
наблюдение за телом без каких-либо целей и ожиданий. Просто вот это сложно, я сразу скажу. Просто наблюдать это сложно, но как-то у вас это может получиться. Вот, наверное, тогда на сегодня все. Всем спасибо. Я чувствую, что у нас собрался такой
Узкий круг --- это хорошо. Мы не хотели делать массовый этот марафон, потому что тема такая непростая. Поэтому работаем узким составом и будет интересно. Всем успехов. На связи. В следующий раз мы общим составом встретимся с
Последний день марафона уже, когда будем подводить итоги. Все. Спасибо. Спасибо. Хорошего отдыха.
\bye
